\chapter{Conclusions and Future Work\label{cha:conclusion}}
\epigraph{We now accept the fact that learning is a lifelong process of keeping
abreast of change. And the most pressing task is to teach people how to
learn.}{\textit{Peter F. Drucker}}
%2020
The main focus of this thesis has been on improving \LLLs support for students
in the universities by providing them with a better learning environment -- an
enhanced \ep~system. For this purpose, the requirements, design, development and
evaluation of the prototype environment has been discussed in detail. This final
chapter summarizes the outcomes of this research and its contributions to the
related fields, along with the opportunities for the future research projects.

\section{Contributions and Outlook}

Despite the narrow focus of this research project -- developing a technical
solution for supporting \LLLsn, -- its contributions to the field and knowledge
should not be underestimated. Among these were:
\begin{itemize}
  \item Identifying the requirements for the \LLLs in universities based on
  academic literature: the initial contribution of this research was the set of
  identified requirements to provide students with comprehensive support for
  their \LLLs in universities. 
  
  \item Development of a formal specification: Following the needs of the major
  stakeholders (students and lecturers), formal requirements for \LLLs support
  in universities were developed. This formal specification that is based on the
  identified requirements can be used or adopted by the developing teams of the
  \ep~systems to improve existing systems in the context of providing support
  for \LLLsn.
  
  \item Development of the functional prototype: The Mahara \ep~system was
  modified to illustrate that these requirements can be successfully
  implemented. The functional prototype was developed to identify the strengths
  and weaknesses of the developed specification, and was an attempt to address
  primarily learners' needs rather than focus on institutional requirements.
  
  \item Development of a multiple perspective evaluation design: Design of
  a proper evaluation was an important task to understand whether the initial
  research goal was achieved. Due to the nature of \LLLsn, evaluating it within
  the scope of this project was a challenge. A multiple perspective evaluation
  was designed to address this issue. It included conducting separate evaluation
  studies using suitable for each study methods to ensure that the perspectives
  of all stakeholders were taken into account. 
\end{itemize}

All these contributions were aimed at achieving the main goal of this research
to provide a technical solution capable of supporting \LLLs in universities.
Based on the outcomes of the evaluation, the overall approach to the problem can
be considered successful.

However, the real value and impact of this research project goes beyond the
technical implementations. When this project started a bit over three years ago,
there were few active projects in the area of providing students with technical
support for \LLLsn. At that time, a lot of studies from the first wave of an
\ep~system implementation as a tool for supporting \LLLs have already been
finished \citep{Batson2010}. Consequently, many universities around the world
that had been trying to adopt \ep~systems for \LLLs support discovered that
incorporating these systems into institutional environment was not an easy task.
To successfully introduce a new technology, it needs to be aligned with the
views of an institution. It also requires a significant investment of time and
effort as well as adjusting curriculum, teaching and learning in accordance with
the new standards of \LLLsn.

Due to the fact that the changes in academic environment occur slowly,
especially if it is a questions of incorporating new technologies
\citep{Molebash1999}, no necessary conditions existed at that time for an
\ep~system uptake to happen. This might explain why many \ep~adoption attempts
have not passed the stage of pilot projects and the system as such have not got
as much recognition as expected.

Despite these past issues, the developments in higher education over the recent
years showed that important changes are happening. It would not be an
exaggeration to say that many universities are now better prepared and more
ready to address the challenges of adopting new technologies for supporting
\LLLsn. This can be seen in numerous examples around the world, for instance
development of Curriculum 2010 Project as a part of \textit{the Strategic Plan
2009-2013} at Curtin University which was aimed at curriculum renewal to utilise
\LLLs skills in the courses \citep{Oliver2007}; declaring support and expanding
opportunities for \LLLs in 2020 road-map at Massey University
\citep{MasseyUniversity2012}; making official statement to provide students with
tools they need to become lifelong learners at the University of Ottawa in 2020
Strategic Goals \citep{UniversityofOttawa2005}.

In the road-maps for 2020, the European Commission is looking into increasing
participation in \LLLsn, guaranteeing funding for education, implementing \LLLs
policies and supporting partnerships between higher education institutions,
students, and employers \citep{EuropeanCommission2010,EuropeanUnion2009}.

The notion of changes can also be noticed in the recent criticism of exclusively
LMS-based learning in universities and movements towards a new type of systems
called Personal Learning Environments (PLE) which are learner-centred and can
provide support for informal learning \citep{Calvani2007}. In this context, \ep
s are called \textit{the DNA of the Personal Learning Environment} and
considered to be a future of the systems supporting \LLLs \citet{Attwell2007a}.

In light of many universities getting ready to make changes and incorporate
technologies for supporting \LLLsn, this research can be considered increasingly
relevant. The ways \ep~technology will be adopted and used in the future by
universities may impact learning of many generations of students. Based on all
the above described changes that currently occur in higher education, the
contribution of this research would also include informing institutional policy
makers of the role and importance of technology for supporting \LLLsn, ongoing
technical developments in this area, and potential solutions that would help to
improve learner experience.

\section{Future Research and Potential Extensions}
The research presented in this thesis is the first step on the way to providing
students with full support for their \LLLs journey in universities. In the
course of this research, a number of issues and questions were raised which
creates potential for further investigations. Future research should aim at
exploring and solving these issues to support the findings of the current
research and extend the knowledge added. The following sections suggest
potential future studies.

\subsection{Production of the prototype}
Due to the high interest in the outcomes of this project among the research
participants, the next logical step for this research would be moving the
\ep~system prototype to production level. Important part of this step would
include re-assessing outcomes of the evaluations to improve the implementations
based on stakeholders recommendations. A full user interface formal evaluation
would need to be conducted taking into account existing usability issues.

Depending on which \ep~platform is used as a base system, this step would also
require to follow the specific for this platform developers' guidelines. Each
developing community has their established methods and rules which need to be
followed by any contributor. For example, to contribute code developed in this
project to the official Mahara \ep~system release would require such steps as
using Mahara code guidelines in all implementations, adapting the
implementations to the latest system version and submitting contribution to the
Gerrit\footnote{\url{http://code.google.com/p/gerrit/}} review system. After
code submission, it has to be independently reviewed by two developers from the
core team, fixed according to their feedback and properly tested. Only after all
these steps, the code can be included in the official system release.

The above example shows a complexity of contributing to the open-source project.
Adding code to the proprietary system is most likely to be even more complicated
process. This potential project should look into these issues.

\subsection{Further evaluations of the prototype}
As was discussed earlier, evaluation of the prototype implementations in the
real world settings was not feasible due to various reasons. Once the system is
in production version, a complex evaluation should be undertaken to support
current findings. Although difficult and often costly, naturalistic evaluation
or evaluation with real users using real systems to solve real problems is
considered to be crucial \citep{Pries-Heje2008}. In such evaluation a lot of
variable should be taken into account, but at the end it is a real \textit{proof
of the pudding} \citep{Venable2010}.

In addition, to check whether the findings can be generalized, evaluations of
the \ep~system with a larger number of participants and using different sampling
techniques where randomization of the sample is possible would be required. This
research used largely qualitative methods which are knows for their poor
generalizability \citep{Trochim2001}. Therefore, one of the potential future
research projects would look into the question of testing the finding from this
perspective.

\subsection[Optimal interface design]{Optimal interface for the implementations
supporting \LLLs requirements} 

As this project was focusing on the development of a functional prototype for
theory testing, user interface development was left outside of the research
scope. However, interface design plays an important role in acceptance of any
system by users as well as it might influence users' productivity and users'
satisfaction from using this system. Potential research project in the area of
HCI would look into the problem of designing an optimal user interface for the
features implemented in \ep~system supporting \LLLsn. 

A challenge of developing an optimal user interface will remain an outstanding
issue over time. The field of designing good and efficient Web interfaces will
always be a \textit{hot topic} due to the rapidly changing technologies aimed on
improving user experience. From this perspective, it would be interesting to
analyse whether the current interface implementations could have been done using
other visual representations and how it would influence learners' perception of
the system.

\subsection{``Seamless'' virtual learning environment for \LLLs support} 
For any learning environment to provide comprehensive support for \LLLs, it
should take into account all the transitions that individuals make during their
lifespan. Investigate how these transitions happen and how they are supported on
a system level would be another potential research area. This could bring up the
questions of improving interoperability standards, extending the range of
environments that support data transfer between them, developing standards for
trustworthy grades, certifications and qualifications export, and many others.

From another angle, it is possible that most of learning transitions would
involve changes of learning environments, both physical and virtual. Does not
matter whether these transitions are from school to university from college to
workplace, or between workplace, it is less likely that systems used by learners
in high school would look the same as systems used in universities. As a result,
with every transition people need to learn how to master new virtual learning
environment, get used to user interface and make use of the new environment most
efficient for productive learning. Can this process be improved or simplified?
If yes, how it can be done to improve learning experience?

\section{General Conclusions}

This thesis has presented a research project aimed on the development of an
\ep~aided learner-centred learning environment capable of providing
comprehensive support for students' \LLLs in universities. Based on the work
presented in this thesis, the following conclusions can be drawn:
\begin{enumerate}
  \item It is possible to develop such a learner-centred virtual learning
  environment that can provide support for \LLLs in universities, as has been
  demonstrated by the implementation of the stakeholders requirements in the
  prototype.
  \item As has been shown by extending the Mahara \ep~system, it is feasible to
  improve an existing \ep~system based on the formal requirements specification
  developed in the course of this project, and therefore to address the
  expectations of the \ep~systems to support \LLLsn.
  \item The findings from the evaluation studies indicate that the prototype
  functionality meets the needs of the stakeholders in both learning and
  teaching contexts. However, these findings should be confirmed by the further
  research investigations suggested in the previous section.
\end{enumerate}

It is important to remind here that the questions raised in this research are
not exhausted. This project just touched on the issues that exist in the area.
There are still challenges in working towards a comprehensive support for \LLLs in
universities, but these go beyond technologies. Overall, this research showed
how much can be achieved by encouraging cooperation between the stakeholders and
listening to the learner's voice. With the rapidly changing world of technology,
it is very important. Understanding the learners' needs will help universities
to invest in the successful adoption of the suitable learning systems providing
students with technology that would assist their \LLLs journey anywhere and at
any time.
