\chapter{Conclusions and Future Work\label{cha:conclusion}}
\epigraph{We now accept the fact that learning is a lifelong process of keeping
abreast of change. And the most pressing task is to teach people how to
learn.}{\textit{Peter F. Drucker}}
%3175
The main focus of this thesis has been on improving \LLLs support for students
in the universities by providing them with a better learning environment -- an
enhanced \ep~system. For this purpose, the requirements, design, development and
evaluation of the prototype environment has been discussed in detail. This final
chapter summarizes the outcomes of this research and its contributions to the
related fields. It discusses the implications for theory and practice, along
with the opportunities for the future research projects.

\section{Contributions and Outlook}

Despite the narrow focus of this research project -- developing a technical
solution for supporting \LLLsn, -- its contributions to the field and knowledge
should not be underestimated. Among these were:
\begin{itemize}
  \item \textit{Identifying the requirements for the \LLLs in universities based
  on academic literature}: the initial contribution of this research was the set
  of identified requirements to provide students with comprehensive support for
  their \LLLs in universities. These requirements are aimed at better supporting
  reflection, communication and collaboration, development and showcasing of
  skills and achievements, and tracking of learning progress.
  
  \item \textit{Development of the requirements specification}: Following the
  needs of the major stakeholders (students and lecturers), formal requirements
  for \LLLs support in universities were developed. This formal specification
  that is based on the identified requirements can be used or adopted by the
  developing teams of the \ep~systems to improve existing systems in the context
  of providing support for \LLLsn.
  
  \item \textit{Development of the functional prototype}: The Mahara \ep~system
  was modified to illustrate that these requirements can be successfully
  implemented. The functional prototype was developed to identify the strengths
  and weaknesses of the developed specification, and was an attempt to address
  primarily learners' needs rather than focus on institutional requirements.
  
  \item \textit{Development of a multiple perspective evaluation design}: Design
  of a proper evaluation was an important task to understand whether the initial
  research goal was achieved. Due to the nature of \LLLsn, evaluating it within
  the scope of this project was a challenge. A multiple perspective evaluation
  was designed to address this issue. It included conducting separate evaluation
  studies using suitable for each study methods to ensure that the perspectives
  of all stakeholders were taken into account. 
\end{itemize}

All these contributions were aimed at achieving the main goal of this research
to provide a technical solution capable of supporting \LLLs in universities.
Based on the outcomes of the evaluation, the overall approach to the problem can
be considered successful.

The value and impact of this research project goes beyond the technical
solutions. When this project started a bit over three years ago, there were few
active projects in the area of providing students with technical support for
\LLLsn. At that time, a lot of studies from the first wave of an \ep~system
implementation as a tool for supporting \LLLs have already been finished
\citep{Batson2010}. As a result, many universities around the world that had
been trying to adopt \ep~systems for \LLLs support discovered that incorporating
these systems into institutional environment was not an easy task. To
successfully introduce a new technology, it needs to be aligned with the views
and policies of an institution. It also requires a significant investment of
time and effort as well as adjusting curriculum, teaching and learning practice
in accordance with the new standards of \LLLsn.

Due to the fact that the changes in academic environment occur slowly,
especially if it involves incorporating new technologies \citep{Molebash1999},
suitable conditions did not exist at that time for an uptake of \ep~systems to
happen. This might explain why many \ep~adoption attempts did not pass the stage
of pilot projects, and the system as such did not gain as much recognition as
expected.

Despite these past issues, the developments in higher education over the recent
years showed that important changes are happening. It would not be an
exaggeration to say that many universities are now better prepared and more
ready to address the challenges of adopting new technologies for supporting
\LLLsn. This can be seen in numerous examples around the world, for instance
development of Curriculum 2010 Project (C2010) as a part of \textit{the
Strategic Plan 2009-2013} at Curtin University which was aimed at curriculum
renewal to utilise \LLLs skills in the courses \citep{Oliver2010}; declaring
support and expanding opportunities for \LLLs in 2020 road-map at Massey
University \citep{MasseyUniversity2012} which started in 2006 as a small group
initiative in one of the university departments and turned into an institutional
policy six years later; making official statement to provide students with tools
they need to become lifelong learners at the University of Ottawa in 2020
Strategic Goals \citep{UniversityofOttawa2011}.

In the road-maps for 2020, the European Commission is looking into increasing
participation in \LLLsn, guaranteeing funding for education, implementing \LLLs
policies and supporting partnerships between higher education institutions,
students, and employers \citep{EuropeanCommission2010,EuropeanUnion2009}.

The notion of changes can also be noticed in the recent criticisms of
exclusively LMS-based learning in universities. These criticisms inspired the
movement towards a new type of system called Personal Learning Environments
(PLE) which are learner-centred and can provide support for informal learning
\citep{Calvani2007}. In this context, \ep s are called \textit{the DNA of the
Personal Learning Environment} and considered to be a future of the systems
supporting \LLLs \citet{Attwell2007a}.

Following from the above described changes, another point has to be made here.
Literature shows that teaching practice and educational policies can be
influenced by new technology. However, this goes against the traditional belief
that the use of technology is limited to fitting or serving teaching goals
\citep{Levin2008}. The need for interaction between pedagogy and technology is
emphasized by \citet{Savin-Baden2006}. They argue that technology is not just
waiting to follow the established teaching strategies, but technology is on
equal terms with teaching. \citet{Cousin2005} claims that technology contributes
to, or even leads, the teaching process. This can be seen in the outcomes of
many research projects that show how technology influences teachers in the way
they design and deliver their courses \citep{Rutledge2012,Wang2002}. Such
research implies that the development of an effective technology can have
broader impact than just serving current teaching practices.

To conclude, in light of many universities getting ready for changes to
incorporate new technologies for supporting \LLLs and an important role that
technology can play in shaping teaching practice and institutional policies,
this research can be considered increasingly relevant. By helping to introduce a
better system, it can aid in further promoting \LLLs in universities. The ways
\ep~technology will be adopted and used in the future by universities may impact
learning of many generations of students.

\section{Implications For Theory and Practice}

This section discusses the lessons learnt and conclusions that can be drawn
based on results of this research. It suggests the changes that could be made in
various areas in order to improve the quality of \LLLs support in universities.

\subsection[Finding Balance]{Finding balance between what is good for teaching
and good for learning}

The conversations with lectures over the course of this research has emphasized
the challenges faced in doing justice to the varied demand of teaching, research
and administrative responsibilities. It is not possible to infinitely add on the
lecturers responsibilities every time a university administration decides to
introduce a new solution for supporting learning. Even now, it is difficult for
lecturers to find a balance in what should be used for teaching and learning.
This research confirmed a common belief that different people usually have
different perspectives. For lecturers it is especially important as they have to
look at things not from only their own teaching perspective, but as well from
the perspective of learners. It was easy to notice that some of the participants
had difficulties with understanding learners' needs due to not being able to put
themselves into their students' place.

On the other hand, it would not be a reasonable decision to ask lecturers to
employ technologies that might significantly affect their established
teaching process in terms of producing high workloads. For example, changing the
way lecturers give material to students or the way they mark learners'
assignments might increase the time that has to be spent on these activities to
get quality outcomes. Therefore, while introducing new systems to the learning
environment, university management should also perform a holistic review of
approaches to teaching that balances new requirements or added workload with
reduced effort in other areas.

\subsection[Cooperation and Communication]{Importance of cooperation in 
introducing new technologies or new teaching principles}

A surprising for the researcher discovery made during the course of this
research was that communication between various departments was in quite poor
state. If the lecturers were more engaged with teaching and learning projects
carried out in their departments or schools, they were less likely to know about
similar research projects in other areas of the university. 

Work with the lecturers at various stages of this project showed the importance
of communication and cooperation between the university departments. The
lecturers who were research participants were usually highly surprised when they
discovered that some other departments were using or had past experience of
employing exactly the same technologies with their students. A lot of issues and
troubles for new institutional developments could have been avoided if proper
communication or presenting departments' research was done across the university.

\subsection[Learning to Use New Technology]{Learning to use new technology}

A curious fact was noticed in the course of this research: the students who
participated in various studies of this project had practically no complaints
about LMS used in their university, although LMS and the \ep~systems were
discussed equally. In contrast, the \ep~system raised criticisms and expressions
of dissatisfaction. How is it that the system that has been built for learners
displeases these very learners so much? A possible explanation for this
phenomena might be in lack of student teaching of how to use \ep~systems.

Looking at the current state of LMS use in universities, technology level help
might be all that is required for students to master this type of system. Use of
LMS is generally simple, and can be summarized in such activities as accessing
and uploading files, and in some cases posting to forums or answering quizzes.

Due to long history of LMS being employed in universities, lecturers already
have their own established strategies of using these systems, especially in
relation to delivery of distance education. Furthermore, help for aspiring
lecturers can be acquired from the literature. For example, Packt e-Learning
Library\footnote{\url{http://packtlib.packtpub.com/e-learning} (Accessed April
16, 2012)} currently publishes books covering Moodle, Blackboard, Sakai, and
other LMS. In addition to numerous online resources and literature on how to use
LMS to their full potential, there are staff trainings that are usually
organized by many universities for those lecturers who need extra assistance.

With the support of university administration \citep{DiBenedetto2005}, now
lecturers need to learn how to guide students through \LLLsn, and need to master
new technology that supports it, \ep~systems in the context of this thesis
\citep{Levin2008}. Compared to LMS, work with \ep~systems is more complex. It
goes beyond simple file management and requires understanding of the importance
of reflection, purposeful selection of artifacts, and demonstration of personal
achievements. Students who have never been taught all these things might not see
the value of an \ep~system as well.

Understanding of these issues is currently developing in the universities
around the world. For instance, based on the case studies presented at
PebbleBash 2010 conference \citep{PebbleLearningLtd2010a}, many universities in
the UK and Australia have moved towards teaching students how to use \ep s,
showcase skills and achievements, write meaningful reflections, and provide
useful peer feedback.

In addition to the above issues, one of the lecturer-participants of this
research noticed that students should start their learning journey with an
\ep~system on early stages of their studying. Doing so, rather than trying to
fit \ep s into the students' established way of learning might result in more
positive outcomes. This, in turn, might lead to higher acceptance of the
\ep~concepts and systems among students. Although, this assumption needs to
be carefully tested in the future.

\section{Future Research and Potential Extensions}
The research presented in this thesis is one of the steps on the way to
providing students with full support for their \LLLs journey in universities. In
the course of this research, a number of issues and questions were raised which
creates potential for further investigations. Future research should aim at
exploring and solving these issues to support the findings of the current
research and extend the knowledge added. The following sections suggest
potential future studies.

\subsection[Enhancement of the prototype]{Enhancement of the prototype towards a
production quality system} 
Due to the high interest in the outcomes of this project among the research
participants, the next logical step for this research would be moving the
\ep~system prototype to production level. Important part of this step would
include putting the evaluation feedback into practice to improve the
implementations based on stakeholder recommendations. In addition to the
evaluation studies already carried out, a full user interface formal evaluation
would need to be conducted taking into account existing usability issues.

Depending on which \ep~platform is used as a base system, this step would also
require to follow the specific for the platform developer guidelines. Each
developing community has their established methods and rules which need to be
followed by any contributor. For example, to contribute code developed in this
project to the official Mahara \ep~system
release\footnote{\url{https://launchpad.net/mahara} (Accessed April 18, 2012)}
would require such steps as using Mahara code guidelines in all implementations,
adapting the implementations to the latest system version and submitting
contribution to the Gerrit\footnote{\url{http://code.google.com/p/gerrit/}
(Accessed April 16, 2012)} review system. After code submission, it has to be
independently reviewed by two developers from the core team, fixed according to
their feedback, properly tested, and documented for future maintenance purposes.
Only after all these steps, the code can be included in the official system
release.

The above example shows a complexity of contributing to the open-source project.
Adding code to the proprietary system is most likely to be even more complicated
process. A future project should look into these issues.

\subsection[Optimal interface design]{Optimal interface for the implementations
supporting \LLLs requirements} 

As this project was focusing on the development of a functional prototype for
theory testing, user interface development was outside of the research scope.
However, interface design plays an important role in acceptance of any system by
users as well as it might influence users' productivity and users' satisfaction
from using this system. Potential research project in the area of HCI would look
into the problem of designing an optimal user interface for the features
implemented in \ep~system supporting \LLLsn.

A challenge of developing an optimal user interface will remain an outstanding
issue over time. The field of designing good and efficient Web interfaces will
always be a \textit{hot topic} due to the rapidly changing technologies aimed on
improving user experience. From this perspective, it would be interesting to
analyse whether the current interface implementations could have been done using
other visual representations and how it would influence learners' perception of
the system.

\subsection{Further evaluations of the prototype}
As was discussed earlier, evaluation of the prototype implementations in the
real world settings was not feasible due to various reasons, such as short
timeframe, resources constraints, and difficulties of evaluating \LLLsn.
However, once the system is in its production version, a complex evaluation
should be undertaken to support current findings. Although difficult and often
costly, naturalistic evaluation, or evaluation with real users using real
systems to solve real problems is considered to be crucial
\citep{Pries-Heje2008}. In such evaluation, a lot of variables should be taken
into account, but at the end it can be called a real \textit{proof of the
pudding} \citep{Venable2010}.

In addition to naturalistic evaluation, to check whether the findings can be
generalized, evaluations of the \ep~system with a larger number of participants
and using different sampling techniques where randomization of the sample is
possible would be required. This research used largely qualitative methods which
are knows for their poor generalizability \citep{Trochim2001}. Therefore, one of
the potential future research projects would look into the issues of testing the
findings of the current project from the perspective of generalizing them.

\subsection[``Seamless'' environment for \LLLs support]{``Seamless'' virtual
learning environment for \LLLs support} 
For any learning environment to provide comprehensive support for \LLLs, it
should take into account all the transitions that individuals make during their
lifespan. Investigate how these transitions happen and how they are supported on
a system level would be another potential research area. This could bring up the
questions of improving existing interoperability standards, extending the range
of environments that support data transfer between them, developing standards
for trustworthy grades, certifications and qualifications export, and many
others.

From another angle, it is possible that most of learning transitions would
involve changes of learning environments, both physical and virtual. Does not
matter whether these transitions are from school to university from college to
workplace, or between workplace, it is less likely that systems used by learners
in high school would look the same as systems used in universities. As a result,
with every transition learners might need to learn how to master a new virtual
learning environment, get used to the user interface and make use of the new
environment in the most efficient for productive learning way. Can this process
be improved or simplified? If yes, how it can be done to improve learning
experience?

\section{General Conclusions}

This thesis has presented a research project aimed on the development of an
\ep~aided learner-centred learning environment capable of providing
comprehensive support for students' \LLLs in universities. Based on the work
presented in this thesis, the following conclusions can be drawn:
\begin{enumerate}
  \item It is possible to develop such a learner-centred virtual learning
  environment that can provide support for \LLLs in universities, as has been
  demonstrated by the implementation of the stakeholders requirements in the
  prototype.
  \item As has been shown by extending the Mahara \ep~system, it is feasible to
  improve an existing \ep~system based on the formal requirements specification
  developed in the course of this project, and therefore to address the
  expectations of the \ep~systems to support \LLLsn.
  \item The findings from the evaluation studies indicate that the prototype
  functionality meets the needs of the stakeholders in both learning and
  teaching contexts. However, these findings should be confirmed by the further
  research investigations suggested in the previous section.
\end{enumerate}

It is important to note here that the questions raised in this research are
not exhausted. This project just touched on the issues that exist in the area.
There are still challenges in working towards a comprehensive support for \LLLs in
universities, but these go beyond technologies. Overall, this research showed
how much can be achieved by encouraging cooperation between the stakeholders and
listening to the learner's voice. With the rapidly changing world of technology,
it is very important. Understanding the learners' needs will help universities
to invest in the successful adoption of the suitable learning systems providing
students with technology that would assist their \LLLs journey anywhere and at
any time.
