\chapter{Conclusions and Future Work\label{cha:conclusion}}
\epigraph{We now accept the fact that learning is a lifelong process of keeping
abreast of change. And the most pressing task is to teach people how to
learn.}{\textit{Peter F. Drucker}}
%240
The main focus of this thesis has been on improving \LLLs support for students
in the universities by providing them with a better learning environment -- an
enhanced \ep~system. For this purpose, the requirements, design, development and
evaluation of the prototype environment has been discussed in detail. This final
chapter summarizes the main contributions of this research to the related
fields, along with the opportunities for the future research projects.

\section{Research Contributions}

The following research contributions to the field were made in course of this
project:

Discovering the requirements for the \LLLs in the universities

Development of requirements specification

Development of the functional prototype

Development of a multi-perspective evaluation design

\section{Future Research}
%What new questions can be developed for deeper research?
The research presented in this thesis is the first step on the way to providing
students with full support for their \LLLs journey in the universities. In the
course of this research, a number of issues and questions were raised which
creates potential for further investigations. Future research should aim at
exploring and solving these issues to support the findings of the current
research and extend the knowledge added. This section suggests potential future
studies.

\begin{itemize}
  \item Seamless \ep~system for \LLLs -- systems transitions from early
  childhood to late adulthood.
  \item 2
  \item 3
  \item 4
  \item 5
  \item 6
\end{itemize}

\section{General Conclusions}

Text