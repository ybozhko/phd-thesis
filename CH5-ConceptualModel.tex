\chapter{Stakeholders Requirements for \LLLc Support in Universities
\label{cha:model}}
\chaptermark{Stakeholders Requirements}

\section{Lecturers' Perspective on \LLLc Support}
\subsection{Participants Profile}
Using a theoretical or criterion sampling strategy (Byrne, 2001; Warren, 2001),
we were looking for the respondents who met certain criteria of being academics
with previous experience of using LMS and ePortfolio systems in their teaching.
Ten academics, mostly the participants of a previous institutional ePortfolio
initiative[1], were approached by email and invited to participate in the
research project. Nine academics from various sections of the university
(College of Business, College of Education, and College of Sciences) accepted
the invitation and agreed to be interviewed. In April-May 2010 we conducted nine
in-depth interviews to gather the data required for the analysis. All interviews
were audio recorded and transcribed to make follow-up analysis easier and more
thorough.
\subsection{Methodology}

We used semi-structured interviews as an instrument for this phase of the
research. Interviews were favoured over the other research methods because they
are suitable for gaining insight into the experience and knowledge of others
(Schostak, 2006). It was important for this project as we had a limited pool o 
potential participants that would satisfy our criteria and, in this case, the
required depth of investigation could not be reached with questionnaires.

The interviews were guided by a number of scenarios each described a particular
situation connected to the problems of lifelong learning support in universities
(See Appendix for a scenario example). The scenarios describe situations from
the teaching perspective, to support talking to lecturers from their
perspective. Topics for these scenarios were selected from the literature review
conducted at the first stage of the project and observations from the reports
and communications with the College of Sciences ePortfolio Initiative project
team members[2] at Massey University.

\subsection{Results}

\section{Students' Perspective on \LLLc Support}
\subsection{Participants Profile}
We used interviews with the stakeholders of the area as an instrument for this
phase of the research. It was favoured over the other research methods because
we had a limited pool of potential participants and needed a tool that would
help us to gain an insight into the experience and knowledge of others. The
interview fit well with this criteria (Schostak, 2006). 

The interviews in this
project were used to explore the current problems and challenges of lifelong
learning support in universities. As this research is focused on system support,
the problems in the teaching process and institutional policies were not among
the topics for discussion during the interviews. The aim was to understand the
gaps and shortcomings that currently exist in e-leaning environments, as well as
to get the interviewees' views on what is needed to make the systems already
used in universities become more efficient and adequate for students’ lifelong
learning support. We also aimed to get feedback on potential system features we
had proposed, based on our literature review and gaps analysis. 

We used a
snowball or chain sampling strategy (Mack, et al., 2005; Marshall and Rossman,
2011) to identify potential participants for the interviews. The academics, who
have participated in the similar interviews at an earlier stage of the project,
were asked to provide student contacts or to inform students of this research in
their classroom. As the use of ePortfolio is relatively new at Massey
University, this technique was used to make sure that students who took part in
the interview were familiar with both systems. We believe that students who have
already used both systems in practice are more experienced and provide richer
data for discussion and analysis. About 30 students were approached by email and
invited to participate in the research project. Overall, nine students from
various schools and colleges of Massey University accepted the invitation and
agreed to be interviewed. In May-September 2010 we conducted nine in-depth
interviews to gather the data required for the analysis. All interviews were
audio recorded to make a follow-up analysis easier and more thorough.

\subsection{Methodology}

The interviews were guided by a number of scenarios (see Figure 1 for an
example) that described some particular situation connected to the problems of
lifelong learning support in universities.

Themes for the scenarios were selected from the literature review and
observations from the reports and communications with the College of Sciences
ePortfolio Initiative team members at Massey University. As the interviews were
conducted with the students, we tried to construct every scenario to describe
the situation from the learner’s perspective.

The participants were asked a set of open-ended questions to elicit their views
on the environments that support lifelong learning in the university. They were
asked how they had been using LMS and ePortfolio together and what advantages
and disadvantages they saw in such a combined approach. We asked for their
opinions on what features their current ePortfolio system lacked, or which
features could make such system more useful and relevant to lifelong learning
support in universities. The interviewees were also invited to suggest
functionalities they would expect to see in a good e-learning environment that
supported lifelong learning in universities.

\subsection{Results}

\section{Requirements Elicitation}

\section{Summary}