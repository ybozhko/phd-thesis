\chapter[Stakeholder Requirements for \LLLc Support ]{Stakeholder Requirements
for \LLLc Support in Universities\label{cha:model}}
\chaptermark{Stakeholder Requirements}
%6860
This chapter describes the results of interviews conducted with lecturers and
students of Massey University as a part of \LLLs requirements analysis. There
were three reasons for involving stakeholders at this stage of the research. The
first reason was that the literature did not provide information on successful
\LLLs support that would go beyond the highly conceptual recommendations. The
second reason was that at this stage it was difficult to understand whether
\ep~as a system created to support \LLLs was effective enough in doing so. The
last reason was that to understand whether \ep~systems follow the
recommendations identified in the literature, these recommendations have to be
translated into system requirements. Therefore, to support the findings from the
literature and the review of the learning spaces in Chapter \ref{cha:systudy},
it was necessary to take the guiding principles of \LLLs support together with
the \ep~system requirements to the practical level. In this case, the main
stakeholders -- students and lecturers -- were invited to analyze the area of
\LLLs support in universities from their perspectives.

\section{Limitations}

As was already mentioned in the methodological limitations section of Chapter
\ref{cha:method}, a limitation for this stage included a small sample size due
to the limited pool of the potential participants who would meet certain
criteria. In case of this research project, one of the main criteria included
having prior experience of using e-learning environments, \ep~systems in
particular, in teaching and learning contexts.

In addition to above mentioned criteria, availability of potential student
participants was reduced even more by the desire of involving a mature student
audience (aged 25 and older). This decision was based on the findings from the
studies in various fields which showed that mature students are more oriented
towards meaning-directed learning \citep{Smith2010}, better express their ideas
\citep{Lea2010} and overall perform better in various tasks due to their life
experience and knowledge \citep{Sherwood1987}. In a personal communication with
the researcher, this notion was also supported by the academic developer
involved in the College of Science \ep~initiative and responsible for
facilitating \ep~use at Massey University. According to their experience,
younger students might lack understanding of the \LLLs concepts and tend to be
guided by teachers and grades. It was expected that more mature students would
be better aware of the \LLLs skills than younger students and would try to look
at their study at university from the \LLLs perspective.

\section{Lecturer Perspective on \LLLc Support}

The goal of the interviews with the lecturers was to explore the current
problems and challenges of \LLLs support in universities from their teaching
perspective. However, as this research is focused on system support, the
problems in the teaching process and institutional policies were not among the
topics for discussion during the interviews. The aim was to understand the gaps
and shortcomings that currently exist in e-learning environments, as well as to
get the interviewees' views on what is needed to make the systems already used
in universities, \ep~systems in particular, become more efficient and adequate
for supporting students in \LLLsn. 

\subsection{Participant Profiles}
Using a theoretical or criterion sampling strategy \citep{Byrne2001,
Warren2001}, this phase of the project looked for respondents who met certain
criteria of being academics with previous experience of using e-learning
systems, \ep~systems in particular, in their teaching. Ten academics, mostly the
participants of a previous institutional
\ep~initiative\footnote{\url{http://science.massey.ac.nz/eportfolios} (Accessed
April 16, 2012)}, were approached by email and invited to participate in this research project. Nine
academics from various sections of Massey University (College of Business,
College of Education, and College of Sciences) accepted the invitation and
agreed to be interviewed. Nine in-depth interviews were conducted in April-May
2010 to gather the data required for the analysis. All interviews were audio
recorded and transcribed to make follow-up analysis easier and more thorough.

\subsection{Methodology}

In-depth, semi-structured, face-to-face interviews were used as an instrument
for this phase of the research. It was favoured over the other research methods
for the following reasons. This kind of interview has an open style while
remaining structured \citep{Gillham2000} which gives an interviewer freedom
within the predefined framework. Due to its openness and flexibility,
semi-structured interview has become widely applicable and popular research
instrument.

Face-to-face interviews have more quality advantages than telephone ones
\citep{Shuy2001}, however this characteristic adds a restriction to the number
of people that are feasible to interview. According to \citet{Gillham2000},
face-to-face interviews require high costs/time and potential interviewees
accessibility which makes it necessary to keep the number of participants to a
minimum, just to cover representativeness. Therefore, every participant becomes
a \textit{key} informant in the project.

Depth of meaning is considered to be central in the interview.
\citet{Johnson2001} described the benefit of in-depth interviewing as the one
that gives an interviewer an opportunity to achieve deep understanding inherent
to participants in some daily activity and as well allows to get the multiple
perspectives on this activity.

Due to all mentioned characteristics and being able to gain insight into the
experience and knowledge of others \citep{Schostak2006}, in-depth
semi-structured interviews were considered to be a suitable instrument for this
project phase. It was important for this project as it had a limited pool of 
potential participants that would satisfy the selection criteria and, in this
case, the required depth of investigation could not be reached with other
methods, such as questionnaires.

The interviews were guided by a number of scenarios each described a particular
situation connected to the problems of \LLLs support in universities (Appendix
\ref{sec:appscenariolect}). The scenarios described situations from the teaching
perspective, to support talking to lecturers from their perspective. Topics for
these scenarios were selected from the literature review conducted at the first
stage of the project and observations from the reports and communications with
the College of Sciences \ep~Initiative project team
members\footnote{\url{http://science.massey.ac.nz/eportfolios/contacts.asp}
(Accessed April 16, 2012)} at Massey University. The participants were also asked open-ended question
(Appendix \ref{sec:appquestlect}) on their experience of using an \ep~system in
their teaching, problems encountered with their current \ep~system, and
improvements they would like to see in the systems.

\subsection{Findings}

Analysis strategies suggested by \citet{Marshall2010} were used to analyse
issues and gaps identified during the interview discussions. Themes to group
these issues were developed. These themes included the topics of integration of
an \ep~system with LMS, addressing graduate attributes, supporting learning
outside of the course boundaries and other issues. The direct quotations in this
and all following sections were taken from interview transcripts.

\textbf{Theme 1: Integration with LMS for supporting course-related activities}

All participants said that including development of \LLLs skills in university
programs changed the way they teach, interact with students, and use e-learning
systems. Talking about the latter, lecturers wanted to see the systems they used
integrated into their regular activities. This meant that systems should not be
external to the teaching and learning process, but be a part of it. Systems
should not be seen as creating extra work or adding to workload. However at
this stage, from staff's point of view, all their experience of using e-learning
technology was about creating more workload.

\shortquote{Academic 1: For us [lecturers] \ep, with all its blogs, pages,
views, presentations, or anything else, will never get any traction until it
behaves like an electronic assignment.}

Lecturers saw the importance of introducing students to various \LLLs skills.
They said that the development of \LLLs skills should be integrated through
everything that students do and supported at the system level as well. They
emphasized that for students the current systems look like two separate worlds.
These worlds both provide some pieces of activities, but cannot be seen as a
whole.

\shortquote{Academic 2: Unless there are no layers built in each year in
connection, students will see that this [work with systems] is really
disconnected.}

Therefore, they would like to see an \ep~system and LMS integrated in a way
such that students would be able to do simple, but valuable activities as a part
of their learning process rather than in addition to it. For example, this might
include introducing students to the importance of being reflective about the
choices and efforts they make; writing reflections or keeping reflective
journals in \ep~system with an opportunity of submitting everything they do as
an assignment to the LMS; getting feedback on their assignments directly to
\ep~after they were marked; and reflecting on this feedback afterwards. This
approach would partly address the requirement of providing support for all
aspects of learning and being able to record progress from various sources. It
would also allow students and lecturers to work in the system that suits their
needs best and provides support for various activities, like developing and
sharing content, assessing, reflecting, or giving feedback.

While talking about communication as an essential part of learning process
\citep{Schaffert2008}, it was mentioned that the feedback cycle between students
and lecturers needed a better support. Lecturers said that they wanted to give
students their feedback on various kinds of work, such as assignments,
reflections and discussions. The majority of academics did not see any problems
if students decided to copy the teacher's feedback to their personal \ep.

\shortquote{Academic 3: For me they have shared their learning with me, so
whatever I put in there from my perspective that is for them to do whatever they
want to do with that. I don't have a problem with them keeping it or putting it
elsewhere, because if I did, I would not put it in there. For me it is about
helping them to learn.}

However, what was really important for lecturers in the process of communicating
with students through feedback was student's response. They wanted to see that
their feedback was not ignored, but was listened to by students and used to
improve their learning outcomes.

\shortquote{Academic 1: Teachers don't like giving feedback if it is ignored.
What would be interesting is if the students had to respond to the feedback that
the staff member gave. Creating some kind of dialog between staff and students.
Conceptually, it is really important that this dialog is going on. It motivates
staff members when they see their feedback is being listened to.}

In summary, academics were looking for the e-learning environment that would
allow students and lecturers to make various educationally valuable tasks so easy,
that they would be done by both, on a regular basis.

\textbf{Theme 2: Addressing graduate attributes}

Graduate attributes could be used as a way of addressing higher-order skills
which is required for successful \LLLs \citep{Hart1999}. All interview
participants agreed that introducing students to graduate attributes/profiles at
an earlier stage of their study was important. It would give students a
complete picture of what kind of graduates they are expected to become and would
give them an opportunity to integrate their knowledge from various sources other
than only from the degree programme.

\shortquote{Academic 3: It is good for students to know what is expected from
them at the end. So they can understand how the assessment and what they do puts
together. Lots of students see everything just like the course that they have to
pass, rather than a coherent course of study that has a reason for being
together. For example, why are the communication skills so important? -- Because
at any job later you will need to be able to talk to people, to be able to
write, because these things are fundamental.}

From the lecturers' point of view, graduate attributes were a balance between
what was required from the university to \textit{produce} good graduates and
what, primarily, had to be achieved by students to become lifelong learners. It
was also suggested that graduate attributes were a challenge to introduce. They
had to be really thought through and integrated right through everything
students do.

\shortquote{Academic 4: They [Graduate Attributes] are a challenge to set.
Students need a bigger picture to go really beyond the university because there
are so many timeframes and stages beyond the university. It will help students
to start understanding that they got lots of talents. They can link it to
examples, which is what gets lost when there is so much going on.}

Lecturers saw graduate attributes as a potential solution to the problem that
students did not consider their formal learning as a part of their \LLLsn.
Lecturers believed that graduate attributes might help students to focus beyond
what is required for a particular programme of study.

Lecturers also said that it was important to give students an opportunity to
develop their own set of graduate attributes, in addition to institutional ones.
This would allow them to describe their aspirations and help them to define
their future goals. However, this should be done carefully as students still
require guidance and support.

\shortquote{Academic 6: So, graduate attributes, I think, are positive thing.
They are just a wider example of what we do now by intended outcomes for each
paper. We should be absolutely explicit of what our students are able to do when
they finish. I think, we want to give students freedom -- absolutely, but not
absolute freedom. It is a freedom within a structured process that says ``You
have to get from here to here, and it is up to you which way to got, but still
you have to get from here to here''.}

From the systems perspective, several respondents suggested providing students
with a comprehensive template that would define a set of generic \LLLs skills,
as well as skills specific to their programme of study. This template should be
populated by students with various examples from their \ep~as they move across
their study supported by the lecturer's guidance and feedback. It should be up
to university policy when and how to evaluate graduate attributes, but grades
should not be kept in the \ep~system. Lecturers thought that only feedback or
references should go to the students' personal \ep~space.

One of the respondents was critical about a template approach. According to his
point of view, there might be a danger that students would see a template as a
simple \textit{ticking checkboxes} activity. To avoid it, there should be a way
of supporting each example with a detailed explanation or reflection on
experience gained.

\textbf{Theme 3: Going beyond the course boundaries}

One of the respondents described a good e-learning environment in universities
as bringing two worlds into one:

\shortquote{Academic 1: It should create two worlds in one for them [students]:
content world and the world where they do their learning. Last world should give
them the confidence to explore whatever they want to. This world should be
within the programme and beyond the programme somehow.}

From a lecturer's perspective, the LMS was a world of content which they
delivered to students and a world where students' formal education happened with
such course-related activities as submitting assignments for marking or discussing
course-related problems on forums. However, as was shown through the review
of learning spaces, in the majority of LMS, students lose access to the items
they have created when they finish a course section or graduate.

All interviewees thought that it was important for students to retain access to
the work that they do. Some stated that due to the institutional LMS policy they
often recommended that their students saved their work externally in order not
to lose it later.

To deal with this problem, lecturers suggested LMS and \ep~systems to be
integrated in a way that would allow for easy data transfer between systems.
Integrated systems should give students an option of transferring all items at
once or copying them one by one.

While discussing data flow options, some concerns about copying collaborative
work were revealed. For example, when students participate in discussions on a
course forum, is it ethical to allow them to copy posts by other participants
into a personal \ep? Should other participants' names be substituted with
pseudonyms when data transfer is performed? The majority of academics said that
the best solution for them would be a general policy for the forum which would
tell students before they join the discussion that their posts might be copied
to another participant's \ep. In this case, any participant in the discussion
should have an option whether they want to become anonymous or not, when the
discussion is copied from one system to another.

\textbf{Theme 4: Non-functional and other issues}

Although, the aim of the interviews was to identify the gaps and shortcomings in
existing systems, other issues, not all connected to technical ones, were
discussed.

One of the issues was that academics noticed that a lot of students struggled
with technology. Therefore, any system developed for students should have
intuitive and visually appealing interface which would satisfy novice users as
well as advanced users.

\shortquote{Academic 5: Some people struggle with technology. There should be
something that they understand quite clear. I think the less you ask them
[system users] the better. As much as possible should be done automatically.}

\shortquote{Academic 1: I think it's all about integration. There are so many
tools there already in both, social world as well as institutional. \ldots Ease
of use is going to be crucial. \ldots bring everything together and let the
systems do what they do best.}

Another issue was that students should be provided with the example of good
\ep, so that they could see what \textit{a good \ep} means, what features make
it \textit{good} and how it can be developed.

\shortquote{Academic 2: I think that students still don't see a sample
\ep~and how it could work for them. Unless it is clear for them that
\ep~can be used this way, they will see this as an extra piece of work
and ask ``what would be the benefit?''}

The last issue was that students and lecturers require institutional support.
Students need to be taught to use the system as well as to understand the
general concept of the \ep~process. It was important to show them how \ep s
could connect course-related activities with their \LLLsn. Lecturers, in turn,
need support from the university on how to include the practice of developing
\LLLs skills to the teaching process.

\shortquote{Academic 6: \ep~for \LLLs is a very good idea, but you have to
introduce it at first year first semester at the university. So, when students
are at the end, they have a record in \ep~of what they have done. And it's
just a tool you learn like learning to use LMS. But to do that, it should be
integrated into the teaching.}
 
\subsection{Discussion}
Analysis of the interviews showed that lecturers looked at an \ep~system as a
part of the learning environment in connection to LMS. It was not surprising as
they were focusing primarily on efficient teaching and making their work easier
at the same time.

The feedback given to the scenarios was generally congruent with the information
found in the literature. The interviews showed that academics saw and understood
the value of \LLLs skills. They were willing to incorporate activities aimed to
develop these skills into their teaching as soon as they were supported on
various levels such as department or institutional. They saw the support on a
system level as an important part of \LLLs support in the university.

The majority of the participants said that the scenarios showed to them were
realistic and described the problems of \LLLs support from various angles. The
participants tended to give a lot of information on potential solutions.
Therefore, it was assumed that they had encountered similar problems and had
been thinking about ways to deal with these.

From the lecturers' perspective, the main challenge in providing system support
for \LLLs would be to make it connected through every learning activity students
do. If the \ep~systems were to be used in the universities for \LLLsn, then they
should be connected with the LMS. In this case, student's acceptance condition
for such an e-learning environment would be the ease with which they could move
between the systems while doing regular learning tasks. According to the
participants' teaching experience, students usually expect an immediate reward
for everything they do. Students lack motivation if learning activities are not
compulsory or not marked. They also have problems with seeing the bigger picture
of their personal and professional development, focusing rather on passing the
courses that lead to the degree qualification. Therefore, lecturers saw the
value of integrating LMS and \ep~systems for \LLLs support and using such
an approach to show students the opportunities for their development as lifelong
learners. An integrated environment would let students gather the small parts
demonstrating their development and so contributing to the bigger picture of
\LLLsn.

\section{Student Perspective on \LLLc Support}

As the overarching aim of this research is to support students' \LLLsn, it is
crucial to understand the requirements from a student perspective in the first
place. Together with the results of the interviews with lecturers, the findings
from the interviews with students would provide valuable background for the
requirements specification of an \ep~enhanced environment in universities.

\subsection{Participant Profiles}

A snowball or chain sampling strategy \citep{Mack2005, Marshall2010} was used to
identify potential participants for the interviews. The lecturers, who 
participated in the interviews at an earlier stage of the project, were asked to
provide the researcher with student contacts or to inform students of this
research in their classroom. As the use of \ep~was relatively new at Massey
University, this technique was used to make sure that students who took part in
the interviews were familiar with e-learning systems such as LMS and \ep. It was
anticipated that students who have already used both systems in practice were
more experienced and would provide richer data for discussion and analysis.
About 30 students were approached by email and invited to participate in the
research project. Overall, nine students from various schools and colleges of
Massey University (College of Education and College of Sciences) accepted the
invitation and agreed to be interviewed. Nine in-depth interviews were conducted
in May-September 2010 to gather the data required for the analysis.

\subsection{Methodology}

Methods used to explore \LLLs support in universities from the student's
perspective were similar to the methods used for the interviews with the
lecturers. The only difference was in the scenarios presented to the students.
This time they were constructed to describe the situations connected to the
problems of \LLLs support in universities from the learner's perspective
(Appendix \ref{sec:appscenariostud}).

In addition to the scenarios, all participants were asked open-ended questions
(Appendix \ref{sec:appqueststud}) to elicit their views on the environments that
can support \LLLs in the university. They were asked for their opinions on what
features their current \ep~system lacked, or which features could make the
system more useful and relevant to \LLLs support in universities. The
interviewees were also invited to suggest \ep~functionality that would provide
support for \LLLs recommendations. It was important to understand what potential
system features could be developed and how existing features could be improved,
based on the literature review and gaps analysis.

\subsection{Findings}

Similarly to the analysis methods used during the interviews with the lecturers,
the issues and gaps were identified from the discussions with the students and
grouped into themes described further in this section. To get different
perspectives on the same problems, some themes follow the discussion topics from
the interviews with the lecturers.

\textbf{Theme 1: Supporting course-related activities vs. going beyond the
course boundaries}

Unlike the lecturers who seemed to have an understanding of how every day
course activities link to the development of \LLLs skills, students admitted 
of having difficulties with finding a balance between doing course-related
activities and having one big personal development picture in mind. They claimed
that current study programs had not been designed properly to encourage them to
look beyond the course boundaries.

\shortquote{Student 6: Everything we learn is disconnected. As a student, I
expect a reward for what I am doing. Probably, if lecturers could make it so
that this reward would connect with the aim of long-term development, it would
be helpful.}

Students said that including \ep s into their studying process at the university
might be helpful in solving this problem. The majority of the participants said
that they liked using \ep s for their learning activities. They found some
things (such as reflection or selection of proper examples) quite challenging,
but noticed that at the end they understood the value of what they had been
doing.

However, in order to use an \ep~system in the university as a system for \LLLsn,
students expected it to be integrated through all tasks in their learning
process. In this case, it meant using \ep~throughout every course, being able to
use the results of their \ep~work as assignments, getting feedback on their
learning progress from the lecturers and mentors, etc. Otherwise, without a
proper integration and until the way academics develop study program and
construct learning activities changes, the situation will most likely remain the
same:

\shortquote{Student 1: We are not required to use it [\ep] this year, so
I just cannot force myself to go there and keep it updated for my study because
it is not integrated. If it was, I would be more likely to do that.}

Although, this might seem to be a problem that requires some complex
administrative and policy changes and cannot be resolved by improving systems
alone, its solution would influence the ways the systems are currently used in
the universities and would depend on functionality to support these changes.

\textbf{Theme 2: Addressing graduate attributes}

All students agreed that graduate attributes could be a good approach that might
show them the full picture of their study. However, as was mentioned in the
previous theme, achieving these attributes should be connected through every
activity that students do in their study.

\shortquote{Student 2: I think that it's really good that there are certain
things you need to have achieved and you commit to achieve them, but it should
be in course design.}

From the systems perspective, students came up with a similar solution to the
one suggested by the lecturers. A template in an \ep~system that would define a
set of \LLLs skills would be sufficient for students to start working towards
these skills. It should be possible to attach examples as evidence from their
\ep~space as well as LMS. In addition, it would be helpful if lecturers could
point out the opportunities for such examples in the learning activities
students do for their degree program study.

It would be important for students to be able to add their own set of skills to
the list of institutional attributes as it could help them to set up their
personal development goals and understand what they want to achieve. Some
students also added that once completed this kind of template might be helpful
in their job, scholarship or other applications.

\textbf{Theme 3: Data flow between systems}

In addition to the previous themes, to emphasise the importance of integrating
LMS with \ep~system through course design, it was confirmed that students need
these systems to be combined. Otherwise, they would see these systems as two
different worlds.

\shortquote{Student 1: Most of my time these days is spent in LMS. My readings
are all there, my assignments are all there, and my forums are there. That's
where I work. So, if \ep~were a part of my e-learning environment in a
seamless manner, I will be more likely to use it of my own initiative.}

Students found it was time consuming to work in two separated environments. It
was difficult to keep track of what was stored in each system and difficult to
transfer from LMS items like assignments feedback or forum posts that were
potentially good examples to showcase in their \ep s.

\shortquote{Student 7: My current course is my focus and that is the most
important thing to me at the moment. But after I finish it, I want to be able to
wrap it up and save to \ep~where I can showcase what I've done.}

Moreover, as in LMS students tended to lose access to their courses once they
were finished or after graduation, students could not treat LMS as a space
suitable for their personal development. Without being able to transfer
everything they have done during their course study, students could lose a lot
of valuable examples of their achievements.

\shortquote{Student 4: My supervisor just told me to save everything, so that
when LMS is closed for me, at least I will have information saved. So, I did it
with every single thing I had.}

To deal with this problem, students wanted LMS and \ep~systems to be integrated
in a way that would allow for easy data transfer between systems. Integrated
systems should give students an option of transferring all items at once,
copying them one by one, or setting up transfer schedule.

Ethics concerns about transferring collaborative work could be resolved by
setting up rules or getting permissions.

\shortquote{Student 3: If we are discussing something with someone on the forum,
I treat their ideas as their intellectual property. So, if I want to use these
ideas, I will ask for their permission and would expect the same from their
side.}

In addition, a suggestion on the general policy made by lecturers was put to
students. It appeared to be suitable to all the participants. 

\textbf{Theme 4: \ep~knowledge management}

Looking at \ep~as a system with long-term access revealed the problem already
mentioned in the previous chapter -- management of \ep~knowledge. An \ep~can be
called a container of knowledge that needs to be organized. Students collect a
lot of artifacts while studying not always knowing which items to select and
where the items they put in their \ep~ should go. As the amount of information
stored increases and students move to supporting their emerging knowledge with
artifacts, it becomes more and more difficult to find items and to structure
them.

\shortquote{Student 3: At the moment my portfolio is not very big, but if I were
using it more intensively, I would imagine that to be one of the problems.
Especially, when I bring to my portfolio all the elements like my personal
stuff, my career or my hobbies.}

The majority of \ep~systems reviewed in the previous chapter provide such
functionality as tagging. However, during the interviews it was discovered that
most of the students do not find this feature useful.

\shortquote{Student 2: I don't use tags, because they don't help me... No, I
don't use them properly.}

From students' point of view, tagging did not give necessary meaning to the
artifacts. Students said that they needed something more than tags such as
\textit{``report''} or \textit{``semester 1''} which could not give much
information over a long period of time. Although, none of the participants could
come up with a solution suitable for them, they all agreed that they need system
functionality that would allow them to build a flexible structure to depict
their emerging knowledge within the \ep.

\textbf{Theme 5: Personal progress tracking}

Progress tracking was mentioned as another functionality currently not provided
by the \ep~systems developers. Progress tracking is required by students to see
their achievements and how they improved over time.

\shortquote{Student 2: I would like to see my progress as a timeline. For
example, I would like to see everything that I did at the first year of my study
and see how I progressed or how my lecturer's comments have improved or my marks
have improved. So, every artifact, everything you've done you could attach to
time and after that you can say ``what was I doing at semester 2?''}

This problem might be related to the problem of managing \ep~knowledge: unless
\ep~is properly structured, developing a required progress picture would be a
time consuming and inefficient activity. One of the solutions offered by
students was to \textit{define the areas of development and separate the
achievements by these areas} followed by to \textit{take snapshots of current
achievements and put them on a timeline}. In this case, a challenging task for
students would be to learn how to define such areas and to understand what
items and why they belong to them.

\textbf{Theme 6: \ep~sharing and communication support}

Being able to get feedback at any stage of \ep~development was very important
for students.

\shortquote{Student 3: For me the most important part of \ep~is feedback. I am
putting things in \ep~and sharing them with others because I want to see what
they think about my achievements.}

Feedback from the outside of institutional environment was no less valuable for
students than feedback from their lecturers. At the time when the interviews
took place not many \ep~supported this functionality. However, as the area of
\ep~systems development is getting more mature, the system engineers now include
sharing options by secret URL and e-mail as well as creating a temporary account
in the system. In addition to these options students mentioned a number of other
sharing requirements: access reminder notifications, pages/artifacts access
tracking, sharing history, and easy re-sharing.

In their current \ep~systems students could not find other ways of responding to
the feedback given by someone except by adding further comments. For a better
support of communication they would expect to be able to take snapshots of the
changes they make or to pin comments to the changes which would emphasize what
exactly was altered and how it addressed the given feedback.

\shortquote{Student 3: \ep~should have something like Wikipedia where I
can edit pages and save versions, see what changed, respond to the feedback
someone had given me. It is some kind of a history of why I made choices, I can
see change from here to here and I can justify that change.}

Overall, students noticed that improvements which they would like to see in
sharing and communication support were minor. However, being added to the system
they would make students' \ep~work much easier and more efficient.

\subsection{Discussion}

Although, it is assumed that looking at their education from \LLLs
perspective depends on maturity of students, the assumptions are that being
properly guided by lecturers, younger students can develop understanding of
\LLLs skills and attributes. 

The current major challenges are to make system usage connected through every
activity students do. Some students lack motivation when learning activities are
not compulsory or not graded. Others would like to use systems more often, but
found it difficult to stay engaged if the systems were not used for learning
activities on a regular basis. To address these problems, considerable changes
to the courses design and institutional policy are required. However, as this
project is focused on technical support, these changes stayed out of scope of
this research.

The remaining and more technical changes included improving long-term aspects of
\LLLs support. These meant providing suitable functionality for managing
\ep~knowledge, progress tracking and better sharing and communication support.
All these features were important for students to be able to work in the system
for their personal and professional development. 

\section{Requirements Elicitation}
\label{sec:elicit}
This section operates with the terms commonly used in software engineering for
requirements management. Based on \citet{Wiegers2003}, \textit{requirement} can
be either a capability that must be implemented in a system, description of how
the system should behave, or a property of the system. \textit{Feature} is a set
of related requirements that provides users with a capability to satisfy their
objectives or needs.

The problems described in the previous sections were translated into a set of
features to be considered for the future implementations. While the aim of this
project is to support students in \LLLsn, because learning and teaching are
closely intertwined, both lecturers and students were interviewed for
requirements elicitation purposes. This can potentially create a problem of
conflicting project stakeholder priorities \citep{Leffingwell2011} because the
majority of lecturers were focusing on their own teaching needs in providing
students with \LLLs support. Therefore, to avoid this problem, students'
requirements were favoured over the requirements of lecturers. 

Table \ref{tab:req} outlines the potential features identified through the
interviews. Features were grouped according to the major themes developed
through the interviews analysis and listed in descending order of priority based
on students' opinion.

\begin{center} \small
    \tablefirsthead{
     \hline
     \multicolumn{1}{|c|}{\textbf{\#}} &
     \multicolumn{1}{c|}{\textbf{Feature}} \\
     \hline}
    \tablehead{
     \hline
     \multicolumn{2}{|l|}{\small\sl continued from previous page}\\
     \hline
     \multicolumn{1}{|c|}{\textbf{\#}} &
     \multicolumn{1}{c|}{\textbf{Feature}} \\
     \hline} 
    \tabletail{
     \hline
     \multicolumn{2}{|r|}{\small\sl continued on next page}\\
     \hline}
    \tablelasttail{\hline} 

	\topcaption{Identified \ep~system features, improvements and additions}
    \begin{supertabular}{| c | p{12.2cm} |}

     \multicolumn{2}{|l|}{\bfit{\ep~knowledge management}} \\ \hline 
     F1.1 & Being able to organize \ep~content in a way that would reflect their
     learning\\ \hline
     
     F1.2 & Being able to establish a link between the concepts of skills
     that are learnt and the practical tasks that are done every day\\ \hline  
     
     F1.3 & Being able to add reflections to any artifact in \ep~repository that
     would suit specific purposes \\  \hline
     \hline

     \multicolumn{2}{|l|}{\bfit{Personal progress tracking}} \\ \hline
     F2.1 & Being able to set up learning goals\\ \hline
     
     F2.2 & Being able to organize data in a timeline way that would show progress
     towards the goals\\ \hline
     
     F2.3 & Be able to track and share achievements from a specific perspective
     \\ \hline
       
     F2.4 & Being able to evaluate own learning progress \\ \hline
     
     F2.5 & Being able to see the changes in reflections, marks, or feedback from
     specific perspectives \\   \hline
     \hline
     
     \multicolumn{2}{|l|}{\bfit{Improved \ep~sharing}}\\ \hline 
     F3.1 & Being able to \textit{pin} feedback to a specific part of
     shared \ep \\ \hline
     
     F3.2 & Being able to respond to given feedback \\ \hline
     
     F3.3 & Being able to point out the changes made according to the feedback
     \\ \hline
     
     F3.4 & Being able to share specific parts of \ep~with relevant audiences \\
     \hline
     
     F3.5 & Provide notifications about shared resources \\ \hline
     
     F3.6 & Provide options for easy re-sharing of information and history of
     access \\ \hline
     \hline

     \multicolumn{2}{|l|}{\bfit{Data flow between systems}} \\ \hline
     F4.1 & Data transfer (export/import) between systems\\ \hline
     
     F4.2 & Being able to provide resolution of confidentiality, ownership, and
     ethics issues \\ \hline
     
     F4.3 &  Being able to bulk export everything that has been done by a
     student from LMS to \ep \\ \hline
     
     F4.4 & Providing notifications about future changes in access to the
     systems provided by the university \\ \hline
     
     F4.5 & Being able to set up automatic synchronization of the data  \\
     \hline
     \hline

    \multicolumn{2}{|l|}{\bfit{Addressing graduate attributes}} \\ \hline
     F5.1 & Provide a way of recording institutional graduate attributes and
     other \LLLs skills \\ \hline
         
     F5.2 & Being able to add own set of skills in addition to institutional
     graduate attributes \\ \hline
     
     F5.3 & Being able to provide own understanding of the skills \\ \hline
     
     F5.4 & Being able to link the skills to examples from the personal
     repository \\ \hline
     
     F5.5 & Being able to showcase achievements connected to the skills \\
     \hline
     
     F5.6 & Being able to import institutional template with the list of graduate
     attributes \\ \hline
     \hline

     \multicolumn{2}{|l|}{\bfit{Going beyond the course boundaries}} \\ \hline
     F6.1 & Being able to link everything students learn as a way
     of understanding development of knowledge and skills \\  \hline
     
     F6.2 & Being able to point out opportunities for good examples of skills
     development\\ \hline  
     
     F6.3 & Support for a dialog between students and audience outside of
     institutional environment \\ \hline
     \hline
     
     \multicolumn{2}{|l|}{\bfit{Supporting course-related activities}} \\ \hline 
	 F7.1 & Integration between an \ep~system and other systems used in
	 university\\ \hline

     F7.2 & Being able to submit \ep~work for formal evaluation \\ \hline
     
     F7.3 & Support for a dialog between lecturers and students \\ \hline 
    \end{supertabular}
    \label{tab:req}
\end{center}

Table \ref{tab:mapping} shows analytical mapping of these features to the
recommendations and guidelines for successful \LLLs discovered in the literature
(Section \ref{sec:needs}). 

\begin{table}[hb] \small
\centering
    \setlength{\abovecaptionskip}{0pt}
	\caption{Matching features to the recommendations}
    \begin{tabular}{|p{9.8cm}|p{2.6cm}|}
     \hline
     \multicolumn{1}{|c|}{\textbf{Guidelines/Recommendation}} &
     \multicolumn{1}{c|}{\textbf{Matching Features}} \\
     \hline
    Universities should provide support for all aspects of learning [G1] & F7.1,
    F4.1, F7.2 \\ \hline 
    Students need guidance in learning [G2] & F3.1, F6.2, F6.3, F5.6, F7.3
    \\ \hline
    Lecturers should be an active facilitators [G3] & F7.3, F6.2, F3.1 \\ \hline
    Learning materials should be organized in the way that would help students
	learn how they learn [G4] & F1.1, F1.2, F6.2 \\ \hline 
    Communication and collaboration are essential parts of learning process [G5]
    & F3.2, F3.3, F3.4, F3.5, F3.6 \\ \hline 
	Learning progress should be recorded from various sources and maintained
    over a long period of time [G6] & F4.1, F4.2, F4.3, F4.4, F4.5 \\ \hline 
    Students need to be aware of their personal achievements [G7] & F2.1, F2.2,
    F2.3 \\ \hline 
	Students should develop understanding and confidence in their knowledge and
	be able to address higher-order skills [G8] & F5.1, F5.2, F5.3, F5.4, F5.5,
	F5.6 \\	\hline 
	Students should be able to evaluate and reflect on their own performance and
	learning progress [G9] & F6.1, F2.4, F2.5, F1.3 \\ \hline 
    \end{tabular}
    \label{tab:mapping}
\end{table}

This mapping shows that all guidelines were covered in the discussions with the
stakeholders. The gaps that exist in the current \ep~systems can also be
identified based on this mapping. Among these gaps, the ones that are apparent
(based on \ep~systems overview in Chapter \ref{cha:systudy}) include helping
students understand how they learn, organizing learning material and knowledge,
developing awareness of personal achievements and learning progress, improving
support for reflection on performance and communicating own learning progress.

At the next stage of this project, a prototype implementation of the required
features was aimed at supporting each guideline and was based on the features
prioritizing by students. As had already been shown in Table \ref{tab:req},
arranged in decreasing order of their priority to students these features
included the functionality related to \ep~knowledge management, learning
progress tracking, improved \ep~sharing, data flow between systems and
addressing graduate attributes.

Formal requirements specification for the implemented features and their
description can be found in Appendix \ref{app:specification} and in the relevant
sections of Chapter \ref{cha:prototype} respectively.

\section{Summary}

This chapter explored \LLLs support from the perspective of the main
stakeholders in university -- students and lecturers. The stakeholders were
interviewed to understand whether their needs coincide with the guiding
principles and recommendations outlined in Chapter \ref{cha:litrev}. All
participants were invited to analyse the \ep~systems functionality, express
their needs and suggest improvements or new features that would help to provide
better support for students in \LLLsn. Information obtained from the interviews
was translated into potential features each mapped to the recommendations for
successful \LLLsn. It allowed this project to move to the next stage of
implementation of the prototype functionality. The next chapter will discuss
design and development of this functional prototype.
