\chapter{Stakeholders Requirements for \LLLc Support in Universities
\label{cha:model}}
\chaptermark{Stakeholders Requirements}

This chapter describes the results of the interviews conducted with the
stakeholders to support literature findings.


\section{Lecturers' Perspective on \LLLc Support}
\subsection{Participants Profile}
Using a theoretical or criterion sampling strategy \citep{Byrne2001,
Warren2001}, this phase of the project looked for the respondents who met
certain criteria of being academics with previous experience of using
e-learning systems, \ep~systems in particular, in their teaching. Ten academics,
mostly the participants of a previous institutional
\ep~initiative\footnote{\url{http://science.massey.ac.nz/eportfolios}}, were
approached by email and invited to participate in this research project. Nine
academics from various sections of the university (College of Business, College
of Education, and College of Sciences) accepted the invitation and agreed to be
interviewed. Nine in-depth interviews were conducted in April-May 2010 to gather
the data required for the analysis. All interviews were audio recorded and
transcribed to make follow-up analysis easier and more thorough.

\subsection{Methodology}

In-depth semi-structured face-to-face interviews were used as an instrument for
this phase of the research. It was favoured over the other research methods for
the following reasons. This kind of interview has an open style while remaining
structured \citep{Gillham2000} which gives an interviewer freedom within the
predefined framework. Due to its openness and flexibility semi-structured
interview has become widely applicable and popular research instrument.

Face-to-face interviews have more quality advantages than telephone ones
\citep{Shuy2001}, however this characteristic adds a restriction to the number
of people that are feasible to interview. According to \citet{Gillham2000},
face-to-face interviews require high costs/time and potential interviewees
accessibility which makes it necessary to keep the number of participants to a
minimum, just to cover representativeness. Therefore, every participant becomes
a \textit{key} informant in the project.

Depth of meaning is considered to be central in the interview.
\citet{Johnson2001} described the benefit of in-depth interviewing as the one
that gives an interviewer an opportunity to achieve deep understanding inherent
to participants in some daily activity and as well allows to get the multiple
perspectives on this activity.

Due to all mentioned characteristics and being able to gain insight into the
experience and knowledge of others \citep{Schostak2006}, in-depth
semi-structured interviews were considered to be a suitable instrument for this
project phase. It was important for this project as it had a limited pool of 
potential participants that would satisfy the selection criteria and, in this
case, the required depth of investigation could not be reached with
questionnaires.

The interviews were guided by a number of scenarios each described a particular
situation connected to the problems of \LLLs support in universities (See
Appendix \ref{cha:interviews} for the scenario examples). The scenarios
described situations from the teaching perspective, to support talking to
lecturers from their perspective. Topics for these scenarios were selected from
the literature review conducted at the first stage of the project and
observations from the reports and communications with the College of Sciences
\ep~Initiative project team
members\footnote{\url{http://science.massey.ac.nz/eportfolios/contacts.asp}} at
Massey University.

\subsection{Findings}

Issues and gaps identified during the interview discussions were analysed and
themes to group these issues were developed. The direct quotations in this
section were taken from interview transcripts.

\textbf{Theme 1: Integration with LMS for supporting course-related activities}

All participants said that including development of \LLLs skills in university
programs changes the way they teach, interact with students, and use e-learning
systems. Talking about the latter, lecturers want to see the systems they use
integrated into their regular activities. This means that systems should not be
external to the teaching and learning process, but be a part of it. Systems
should not be seen as creating extra work or adding to workload. However, at
this stage, from staff's point of view, all their experience of using e-learning
technology is about producing their workload.

\shortquote{Academic 1: For us \ep, with all its blogs, pages, views,
presentations, or anything else, will never get any traction until it behaves
like an electronic assignment.}

Lecturers see the importance of introducing students to various \LLLs skills.
They say that the development of \LLLs skills should be integrated through
everything that students do and supported at the system level as well. They
emphasize that for students the current systems look like two separate worlds.
These worlds both provide some pieces of activities, but cannot be seen as a
whole.

\shortquote{Academic 2: Unless there are no layers built in each year in
connection, students will see that this [work with systems] is really
disconnected.}

Therefore, they would like to see an \ep~system and LMS integrated in a way
such that students would be able to do simple, but valuable activities as a part
of their learning process rather than in addition to it. For example, this may
include introducing students to the importance of being reflective about the
choices and efforts they make; writing reflections or keeping reflective
journals in \ep~system with an opportunity of submitting everything they do as
an assignment to the LMS; getting feedback on their assignments directly to
\ep~after they were marked; and reflecting on this feedback afterwards. From the
perspective of a combined systems approach, this will allow students and
lecturers to work in the system that suits their needs best and provides support
for various activities, like developing and sharing content, assessing,
reflecting, or giving feedback.

Talking about feedback, there is also a need for a better support of this aspect
of communication between students and lecturers. Lecturers want to give students
their feedback on various kinds of work like assignments, reflections and
discussions. The majority of academics do not see any problems if students
decide to copy the teacher's feedback to their personal \ep.

\shortquote{Academic 3: For me they have shared their learning with me, so
whatever I put in there from my perspective that is for them to do whatever they
want to do with that. I don’t have a problem with them keeping it or putting it
elsewhere, because if I did, I would not put it in there. For me it is about
helping them to learn.}

What is really important for lecturers in the process of communicating with
students through feedback is student's response. They want to see that their
feedback is not ignored, but is listened to by students and used to improve
their learning outcomes.

\shortquote{Academic 1: Teachers don't like giving feedback if it is ignored.
What would be interesting is if the students had to respond to the feedback that
the staff member gave. Creating some kind of dialog between staff and students.
Conceptually, it is really important that this dialog is going on. It motivates
staff members when they see their feedback is being listened to.}

In summary, academics are looking for the e-learning environment that will allow
students and lecturers to make various educationally valuable tasks so easy,
that they are done by both, on a regular basis.

\textbf{Theme 2: Addressing graduate attributes}

All interview participants agreed that introducing students to graduate
attributes/profiles at an earlier stage of their study is important. It gives
students one complete picture of what kind of graduates they are expected to
become and gives them an opportunity to integrate their knowledge from various
sources other than only from the degree programme.

\shortquote{Academic 3: It is good for students to know what is expected from
them at the end. So they can understand how the assessment and what they do puts
together. Lots of students see everything just like the course that they have to
pass, rather than a coherent course of study that has a reason for being
together. For example, why are the communication skills so important? -- Because
at any job later you will need to be able to talk to people, to be able to
write, because these things are fundamental.}

From the lecturers' point of view, graduate attributes are a balance between
what is required from the university to \textit{produce} good graduates and
what, primarily, has to be achieved by students to become lifelong learners. It
was also suggested that graduate attributes are a challenge to introduce. They
have to be really thought through and integrated right through everything
students do.

\shortquote{Academic 4: They [Graduate Attributes] are a challenge to set.
Students need a bigger picture to go really beyond the university because there
are so many timeframes and stages beyond the university. It will help students
to start understanding that they got lots of talents. They can link it to
examples, which is what gets lost when there is so much going on.}

Lecturers see graduate attributes as a potential solution to the problem that
students do not consider their formal learning as a part of their \LLLsn.
Lecturers believe that graduate attributes might help students to focus beyond
what is required for a particular programme of study.

Lecturers also think that it is important to give students an opportunity to
develop their own set of graduate attributes, in addition to institutional ones.
This would allow them to describe their aspirations and help them to define
their future goals. However, this should be done carefully as students still
require guidance and support.

\shortquote{Academic 6: So, graduate attributes, I think, are positive thing.
They are just a wider example of what we do now by intended outcomes for each
paper. We should be absolutely explicit of what our students are able to do when
they finish. I think, we want to give students freedom -- absolutely, but not
absolute freedom. It is a freedom within a structured process that says ``You
have to get from here to here, and it is up to you which way to got, but still
you have to get from here to here''.}

From the systems perspective, several respondents suggested to provide students
with a comprehensive template that would define a set of generic \LLLs skills,
as well as skills specific to their programme of study. This template should be
populated by students with various examples from their \ep~as they move across
their study supported by the lecturer's guidance and feedback. It should be up
to university policy when and how to evaluate graduate attributes, but grades
should not be kept in the \ep~system. Lecturers think that only feedback or
references should go to the students' personal \ep~space.

One of the respondents was critical about a template approach. According to his
point of view, there might be a danger that students would see a template as a
simple \textit{ticking checkboxes} activity. To avoid it, there should be a way
of supporting each example with a detailed explanation or reflection on
experience gained.

\textbf{Theme 3: Going beyond the course boundaries}

One of the respondents described LMS and \ep~integration as bringing two
worlds into one:

\shortquote{Academic 1: It should create two worlds in one for them: content
world and the world where they do their learning. Last world should give them
the confidence to explore whatever they want to. This world should be within the
programme and beyond the programme somehow.}

From a lecturer's perspective, the LMS is a world of content which they deliver
to students and a world where students' formal education happens with such
course-related activities as submitting assignments for marking or discussing
course-related problems on forums. However, in the majority of LMS, students
lose access to the items they have created when they finish a course section or
graduate.

All interviewees think that it is important for students to retain access to the
work that they do. Some stated that due to the institutional LMS policy they
often recommend that their students save their work externally in order not to
lose it later.

To deal with this problem, lecturers want LMS and \ep~systems to be integrated
in a way that would allow for easy data transfer between systems. Integrated
systems should give students an option of transferring all items at once or
copying them one by one.

While discussing data flow options, some concerns about copying collaborative
work were revealed. For example, when students participate in discussions on a
course forum, is it ethical to allow them to copy posts by other participants
into a personal \ep? Should other participants' names be substituted when
data transfer is performed? The majority of academics said that the best
solution for them would be a general policy for the forum which would tell
students before they join the discussion that their posts might be copied to
another participant's \ep. In this case, any participant in the
discussion should have an option whether they want to become anonymous or not,
when the discussion is copied from one system to another.

One more problem, mentioned by the academics, was about involving feedback from
the outside. Although some systems may provide access to ePortfolio pages for
users who have no account in the system through a secret URL, they say that that
is not enough. Half of the respondents do not like secret URLs, as they are less
secure than having an account in the system. As an option, in addition to a
secret URL, they would prefer to have a temporary account to be created when
students share their ePortfolio with someone by e-mail.

\textbf{Theme 4: Non-functional and other issues}

Although, the aim of our interviews was to identify the gap and shortcomings in
existing systems, other issues, not all connected to technical ones, were
discussed.

Academics noticed that a lot of students struggle with technology. Therefore,
any system developed for students should have highly intuitive and visually
appealing interface which will satisfy novice users as well as advanced users.
Where it is possible tasks, which users want to do, should be done
automatically.

\shortquote{Academic 5: Some people struggle with technology. There should be
something that they understand quite clear. I think the less you ask them
[system users] the better. If systems are really integrated, as much as possible
should be done automatically.}

\shortquote{Academic 1: I think it's all about integration. There are so many
tools there already in both, social world as well as institutional. In this
project ease of use is going to be crucial. You should bring everything together
and let the systems do what they do best.}

Students should be provided with the example of good \ep, so that they
could see what \textit{a good \ep} means, what features make it
\textit{good} and how it can be developed.

\shortquote{Academic 2: I think that students still don't see a sample
\ep~and how it could work for them. Unless it is clear for them that
\ep~can be used this way, they will see this as an extra piece of work
and ask ``what would be the benefit?''}

Students and lecturers require institutional support. Students need to be taught
to use the system as well as to understand the general concept of the \ep~
process. It is important to show them how \ep s can connect course-related
activities with their lifelong learning; Lecturers, in turn, need support from
universities on how to include the practice of developing lifelong learning
skills to the teaching process.

\shortquote{Academic 6: \ep~for \LLLs is a very good idea, but you have to
introduce it at first year first semester at the university. So, when students
are at the end, they have like a record in \ep~of what they have done. And it's
just a tool you learn like learning to use LMS. But to do that, it should be
integrated into the teaching.}
 
\subsection{Discussion}
Results showed that lecturers look at \ep~systems as a part of learning
environment in connection to LMS. It was resonable because they are focusing on
efficient teaching and making their work easier at the same time.

The feedback given to the scenarios was generally congruent with the information
we had found in the literature. The interviews showed that academics see and
understand the value of \LLLs skills. They are willing to incorporate activities
aimed to develop these skills into their teaching as soon as they are supported
on various levels. They see the support on a system level as an important part
of \LLLs support in the university.

The majority of the participants said that the scenarios we showed to them are
realistic and describe the problems of lifelong learning support from various
angles. Scenarios were mostly used to introduce the general topic and to start
the discussion. The participants tended to give a lot of information on
potential solutions. Therefore, it was assumed that they have encountered
similar problems and have been thinking about ways to deal with these.

Academics said that the main challenge in providing system support for lifelong
learning is to make it connected through every learning activity students do.
They think that student’s acceptance condition for such an e-learning
environment will be the ease with which they can move between systems while
doing regular learning tasks. According to the academics’ experience, students
expect an immediate reward for everything they do. Students lack motivation if
learning activities are not compulsory or not marked. They also have problems
with seeing the bigger picture of their personal and professional development,
focusing rather on passing the courses that lead to degree qualification.
Therefore, academics see the value of using LMS and ePortfolio systems for
lifelong learning support by showing students the opportunities for their
development as lifelong learners. An integrated environment would let students
gather the small parts demonstrating their development and so contributing to
the bigger picture of \LLLsn.

\section{Students' Perspective on \LLLc Support}
\subsection{Participants Profile}

A snowball or chain sampling strategy \citep{Mack2005, Marshall2010} to identify
potential participants for the interviews. The lecturers, who have participated
in the interviews at an earlier stage of the project, were asked to provide the
researcher with student contacts or to inform students of this research in their
classroom. As the use of \ep~is relatively new at Massey University, this
technique was used to make sure that students who took part in the interview
were familiar e-learning systems such as LMS and \ep. It was anticipated that
students who have already used both systems in practice are more experienced and
provide richer data for discussion and analysis. About 30 students were
approached by email and invited to participate in the research project. Overall,
nine students from various schools and colleges of Massey University accepted
the invitation and agreed to be interviewed. Nine in-depth interviews were
conducted in May-September 2010 to gather the data required for the analysis.
The procedure was similar to the interviews with the lecturers: all interviews
were audio recorded to make a follow-up analysis easier and more thorough.

\subsection{Methodology}

Methods used to explore \LLLs support in universities from the student's
perspective were similar to the methods described in the previous section.
In-depth semi-strucutred interviews were used to look at the current problems
and challenges in the area. As this research was focused on system support, the
problems in the teaching process and institutional policies were not among the
topics for discussion during the interviews with the students. The aim was to
understand the gaps and shortcomings that existed in e-leaning environments, as
well as to get the interviewees' views on what is needed to make the systems
already used in universities become more efficient and adequate for students'
\LLLs support.

Another goal of the inteviews was to get feedback on potential system features
proposed to students, based on the literature review and gaps analysis. The
interviews were guided by a number of scenarios (see Appendix
\ref{cha:interviews} for the scenario example) constructed to described the
situations connected to the problems of \LLLs support in universities from the
learner's perspective.

% The participants were asked a set of open-ended questions to elicit their views
% on the environments that support \LLLs in the university. They were
% asked how they had been using LMS and ePortfolio together and what advantages
% and disadvantages they saw in such a combined approach. We asked for their
% opinions on what features their current ePortfolio system lacked, or which
% features could make such system more useful and relevant to lifelong learning
% support in universities. The interviewees were also invited to suggest
% functionality they would expect to see in a good e-learning environment that
% supported lifelong learning in universities.

\subsection{Findings}

Analysis strategies suggested by \citet{Marshall2010} were used to analyse
interviews data. The issues and gaps identified during the interview
discussions and were grouped into themes.

\textbf{Theme 1: Supporting course-related activities vs. going beyond the
course boundaries}

Students explained that without a proper course design it is very difficult for
them to look beyond the course boundaries.

\shortquote{Student 6: Everything we learn is disconnected. As a student, I
expect a reward for what I am doing. Probably, if lecturers could make it so
that this reward would connect with the aim of long-term development, it would
be helpful.}

Students like using \ep~for their learning activities. They find some
things (like reflection or selection of proper examples) quite challenging, but
at the end they admit that they understand the value of what they were doing. In
order to use combined ePortfolio and LMS students want them to be integrated
through all tasks in their learning process in the university. For example, this
may include keeping reflective journal in ePortfolio system with an opportunity
of submitting it as an assignment to the LMS; getting feedback on assignments
directly to ePortfolio after they were marked; and reflecting on this feedback
afterwards. From the systems perspective, this will allow students and lecturers
to work in the system that suits their needs best and provides support for
various activities, like developing and sharing content, assessing, reflecting,
or giving feedback. 

Unfortunately, without proper integration and until the way academics develop
study program and construct learning activities changes, the situation will
remain the same:

\shortquote{Student 1: We are not required to use it [ePortfolio] this year, so
I just cannot force myself to go there and keep it updated for my study because
it is not integrated. If it was, I would be more likely to do that.}

\textbf{Theme 2: Addressing graduate attributes}

All students agreed that graduate attributes would be a good approach that might
show them the full picture of their study. However, as it was mentioned in
previous theme, achieving these attributes should be connected through every
activity that students do in their study.

\shortquote{Student 2: I think that it's really good that there are certain
things you need to have achieved and you commit to achieve it, but it should be
in course design.}

From the systems perspective, students would like see a comprehensive template
in ePortfolio system that would define a set of generic lifelong learning
skills, as well as skills specific to their programme of study. This template should be populated with various examples from LMS and their ePortfolio space as students move
across their study supported by the lecturer’s guidance and feedback. It is important for students to be able to
add their own set of skills to the list as it would help them to set up their personal development goals and
understand what they want to achieve. Students think that when completed this kind of template might be helpful
in their job, scholarship or any other applications.

\textbf{Theme 3: Data flow between systems}

In addition to the previous themes, to emphasise the importance of integrating LMS with ePortfolio
system, we came to the conclusion that students see these systems as two different worlds.

\shortquote{Student 1: Most of my time these days is spent in LMS. My readings
are all there, my assignments are all there, and my forums are there. That’s where I work. So, if ePortfolio were a part of my e-learning
environment in a seamless manner, I will be more likely to use it of my own
initiative.}

Students find it is time consuming to work in two separated environments. It is difficult to keep track of
what is stored in each system and difficult to transfer from LMS items like assignments feedback or forum posts
that are potential good examples to showcase in their ePortfolio.

\shortquote{Student 7: My current course is my focus and that is the most
important thing to me at the moment. But after I finish it, I want to be able to
wrap it up and save to ePortfolio where I can showcase what I've done.}

Moreover, in LMS students tend to lose access to their courses once they are finished and they lose
complete access to LMS once they graduate. As a result students cannot treat LMS as a space suitable for their
personal development and, without being able to transfer everything they have done during their course study,
students lose a lot of valuable examples of their achievements.

\shortquote{Student 4: My supervisor just told me to save everything, so that
when LMS is closed for me, at least I will have information saved. So, I did it
with every single thing I had.}

To deal with this problem, students want LMS and ePortfolio systems to be integrated in a way that
would allow for easy data transfer between systems. Integrated systems should give students an option of
transferring all items at once or copying them one by one. Also while discussing data flow options, some
concerns about ethics of transferring collaborative work were revealed, such as copying course forum posts
made by other participants into a personal ePortfolio.

\shortquote{Student 3: If we are discussing something with someone on the forum,
I treat their ideas as their intellectual property. So, if I want to use these ideas, I will ask for their permission and would expect the same
from their side.}

The majority of students said that the best solution for them would be a general policy for the forum
which would tell students before they join the discussion that their posts might
be copied to another participant’s ePortfolio. In this case, any participant in the discussion can chose whether they want to become anonymous or
not, when the discussion is copied from one system to another.


\textbf{Theme 4: \ep~knowledge management}

Looking at\ep~as a system with long-term access reveals the problem already
mentioned in the previous chapter -- management of \ep~knowledge. \ep~is like a
container of knowledge that needs to be organized. Students collect a lot of
artifacts while studying and they do not always know where these artifacts
belong to. As the knowledge emerges it becomes more and more difficult to find
items and to structure them.

\shortquote{Student 3: At the moment my portfolio is not very big, but if I were
using it more intensively, I would imagine that to be one of the problems.
Especially, when I bring to my portfolio all the elements like my personal
stuff, my carrier or my hobbies.}

The majority of \ep~systems reviewed in the previous chapter provide such
functionality as tagging. However, during the interviews it was discovered that
most of the students do not find it useful.

\shortquote{Student 2: I don't use tags, because they don't help me... No, I
don't use them properly.}

From students' point of view, tagging does not give necessary meaning to the
artifacts. Students said that they need something more than tags such as
\textit{report} or \textit{semester 1} which will not give much information over
a long period of time. Although, none of the participants could come up with a
solution suitable for them, they all agreed that they need system functionality
that would allow them to build a flexible structure out of the \ep~knowledge.

\textbf{Theme 5: Personal progress tracking}

Current \ep~systems do not provide a suitable progress tracking
functionality which is required by students to see their achievements.

\shortquote{Student 2: I would like to see my progress as a timeline. For
example, I would like to see everything that I did at the first year of my study
and see how I progressed or how my lecturer's comments have improved or my marks
have improved. So, every artefact, everything you've done you could attach to
time and after that you can say ``what was I doing at semester 2?''}

This problem might be related to the problem of managing \ep~knowledge: unless
\ep~is properly structured, it will be difficult to develop a required progress
picture. One of the solutions offered by students was to \textit{define the
areas of development and separate the achievements by these areas} followed by
to \textit{take snapshots of current achievements and put them on a timeline}.
In this case, a challenging task for students will be to learn how to define these areas and to
understand what items and why belong to them.

\textbf{Theme 6: ePortfolio sharing}

Being able to get feedback at any stage of ePortfolio development is very
important for students. 

\shortquote{Student 3: For me the most important part of ePortfolio is feedback.
I am putting things in ePortfolio and share them with others because I want to
see what they think about my achievements.}

Feedback from the outside of institutional environment is no less valuable for students than feedback
from their lecturers. Currently not many ePortfolio support this functionality. Although some systems may
provide access to ePortfolio pages for users who have no account in the system through a secret URL, students
say that that is not enough. Half of the respondents do not like secret URLs and called them less secure than
having an account in the system. As an option, in addition to a secret URL, students would prefer to have a
temporary account to be created when they share their ePortfolio with someone by e-mail. Later they would like
to be able to tracking access to ePortfolio and see whether it was viewed.

In current ePortfolio system students cannot see the way to properly respond to the feedback given by
others. They would like to be able to show that feedback is not ignored, but used to improve their learning
outcomes.

\shortquote{Student 3: ePortfolio should have something like Wikipedia where I
can edit pages and save versions, see what changed, respond to the feedback
someone had given me. It is some kind of a history of why I made choices, I can
see change from here to here and I can justify that change.}

\subsection{Discussion}

During the interviews it was discovered that older (\textgreater 25 years old)
and more mature students are better aware of the \LLLs skills than younger
students. These students are trying to look at their study at university from
\LLLs perspective, while younger students admitted that they see their study as
getting marks and passing papers rather than developing skills. This seems to be
a coomon problem among students mentions during the interviews with lecturers as
well.

So, it can be assumed that looking at their education from \LLLs perspective
depends on maturity of students, but being properly guided by lecturers, younger
students can develop understanding of \LLLs skills and attributes. The major
e-learning environment challenge at the moment is to make system usage connected
through every activity students do. Some students lack motivation if learning
activities are not compulsory or not graded. Others would like to use systems
more often, but find it difficult to be engaged if the systems are not used for
learning activities on a regular basis. 

After analyzing ePortfolio system as a long-term tool, students think that it
does not provide enough functionality for managing ePortfolio knowledge,
progress tracking and sharing. All these features are important for students to
be able to work in the system for their personal and professional development.
Therefore, currently a lot of improvements are required before ePortfolio system
can be accepted by students as a primary tool for \LLLs support.

\section{Requirements Elicitation}

The themes described in the previous sections were translated into a set of
formal requirements for the future implementation.

Table \ref{tab:req} summarizes the requirements developed to address various problems
identified through the interviews:

\begin{table}[htb]
  \begin{center}
    \begin{tabular}{| l | p{6.5cm} |}
    \hline
     \multicolumn{1}{|c|}{\textbf{Guidelines}} &
     \multicolumn{1}{c|}{\textbf{Description}} \\
     \hline
     Some & Some \\ \hline
     Some & Some \\ \hline
     Some & Some \\ \hline
    \end{tabular}
  \end{center}
  \caption{Table}
  \label{tab:req}
\end{table}

% A good e-Portfolio system will allow the concurent presentation of differing
% artefacts to different audiences, but at the same time saving duplication of
% materials as all artefacts will be hosted together.

\section{Summary}

This chapter explored \LLLs from the perspective of the main stakeholders --
students and lecturers.

The interviews with lecturers on the topic of \LLLs support resulted in a lot
of valuable information. The feedback given to the scenarios was generally
congruent with the information we had found in the literature. The interviews
showed us that academics see and understand the value of lifelong learning
skills. They are willing to incorporate activities aimed to develop these skills
into their teaching as soon as they are supported on various levels. They see
the support on a system level as an important part of lifelong learning support
in the university.
