\chapter{Evaluation\label{cha:evaluation}}
%2484
Throughout this thesis the requirements, design and prototype implementation of
the \ep~system that can support \LLLs has been discussed. This chapter focuses
on the evaluation of this research and its contributions based on three studies.
The studies were designed to look into the specific aspects of \LLLs support and
understand how well developed features satisfy the requirements identified
earlier by the lecturers and students. 

The number of studies represents perspectives of all of the stakeholders that
have been involved in this research earlier, i.e., lecturers and students.
Evaluation from the students' perspective has been split into two studies as the
requirements elicitation stage discovered that the perception of \LLLs depends
on the maturity of students. Therefore, it was decided to use different
evaluation approaches for each group of students.

The first section of this chapter describes the approaches and data collection
methods used in the evaluation stage. Further, each of the three studies is
presented with the detailed participants profile description, exercise protocol
followed by the study, artifacts collected and the results of data analysis.
Each study section ends with conclusions that look into recommended improvements
to the \ep~system prototype and its processes.

\section{Design Overview}
As this chapter aims to evaluate this research and its contributions, the
relevant research question and its sub-questions are restated here:

\textit{Does the extended environment meet the needs of the stakeholders in
university teaching and learning contexts?}

\textit{\begin{itemize}
  \item How can lecturers use new features to provide students with their
  guidance and help them to understand \LLLs skills?
  \item How can students address \LLLs skills using new features?
  \item How can new features help students track their learning progress, manage
  \ep~knowledge and content, demonstrate and share their achievements with
  others?
\end{itemize}}

To answer these questions, three studies were carried out independently from
each other. The results were used to evaluate the developed prototype from three
different perspectives: lecturers, mature students and less experienced
students. Each study followed its own exercise protocol described in the related
sections. Detailed design of each study can be found further in this chapter in
Sections \ref{sec:one}, \ref{sec:two}, and \ref{sec:three} respectively.

Data collection for analysis was performed using both quantitative and
qualitative methods to support the principles of multiple sources and multiple
perspectives of data advocated by many researchers
\citep{Yin2009,Maimbo2005,Marshall2010}. The following techniques were used over
the course of all studies:

\begin{itemize}
  \item Behavior observations made by the researcher during all studies.
  \item Observation data in form of digital photographs taken during the group
  experiments.
  \item Audio recordings of the face-to-face interviews collected to facilitate
  more thorough interview analysis.
  \item Open-ended questions asking for participants opinion on the features and
  tools used in the studies.
  \item Close-ended questions that required participants' evaluation based on
  ten point scale, from \textit{not useful at all} to \textit{highly useful}, 
  included in exit questionnaire.
  \item Physical artifacts in form of paper records made by participants of the
  group experiments.
  \item Digital artifacts in form of electronic records in the \ep~system made
  by participants of the groups experiments and case studies.
  \item System logs demonstrating \ep~system use by case study participants.
  \item Additional evidence, comments and opinions collected by the researcher
  through informal discussions that would help to support analysis and
  conclusions.
\end{itemize}

Audio recordings as well as digital photographs were collected with the
permission of the participants and performed without interrupting the process of
each study. More detailed description of data collection methods for each study
can be found in the relevant sections this chapter and associated appendices. 

\section{Ethical Considerations}

Similarly to the earlier stage of the requirements elicitation, the Massey
University Ethics Approval process was followed for the evaluation studies of
this research. Analysis of the evaluation design with the Human Ethics Chair
concluded that none of the three studies required full ethical approval.
Therefore, a ``Low Risk Notification" was submitted to the Massey University's
Low Risk Database.

Ethics documentation, such as the Ethics Approval Letter, participants'
information sheets and consent forms, used for this research stage can be found
in Appendix \ref{cha:app5}.

\section{Study One. Exploratory Evaluation by Lecturers}
\label{sec:one}

This study\ldots

\subsection{Goals}

The goal for this study was:

\smallquote{Write a goal}

\subsection{Research Protocol}

This study did not follow strict research protocol. It was rather a semi-formal
discussion with the participants where they tried

Each participant was demonstrated the developed features with the description of
their intended use.

\subsection{Participants Profile}

Techniques used to identify participants for this study consisted of snowball
technique + previous participants.

Lecturers who had experience of using an \ep~system with their students were
approached by an informal invitation to evaluate developed features based of
demonstration and their personal experience.

Overall the data was gathered from eight participants, all lecturers at
various schools and colleges of Massey university.

Demographics of the participants:

College of Science - three senior lecturers (two engineering and one veterinary
sciences) one of which is also a program coordinator.

College of Education - four lecturers in personal and group interviews during
the semester staff meeting.

A learning consulting representative who assists adoption of learning
technologies in university courses - 1

\subsection{Results}

Demonstration + Personal interviews with participants.

What they liked:
- Visual nature of concept mapping
- simplicity of making complex things
- challenge of conceptual understanding
- authentic evidence that students meet the outcomes

Potential challenges:
- training of lecturers as well as students 
- assistance throughout the entire process
- 

\subsection{Conclusions}

After the formal part, five out of nine participants asked when they could
expect to see the demonstrated features in the official release of the
\ep~system that they had been using in the university and expressed willingness
to include these into their work with students. Arguably, this can be considered
as an indicator of value and usefulness of the developed features recognized by
the participants during the demonstration.

\section{Study Two. Group Experiments -- \LLLc Skills Development and
Demonstration}
\label{sec:two}

This study investigates students perception of the concept mapping tools as a
part of the \ep~system in terms of constructing and sharing knowledge,
understanding and demonstrating achievements, and linking practical experiences
to the conceptual skills.

\subsection{Goals}
The goal for this study was:

\smallquote{To find out whether undergraduate students find concept mapping embedded
into the \ep~system helpful in terms of addressing graduate attributes, learning
objectives and \LLLs skills and tracking their progress in learning.}

To support achievements of this goal the main objectives were:

\begin{itemize}
  \item Determine whether concept mapping embedded into the \ep~system provides
  students with a suitable tool for addressing graduate attributes and \LLLs
  skills;
  \item Investigate whether developing of concept maps influences students'
  understanding of personal learning achievements and skills that they develop
  during degree program study;
  \item Determine whether concept mapping embedded into the \ep~system can be
  successfully used to demonstrate personal achievements and skills;
  \item Analyse how useful students' consider concept mapping methods as a part
  of the \ep~system.
\end{itemize}

\subsection{Research Protocol}

To ensure a better control over experiment settings, it was conducted with the
small groups of students at a time (35 participants in total). All groups
followed the similar process outlined in Study Protocol in Appendix
\ref{cha:app6} with up to two hours for the entire experiment to complete. Time
of completion might have varied between groups due to the differences of
participants' previous experiences. For example, some of the students did not
require explanation of \LLLs concepts or tutorial on how to construct concept
maps.

Work with each group of students began with an introduction to the research,
activities to be performed and an explanation of participants' rights.
Ethics documents, such as the information sheet and consent form, were
distributed. Signed consent forms to participate in the experiment had been
collected before any other activities and exercises were started.

The introduction included a brief overview of the research project, a
presentation on \LLLs concepts, demonstration of the \ep~system and concept
mapping as a method of constructing knowledge.

After introduction part, each participant received pen and paper to work with
the first exercise, examples of institutional attributes and courses learning
objectives, examples of concept maps, unique access account to the prototype
\ep~system, user manual for the \ep~system concept mapping tool and artifacts
fragments extraction, and exit questionnaire.

Second part of the experiment consisted of two exercises: the first one
performed on paper and the second one performed using the \ep~system tools. At
the end, all participants were asked to complete the exit questionnaire to
collect their opinion on the tools and methods they had just used.

\begin{figure}[htb]
\centering
\includegraphics[width=1\textwidth]{CH7-F7-Settings}
\caption{Experiment settings}
\label{fig:settings}
\end{figure}

Figure \ref{fig:settings} shows the settings in which the experiment was
performed. Standard university computer classrooms were used for the experiment.
Participants did not have allocated places which allowed them to choose more
comfortable settings for the duration of the experiment. The researcher was
available at any time during the experiment for assistance.

Although, working independently during the experiment was highly encouraged,
students were allowed to do exercises in pairs or groups up to three. In case
students were working in pairs, they still had to complete the exit
questionnaire independently.

\FloatBarrier

\subsection{Participants Profile}

As it was already mentioned in the previous section, thirty five students from
various undergraduate programs at Massey University agreed to participate in
this experiment. Table \ref{tab:degree} shows the wide range of degree programs
the students were enrolled in.

\begin{table}[htb]
  \begin{center}
    \begin{tabular}{| l | c |}
    \hline
     \multicolumn{1}{|c|}{\textbf{Degree Program}} &
     \multicolumn{1}{c|}{\textbf{Number}} \\
     \hline
     Bachelor of Engineering & 2 \\ \hline
     Bachelor of Health Science & 1 \\ \hline
     Bachelor of Science & 3 \\ \hline
     Bachelor of Veterinary Science & 7 \\ \hline
     Graduate Diploma in Secondary Teaching & 14 \\ \hline
     Graduate Diploma in Primary Teaching & 7 \\ \hline
     Bachelor of Information Sciences & 1 \\ \hline
    \end{tabular}
  \end{center}
  \caption{Degree programs of the experiment participants (n = 35)}
  \label{tab:degree}
\end{table}

All participants were volunteers recruited through the announcements in the
classroom of various courses in College of Education and College of Science or
through the poster invitation distributed through the lecturers at Massey
University.

As shown in Figure \ref{fig:demograph}, participants' age ranged largely from
twenty to thirty years which was expected as this experiment aimed on studying
undergraduate students. Gender distribution was relatively equal with fifteen
male and twenty female participants.

\begin{figure}[htb]
\centering
\includegraphics[width=1\textwidth]{CH7-F5-Demographics}
\caption{Participant demographics (n = 35)}
\label{fig:demograph}
\end{figure}

Twenty seven students reported that they were familiar with the concepts
of \LLLs before the experiment. However, only thirteen of these twenty seven
students said that they were familiar with the term \textit{graduate
attributes}. They were primarily Teaching Diploma students. As it was discovered
in informal post-experiment discussion, this might be explained by the fact that
College of Education of Massey University uses term \textit{teaching profile}
instead of \textit{graduate attributes} to describe \LLLs skills and
competences.

In the background section, twenty nine students said that they knew about
\textit{\ep} prior to the experiment. Seven of these students reported using an
\ep~system to demonstrate their \LLLsn. They were Veterinary Science students
who had \ep~work included into their degree program curriculum.

\subsection{Activities and Artifacts}

Figure \ref{fig:procedure} provides an overview of the experiment procedure and
the activities or tasks performed for each part.

\begin{figure}[htb]
\centering
\includegraphics[width=1\textwidth]{CH7-F8-Activities}
\caption{Experiment procedure}
\label{fig:procedure}
\end{figure}

Length of the presentations and information content of the introduction part
varied depending on the participants' experience.

List of the artifacts used by the students in this experiment was following:

\begin{itemize}
  \item Ethics documents: a) information sheet, and b) consent form;
  \item Pen and paper for the first concept mapping exercise;
  \item Examples of institutional attributes and courses learning objectives
  taken from the Massey University web-site;
  \item Examples of concept maps in the \ep~system created by the researcher;
  \item Instruction sheet with the unique access account to the \ep~system;
  \item User manual for the \ep~system concept mapping tool;
  \item User manual for artifacts fragments extraction;
  \item The exit questionnaire.
\end{itemize} 

Appendix \ref{cha:app6} provides samples of documents used in the experiment.
Participants' responses from the exit questionnaire can be found in Section
\ref{sec:responses} of this appendix.

\subsection{Data Analysis and Results}

The data collected during this evaluation were based on questionnaire results,
observations and system records.

\subsubsection{Observed Behaviour}

Based on observations, the most difficult for students was the very beginning of
each of the exercises. In the informal discussion after the experiment, some
of the students admitted that having a clean sheet of paper or an empty
\ep~system account in front of them as they started was rather intimidating. At
first, students expected to be told what to draw and which concepts to include
into their diagrams. After a short explanation that there were no right or wrong
concept maps that could represent their own learning experience or skills, the
process of working with concept maps moved on.

Students, new to the concept mapping, followed the techniques presented in the
introduction tutorial on how to build the maps. This included deciding the main
message or key concept of their map followed by identifying the related
concepts. When this was done, they tried to organize the concepts into maps.

% TODO: photo of the concept map on paper

More experienced students preferred to follow their own established procedure of
creating concept maps. Some of them did not spend time making a list of the
concepts, but were adding concepts straight to the maps making corrections when
necessary.

Due to the lack of information about previous experiences of the participants,
in some cases groups consisted of students with very different knowledge of the
discussed concepts. The problem with this situation was that while the students
with no experience required assistance, more experienced participants tended to
jump ahead of the group in doing exercises, following manual on their own and
not waiting for the demonstration of the evaluated functionality of the
prototype. At this stage, it is not possible to analyse whether this had any
influence on the results of the evaluation, as the demonstration which those
students would have followed, had the same information content as the user
manual on the prototype functionality. Therefore, it possible to assume here
that all students worked under the same conditions.

All students were able to complete exercises in allocated time.
 
\subsubsection{Exit Questionnaire Results}
The exit questionnaire results was the main data collection source for the user
feedback and evaluation.

In general, the responses were very positive. Comments made by the participants
during the exercises showed that they enjoyed the work with concept mapping
tools in the \ep~system.

Based on the responses to the Section B of the exit questionnaire, students
found \ep~concept mapping to be a valuable experience. According to their
comments, the tools and methods that were used during the experiment helped
them to think about bigger picture of their learning, make links between the
concepts they have learnt at the university, think how these concepts contribute
to the skills development, and organize their previous experience in a 
structured way that could be shared with others.

Summarized responses to the Section B of the exit questionnaire can be found in
Section \ref{sec:responses} of Appendix \ref{cha:app6}.

Figure \ref{fig:overall} shows overall scores for the used tools.

\begin{figure}[htb]
\centering
\includegraphics[width=1\textwidth]{CH7-F1-Chart}
\caption{Overall views on usefulness of the used tools and methods}
\label{fig:overall}
\end{figure}

The questions that can be asked here is whether participants' previous
experience with either \ep s systems or familiarity with \LLLs concepts
influenced the perception of usefulness of concept mapping tool as a part of the
\ep~system.

For this purpose, the results were split into three groups (Figure
\ref{fig:groups}) based on the differences in participants' experience
reported in the background section (Section A) of the exit questionnaire:

\begin{figure}[htb]
\centering
\includegraphics[width=0.5\textwidth]{CH7-F6-Groups}
\caption{Experiment groups based on student profiles}
\label{fig:groups}
\end{figure}

\FloatBarrier

\textit{Group A} represents participants who were familiar with the concepts of
\LLLs prior to the experiment. \textit{Group C} consists of students who have
not been using any kind of \ep~system to demonstrate their \LLLs prior to the
experiment. \textit{Group B} represents an intersection of groups A and C and
therefore consists of students familiar with \LLLsn, but who have not been using
\ep~systems. In this case, Group \textit{Group A} also represents students who
have used an \ep~system for \LLLs purposes prior to the experiment. Variable
\textit{n} represents a number of participants in each group respectively.

According to such distribution of the participants into groups, the view on
usefulness of the \ep~system concept mapping tool were following:

\begin{figure}[htb]
\centering
\includegraphics[width=1\textwidth]{CH7-F2-Chart-GA}
\caption{Group A views on usefulness of the used tools and methods}
\label{fig:group1}
\end{figure}

\begin{figure}[htb]
\centering
\includegraphics[width=1\textwidth]{CH7-F3-Chart-GB}
\caption{Group B views on usefulness of the used tools and methods}
\label{fig:group2}
\end{figure}

\begin{figure}[htb]
\centering
\includegraphics[width=1\textwidth]{CH7-F4-Chart-GC}
\caption{Group C views on usefulness of the used tools and methods}
\label{fig:group3}
\end{figure}

\FloatBarrier

As it can be seen, participants of all groups gave high scores as a measure of
usefulness of the concept mapping tools they had used. It can be also concluded
that the previous experience with the overall concepts did not have substantial
influence on the score distribution. However, it must be noticed that Group B
has a significantly larger number of students compared to Groups A and C.
Nevertheless, the conclusion drawn from this analysis can be that the novice
users find the tools that had been used during the experiment the same useful
as the students familiar with either one or both concepts of \LLLs and \ep s.

\subsubsection{Concept Maps Created by Students in the \ep~System}



\subsection{Conclusions}

Students who have had prior experience with using the \ep~system to demonstrate
their \LLLs skills said that concept mapping was a great idea and worked better
for them than the \ep~tools they had used before.

Although, usability was not in focus of the prototype requirements, a large
number of students noticed that concept mapping was easy and straightforward to
use. In contrast, extracting fragments of the artifacts and adding them as
examples to the concepts was more complicated and less intuitive task and
required detailed instructions. However, a common agreement among the
participants was that with proper training and experience all tasks can be
easily performed on daily basis.

A danger of testing software rather than evaluating the overall method and its
concepts. Unfortunately, it was impossible to avoid the responses that indicated
that participants were looking at the software and its functionality instead
trying to understand what stands behind it. This could be seen in such comments
like \textit{``Data format on the page is not right for New Zealand''},
\textit{``I would like to have a choice of color boxes''}, or \textit{``Bigger,
more approachable toolbar would be nice''}. However, there were no more than
four responses like these for each question.

Among recommended improvements were:

\begin{itemize}
  \item One
  \item Two
  \item Three
\end{itemize}
 
Overall, it can be concluded that the fundamental goal of this study to
evaluate whether the students find prototype tools helpful was accomplished. It
showed that students perception of the concept mapping as a part of the
\ep~system was positive and it provided a suitable means of addressing graduate
attributes and \LLLs skills.

\section{Study Three. System Validation by Experienced Students}
\label{sec:three}

This section

\subsection{Goals}

The goal for this study was to

\subsection{Research Protocol}

Appendix \ref{cha:app7}.

\subsection{Participants Profile}

Case studies are commonly used as evaluation method \citep{Yin2012}

 Strategy used to identify suitable cases. According to \citet{Stake1995},
 while selecting cases researchers should try to keep a balance between the
 uniqueness and the ordinariness.

Maximum variation cases \citep{Flyvbjerg2006}: cases were different in
one dimension: experience with using \ep~system.

\subsection{Activities and Artifacts}



\subsection{Data Collection and Analysis}

Face-to-face interview, system logs, cmaps, artifacts uploaded to the system.

\subsection{Conclusions}
 
\section{Summary}