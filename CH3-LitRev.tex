\chapter{Literature Review \label{cha:litrev}}

This chapter focuses on the concepts that form the background of lifelong
learning and its link to universities. By discussing systems currently used in
the university environment, such as Learning Management Systems (LMS) and
ePortfolio system, the current state of the art is established and the need for
further development is outlined. This chapter focuses on the concepts that form
the background of lifelong learning and its link to universities. By discussing
systems currently used in the university environment, such as Learning
Management Systems (LMS) and ePortfolio system, the current state of the art is
established and the need for further development is outlined.

This chapter focuses on the concepts that form the background of lifelong
learning and its link to universities. By discussing systems currently used in
the university environment, such as Learning Management Systems (LMS) and
ePortfolio system, the current state of the art is established and the need for
further development is outlined. This chapter focuses on the concepts that form
the background of lifelong learning and its link to universities. By discussing
systems currently used in the university environment, such as Learning
Management Systems (LMS) and ePortfolio system, the current state of the art is
established and the need for further development is outlined.

\section{Literature Review Process}
The literature review to support this project was conducted by systematically
reading and reviewing books, journals and conference proceedings in the area of
research. The main methods to identify relevant literature were recommendations
of a domain expert and a library search. Relevant articles were identified by
reading titles and abstracts of selected journals articles and papers in
conference proceedings. Where possible the latest ten years of issues of the
following journals were looked through: ``British Journal of Educational
Technology'', ``International Journal of Lifelong Education'', ``European
Journal of Education'', ``Lifelong Learning in Europe'', ``International Journal
of Emerging Technologies in Learning'', ``New Zealand Journal of Adult
Learning'', ``Journal of Computer Assisted Learning'', ``European Journal of
Engineering Education'', and ``International Journal of ePortfolio''. In
addition, a keyword search was carried out on the Internet and academic
resources (such as Education Research
Complete\footnote{\url{http://www.ebscohost.com/academic/education-research-complete}},
Academic Search
Premier\footnote{\url{http://www.ebscohost.com/academic/academic-search-premier}},
Directory of Open Access Journals\footnote{\url{http://www.doaj.org/}},
Google\footnote{\url{http://google.com}}, Google
Scholar\footnote{\url{http://scholar.google.com}}) to cover some conference
publications not available in the library. The following keywords and
combinations of keywords were used in the search: ``lifelong learning'',
``life-long learning'', ``e-learning'', ``ePortfolio'', ``e-portfolio'', and
``electronic portfolio''.

This review helped to discover previous work in the area, to explore methods
which can by applied to this research, to increase the depth and breadth of
knowledge of the field, and to identify domain experts and other people working
in the same field which could be valuable to contact. Besides finding relevant
information in the literature, it was also notable to identify the gaps that
currently exist. These gaps are based on facts that although a lot of work has
been done on developing lifelong learning theories as well as developing
technologies for education and learning, there are little substantial work done
on combining these two areas. Reviewing the literature is a continuous process.
Therefore, the literature review for this research was updated by actively
acquiring and reading the relevant articles emerging in the literature.

\section{The General Concept of \LLLc}
The concept of lifelong learning consists of a variety of meanings, models and
ideas (Jarvis, 2004).

The origin of the term 'lifelong learning' goes back to the early 20th century
and is contributed to by John Dewey (2004). From his perspective, lifelong
learning had to be centered on the individual's ability to take an active role
in democratic society. He saw education as a learning process which is
influenced by the growth of the individual and society, both interlinked.
Dewey’s key to lifelong learning was in developing active learning, enabling the
individual to reflect and change throughout life, emphasizing that non-formal
education was as important as formal education.

The concept of ‘lifelong education’ appeared in 1972 after Edgar Faure’s Report
“Learning to Be” for UNESCO. His concept was announced to be the leading one for
the reform in education. Faure’s Report used four principles for the lifelong
education architecture (Faure, et al., 1972): vertical integration (education
should occur throughout one’s life), horizontal integration (acceptance of
non-formal and formal education), the democratization of education (more
widespread involvement of learners) and learning society (restructuring of
educational system).

Now, 30 years after the idea of this lifelong education was introduced, many
governments rediscover not lifelong education, but lifelong learning (Boshier,
2000). This shift was not only semantic, but also substantive, which showed that
lifelong learning and lifelong education are not the same: lifelong education
aimed to develop more humane individuals and communities, while lifelong
learning's goal is in retaining and learning new skills that would help
individuals adapt to rapid changes in their workplace (Medel-Añonuevo, et al.,
2001). Lifelong learning is based on the notion of the individual learner as a
consumer. And as a result if consumers do not decide to take advantage of all
the opportunities they have – then it is only their fault. Therefore, being
constructed as individual activity learning depends entirely on personal
motivation. Unlike learning, education is a provided service (Boshier, 2000)
that requires someone to be responsible for providing resources, developing
policies, etc. The emphasis on ‘learning’ rather than ‘education’ is significant
(Tuijnman and Boström, 2002), as it moves focus from the institutions onto the
individual.

In terms of purposeful learning activities lifelong learning consists of the
following components (Longworth, 2003; Tuijnman and Boström, 2002):

\begin{itemize}
  \item Formal learning (institutionally graded, and hierarchically structured
system, often leads to qualification);
  \item Non-formal learning (organized systematic educational activity external
to formal education);
  \item Informal learning (planned or not planned, but conscious learning from
the experience);
  \item Incidental learning (not intentional, an accompaniment to everyday life,
learning during the action).
\end{itemize} 

Some researchers recognize two categories of lifelong learning, formal and
non-formal, leaving informal and incidental parts of it as the elements of
non-formal learning (Longworth, 2003).

Boshier (2000) states that at present the formal and non-formal categories are
like two parallel lines which seldom touch. Lifelong learning as well
encompasses the elements of self-direction, long-term and life-wide learning.
Therefore, it should also recognize the fact that learning also takes place
outside the formal education system and is guided by the learners themselves
(Schuetze and Casey, 2006). 

The European Commission defined Lifelong learning in its 2000 report (European
Commission, 2000) as:


\section{\LLLc in Universities}

\section{Requirements for Successful \LLLc}

\section{Summary} 