\begin{abstract}
\setlength{\parskip}{5pt}
Lifelong learning is seen as a self-directed pursuit of knowledge or skills that
occur throughout one's life. While this concept is not new, the importance of
\LLLs skills, in addition to academic and subject knowledge, has been
increasingly emphasised in the workplace and public policy over the last decade.
Higher education institutions, universities in particular, recognise the
importance of \LLLs and define their own strategies to promote it. Such
strategies include university development of graduate profiles which represent
the core learning outcomes, skills and qualities, that students should acquire
during their university education.

The problem identified and addressed in the current research is the lack of
comprehensive technical support solutions for \LLLs in universities. Currently,
only basic level systems are available, such as \ep~systems or accommodation of
Web 2.0 tools into university settings. However, the shortcomings of these
systems and tools, are hindering their full adoption, and as such the necessary
support for \LLLs is not available.

Through a literature review process followed by stakeholder interviews, this
thesis analyses the needs for supporting \LLLs in universities. According to
this analysis, better support is required for reflection, communication and
collaboration, development and showcasing of \LLLs skills, and tracking of
learning progress. These identified needs are then translated into requirements
that are used to create a prototype system that extends a current \ep~system,
Mahara, with new features, to provide institutional support for \LLLsn.

A number of studies, involving both lecturers and students, are conducted to
evaluate whether the prototype can be a part of the environment that provides
comprehensive support for \LLLs in universities. The results indicate that the
new features can be successfully adopted by students to help development and
understanding of \LLLs skills, address institutional graduate attributes, track
learning progress, as well as manage and share this knowledge with others. In
addition to these student-focused results, lecturers respond positively to
incorporating the prototype into their teaching. They see the opportunities for
employing the new features to provide students with the guidance through their
\LLLs journey at the university.

This research is the initial stage of providing comprehensive technical support
of \LLLs in universities. It draws attention to the influence that technology
has on teaching and learning, encourages cooperation between stakeholders, and
shows the importance of listening to the learner's voice.
\end{abstract}