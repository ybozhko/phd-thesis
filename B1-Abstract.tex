The concept of \LLLs is based on the principle of the self-directed pursuit of
knowledge or skills that occur throughout one's life. While the concept is not
new, the importance of \LLLs skills in addition to academic and subject
knowledge has been increasingly emphasised in the workplace and public policy
over the last decade. Higher education institutions, and universities in
particular, recognise the importance of \LLLs and define their own strategies to
promote it such as including learning attributes in their graduate profiles.
Yet, at this stage, \LLLs support provided in universities is not strong enough
to meet the needs of learners.

This research project explores theoretical concepts, technical solutions and
\LLLs support needs of universities. As it is shown in the literature review,
theories in this area have already been developed followed by raising awareness
and attempts at universities to support \LLLsn. Currently, basic level technical
solutions are available, such as \ep~systems or accommodation of Web 2.0 tools
into university settings, but their shortcomings are hindering full adoption.

In the course of this thesis, this research explored the needs and analyses
requirements for \LLL support in universities and suggests improving an
\ep~system as a part of the institutional support for learning. Based on the
indentified requirements, new features are implemented in the \ep~system
prototype to evaluate whether they can be a part of the environment which
provides comprehensive support for \LLLs in universities.
