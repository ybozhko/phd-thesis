The concept of lifelong learning is based on the principle of the self-directed
pursuit of knowledge or skills that occur throughout one’s life. While the
concept is not new, the importance of lifelong learning skills in addition to
academic and subject knowledge has been increasingly emphasised in the workplace
and public policy over the last decade. Higher education institutions, and
universities in particular, recognise the importance of lifelong learning and
define their own strategies to promote it such as including learning attributes
in their graduate profiles. Yet, at this stage, lifelong learning support
provided in universities is not strong enough to meet learners' needs. 

This research project explores theoretical concepts, available technical
solutions and lifelong learning support needs of universities. As it is shown in
the literature review, theories in this area have already been developed
followed by raising awareness and attempts at universities to support lifelong
learning. Currently basic level technical solutions are available, such as
ePortfolio systems or accommodation of Personal Learning Environments (PLE) into
university settings, but their shortcomings are hindering full adoption. 

This PhD research proposes a learner-centered e-learning environment which will
provide comprehensive support for lifelong learning. This environment will be
built on an institutionally focused Learning Management System (LMS) and a
learner focused ePortfolio system. While these systems already have some
low-level connections, extensions are required to adequately support lifelong
learning: students need to be in charge of their own learning progress; they
need to be able to choose the environment that serves their needs best and has a
smart data workflow to easily connect to their institution's environment; the
approach should be streamlined for both, teachers and students. 