Lifelong learning is a self-directed pursuit of knowledge or skills that
occur throughout one's life. While this concept is not new, the importance of
\LLLs skills in addition to academic and subject knowledge has been increasingly
emphasised in the workplace and public policy over the last decade. Higher
education institutions, universities in particular, recognise the importance
of \LLLs and define their own strategies to promote it. Such strategies include
universities developing graduate profiles which represent the core learning
outcomes, skills and qualities that students should acquire during their
university education.

The problem identified in the current research is the lack of comprehensive
technical support solutions for \LLLs in universities. Currently, only basic
level systems are available, such as \ep~systems or accommodation of
Web 2.0 tools into university settings. However, the shortcomings of these
systems and tools, are hindering their full adoption, and as such the necessary
support for \LLLs is not available.

This thesis extracts and analyses stakeholder requirements for supporting \LLLs
in universities. These requirements are then used to create a prototype system
that extends a current \ep~system, Mahara, with new features to provide
institutional support for \LLLsn.

A number of studies, involving both lecturers and students, were conducted to
evaluate whether the prototype can be a part of the environment that provides
comprehensive support for \LLLs in universities. The results indicate that the
new features can be successfully adopted by students to develop an understanding
of \LLLs skills, address institutional graduate attributes, track their learning
progress, manage their knowledge, and share it with others. In addition to these
results, lecturers responded positively to incorporating the prototype into
their teaching. This research is the initial stage of providing comprehensive
technical support of \LLLs in universities.



