\chapter{Review of Institutional and Open Learning Spaces \label{cha:systudy}}
%6200
In the previous chapter it was shown that many factors need to combine to fully
support lifelong learning at universities: changes in the way of thinking of
both students and lecturers, support at the department or institutional levels,
provision of training for staff, and personal motivation of learners. This
chapter looks at \LLLs support from another angle, such as technical support. It
reviews the area of technology and systems available for supporting various
aspects of learning, and examines how availability of a suitable e-learning
environment can possibly aid to the \LLLs support required by students in
universities.

The virtual learning spaces of universities are dominated by Learning Management
Systems (LMS) supporting course-related work. LMSs are often closed systems that
require user accounts and access permissions to the learning space. These
closed systems contrast to open learning spaces provided by Web 2.0, and social
networking tools in particular, which are characterised by open access allowing
individuals to participate under their own direction in contributing
information. Social networking includes sharing, exchanging and reflecting which
provides benefits for learning. 
 
To understand the barriers and issues of utilising closed and open learning
spaces within the university environment, this chapter first explores the
currently dominant LMSs and then contrasts them with the Web 2.0 virtual social
spaces. Finally, it introduces an \ep s and \ep~systems as a potential solution
that can help to close the gap that exists between the institutional and
personal learning environments. \ep~characteristics as well as strength and
weaknesses of selected \ep~systems are reviewed. This chapter concludes with an
analysis investigating whether currently \ep~as a system is mature enough to be
a part of the environment that provides comprehensive support for \LLLsn.

\section{Learning Management Systems}
Higher education institutions, universities in particular, have fully embraced
computer systems to support teaching and learning. According to a survey
conducted by the OECD Centre for Educational Research and Innovation
encompassing universities in 13 countries 89\% of responding universities were
using LMS institution-wide \citep{OECD2005}. The American Society for Training
and Development published the results of their 2009 \textit{Learning Circuits}
survey according to which 91\% of their respondents are using some kind of LMS
in their organization or institution \citep{Ellis2009}. Further indications of
uptake can be seen when visiting institutional websites, looking at user
statistics provided by system suppliers such as
Moodle\footnote{\url{http://moodle.org/sites}} or
Sakai\footnote{\url{http://sakaiproject.org/community-home}}, or by following
discussions in the academic literature \citep{Browne2006,Collis2004}.

According to Chapman \citeyearpar{Chapman2009}, there is no common definition of
a Learning Management System. A comprehensive description of LMS provided by
Watson \citeyearpar{Watson2007} states that it is a technology that can handle
all aspects of the learning process such as: delivering and managing learning
content; assessing learning of individuals and groups; tracking the progress
towards meeting learning goals; and collecting and presenting data for
controlling the learning process in institution or organization through virtual
classroom or instructor-led courses.

The systems are referred to as Virtual Learning Environments, Course Management
Systems or Learning Management Systems (LMS), the term used in this thesis. A
number of on-line information and communication tools are usually integrated in
such an environment into a single virtual location \citep{Morgan-Klein2007}
providing users with an access to teaching and learning materials, such as
lecture slides or exercises. A virtual space of LMS is shared by staff and
students of a particular course. This space forms a platform for course
discussions and facilitate assessment, both via on-line testing and for
submission and return of assignments.

The use of LMS in universities is characterised by a strong institutional focus
\citep{Siemens2004}. Access to the LMS depends on current enrolment with the
institution and is organised around course structures. This means students have
access to only the courses they are enrolled in or cohort based courses (e.g.
doctoral students community) and only for the duration of these courses. The
learning spaces for the different courses a student is enrolled in are separate.
LMS is based on a hierarchy of user access rights. The lecturer in charge
determines the tool-set for their course and sets the parameters that define the
involvement of the students. The lecturer has access to all information stored
for their course in the LMS, leaving no or only very limited private space for
the student. The content and use of the LMS is focused fully on the course
requirements. As a course-focused virtual learning space, LMS make a huge
contribution to the delivery of both face-to-face and distance courses in
today's universities.

\section{Web 2.0 and Social Virtual Spaces}
Outside the higher education sector, in the open Internet domain, the Web 2.0
social networking tools have been firmly established. Tools are available for
the sharing of images, photos and video clips. Individuals can communicate with
others in synchronous and asynchronous forms, and in access-protected as well as
open formats. Individuals can consume information on the widest possible range
of topics and can as well contribute. Web 2.0 is characterised by open access,
availability to anyone who has an Internet connection, and with the level and
kind of participation determined solely by the individual. With freedom comes
responsibility, and the responsibilities for taking up opportunities as well as
for \textit{safe} conduct in the Web 2.0 space lie with the individual.

\begin{figure}[htb]
\centering
\includegraphics[width=1\textwidth]{CH4-F0-WEB20}
\caption[Web 2.0 Landscape]{Web 2.0 Landscape \citep{Dawson2007}}
%http://www.rossdawsonblog.com/Web2_Framework.pdf
\label{fig:web20l}
\end{figure}

Web 2.0 plays an important role in today's society and is used for social and
commercial purposes. Examples from a variety of areas show the popularity and
impact of Web 2.0: virtual sports leagues attract millions of participants
\citep{Holahan2006}; politicians use blogs and podcasts in fighting for
voters \citep{Capell2006}; business model is changing trying to adopt Web 2.0
characteristics \citep{Wirtz2010}; communication with customers are used to
increase revenue \citep{Havenstein2007}; communication pathways in research
communities are changing \citep{Ashling2007}; Web 2.0 portals are used in health
care to increase access to and enrich the quality of the information available
\citep{Gorlitz2010,Metzger2011}; video-blogging facilitates new ways of sharing
\citep{LibraryTechnologyReports2007}; the music industry is being transformed
\citep{Holahan2007}; genealogy research has become accessible to the public
\citep{MacMillan2007}; hotel industry is adopting Web 2.0 technologies to
enhance customers' travel information and simplify access to the booking engines
\citep{Leung2011}.

Certainly, not all uses of Web 2.0 are linked to learning, especially when
thinking of the university context. But, in light of the \LLLs skills expected
from today's higher education graduates, the potential of Web 2.0 for supporting
learning becomes obvious \citep{Tian2011}. This potential is confirmed by
research studies that investigate the links between the two areas: Churchill
\citeyearpar{Churchill2009} examines the use of blogs in support of learning;
Wheeler, Yeomans and Wheeler \citeyearpar{Wheeler2008} look at student-generated
content using wikis; Boulos and Wheeler \citeyearpar{Boulos2007} investigate Web
2.0 tools for social communication in a learning context; Klamma and his team
\citeyearpar{Klamma2007} analyse a potential use of social software for
collaboration and informal learning. Yet, when designing education that
integrates Web 2.0 technologies the skill levels of students have to be
considered. While it is widely assumed that today's student generation is
Internet savvy, it has to be acknowledged that quite a number of students have
limited Web 2.0 skills. They are either not familiar with the technologies, or
have only basic level skills \citep{Kennedy2008}.

\section{Gap Between Learning Environments}
Students in universities have access to both environments, the institutionally
focused LMS and the individually focused Web 2.0. On large, these two virtual
worlds remain separate, both in the students' and the institutions' minds, with
a distinction being made between \textit{serious learning} and \textit{play}
\citep{Freire2008}. Many students cannot transfer their technology skills
employed in a social Web 2.0 context into academic learning, which is both a
motivational and a skill transfer issue \citep{Katz2005}. The information
technology sections of universities draw a clear line between institutionally
provided, controlled and supported LMS services and the \textit{wild west} of
the Web \citep{Havenstein2007a}. While they cannot effectively restrict access
to Web 2.0 tools, they can deny institutional support and responsibility for
quality of service. Educational researchers and individual academics have
identified the potential of social networking tools for teaching and learning.
This has led to the incorporation of open access Web 2.0 tools into some courses
in universities, as it has been illustrated earlier in this chapter.

In response to the popularity of Web 2.0 tools and their potential for learning,
LMS system providers have started to integrate social networking functionality
into their systems (as can be found in functional specifications of system
vendors). Discussion forums, blogs and wikis have been added to the tool-sets of
LMS. Yet, the important Web 2.0 characteristic of open access has been removed
as these tools have been bound into the institutional LMS framework. Access is
linked to course enrolment and under institutional control. Student-generated
content is accessible to the lecturers in charge and tool use is directed by
relevance to the respective course. The value for teaching and learning remains,
but learning is limited to the boundaries of course content and purpose
\citep{Mott2010}.

Facebook and Blackboard LMS can serve as an example that the gap between
these environments is wide and not easily bridged. An integrated application
using the Facebook social networking platform was included into the Blackboard
Learn software. Blackboard Inc. believed that such an approach would enable
students to stay connected, not only inside their classroom, but also outside
\citep{BlackboardInc.2009}. However, reviewing users' feedback on the Web (as
can be found by searching for the keywords \textit{Blackboard},
\textit{Facebook} and \textit{integration}) shows that this integration approach
was not accepted by the learner community. Users were concerned about
application security and the privacy of information stored in this social
networking environment. A number of students hesitated conducting their social
communication in such close proximity to their classroom work. 

Considerations outlined in this section bring up a need for a virtual space that
has to meet the requirements for successful \LLLs (based on Chapter
\ref{cha:litrev} recommendations) and facilite the development of \LLLs skills.
This space has to be integrated within universities and accepted by student
learners. It has to bridge institutional and personal learning. The virtual
space has to be safe, secure and provide students with a long-term access. It
should also facilitate both formal and informal learning and allow for social
networking and for collaboration. Such space needs to put students in charge of
their learning and offer them privacy for exploration, however still allow
for guidance from the lecturers. It should allow students to continue learning
informally even upon completing the formal courses (and losing access to the
LMS artifacts). This space has to provide a long-term accessible, safe
repository for storing artifacts demonstrating achievements. It needs to be a
\textit{professional} space that remains uncluttered from purely social
communication.

Taking into account all these requirements, the next section introduces \ep~as a
part of university's learning environment that has potential to provide support
for \LLLsn.

\section{ePortfolio}
For a long time physical portfolios have been used by artists as presentation
tools to collect, organize and showcase their artwork. The aim was to convince
potential customers of the artists' competence. Starting from two decades ago
portfolios were adopted by educators to assess the quality of teaching
\citep{VanTartwijkJ.2004}. Since then portfolios have been used for many
different purposes and as a consequence portfolio types such as showcase,
development and assessment have been defined.
 
Electronic portfolios or ePortfolios are a digital representation of physical
portfolios. The EDUCAUSE National Learning Infrastructure Initiative
(NLII)\footnote{\url{http://www.educause.edu}} (cited by
\citealp{IMSGlobalLearningConsortium2005}) defines ePortfolio as:

\longquote{ePortfolio is a collection of authentic and diverse evidence, drawn
from a larger archive, that represents what a person or organization has learned
over time, on which the person or organization has reflected, designed for
presentation to one or more audiences for a particular rhetorical purpose.}

\subsection{Characteristics of Portfolios and ePortfolio Systems}
The term portfolio is used in many different ways. As it was already
mentioned, an important distinction can be made along the lines of purpose of a
portfolio, namely for development, showcase, assessment or competences
\citep{VanTartwijkJ.2004}.

Development portfolios or repositories support the learning and development of a
learner over a period of time. They contain material and artifacts related to
learning, reflections and feedback. It is important that the material stored in
these repositories is private to the learner. It is up to the learner to decide
when and what to share with whom. The learner needs to reflect on the material
collected and on his/her development in relationship to criteria or skills. The
giving and receiving of feedback are important aspects of the learning processes
around development portfolios.

Showcase portfolios tend to display examples of learner's best work. These
presentations contain reflection and supporting evidence. They are composed for
a specific purpose and audience, e.g. a review committee, potential employer or
sponsor.

Portfolios are often linked to assessment. Type of portfolio and type of
assessment have to be carefully adjusted to each other. Assessment portfolios
demonstrate learner's competencies and skills in well-defined areas. They can be
used for both formative and summative assessment. For formative assessment the
learner documents work and reflects on it, the assessor provides feedback that
assists the learner in future development. Summative assessment requires
predefined criteria of what is to be assessed allowing the learner to organize
work examples according to these criteria. In the design of the assessment
approach one has to be very careful to specify clearly what is to be assessed:
subject specific work, reflections, \LLLs skills, or presentation.

Portfolios for competences combine elements of both development and showcase
portfolios and are, to a certain degree, linked to assessment. In professional
areas, like health services, teacher education or engineering, the accreditation
of graduates and the continuing accreditation of professionals are often linked
to the demonstration of competencies. Portfolios have proven to be excellent
tools for this process. The candidate collects evidence, reflects on their
practice and might invite feedback, all processes covered by portfolio
approaches. The accreditation occurs based on the information provided in the
portfolio.

From organizational perspective, Lorenzo and Ittelson \citeyearpar{Lorenzo2005}
distinguish three major types of portfolios: student, teaching and
institutional portfolio.

\begin{figure}[htb]
\centering
\includegraphics[width=0.8\textwidth]{CH4-F1-EP}
\caption[ePortfolio key processes]{ePortfolio key processes \citep{Malloff2010}}
%http://www.flickr.com/photos/kootenayleadership
\label{fig:ep}
\end{figure}

Despite these variations, there are several key processes included in most if
not all portfolio work, as is displayed in Figure \ref{fig:ep}.
 
Similarly, Cambridge \citeyearpar{Cambridge2010} emphasized the importance of
the following activities in portfolio process:

\begin{itemize}
  \item Capture -- collecting/gathering information and evidence from various
  sources;
  \item Management -- aggregating captured evidence, sorting, indexing, ensuring
  accessibility over time;
  \item Reflection -- making sense of evidence, understanding own experience and
  achievements;
  \item Composition -- linking up the components together and making them
  available to others;
  \item Analysis -- understanding if additional evidence is needed, reflecting
  on feedback, keeping up dialog with others.
\end{itemize}

While portfolio work can be conducted without the help of electronic systems,
such systems assist with many tasks around document collection, recording of
information, sorting through data and communicating with others. According
to Tosun and Baris's \citeyearpar{Tosun2011} research, \ep~compared to portfolio
has valuable extra features such as: a wider context and serving different
groups; archiving; cooperation and reorganization; publication and link building.
Many systems, from general Web tools to specialised applications, can be used to
support portfolio work. A comprehensive overview can be found at Helen Barrett's
ePortfolio web-site \citep{Barrett2008}. This section focuses on systems
specialised for portfolio work.

\ep~systems are centered around the individual and their needs. They provide the
individual with a space for storing documents of any electronic format. In this
space the user creates a repository of artifacts related to all aspects of their
learning and professional development. There are tools for reflection, commonly
in form of blogs. In contrast to open Web 2.0 systems, access to both files and
reflections is by default set to the individual. There is no hierarchy between
users in which one higher-level user could see the work of a lower-level user.
The individual can select to share their work with others and has full control
over whom to share with, for which period of time. ePortfolio systems provide
easy to use tools for constructing presentations that combine artifacts and
reflections and that can voluntarily be shared with others. The systems allow
each individual to form groups and identify partners for exchange. To a varying
degree the ePortfolio systems incorporate guidance towards reflection and
self-directed learning. \ep~systems provide a set of features that in
combination are well suited to support \LLLsn. Each of the features looked at
separately can be found in other computer systems or Web 2.0, but their
combination within one system makes ePortfolio systems so valuable.

\subsection{ePortfolio Systems Overview}
The following sections explore the features and functionality of various
\ep~systems. These specific systems were chosen for their level of success in
learning communities and current development status. Four proprietary
(PebblePad, BlackBoard \ep, Desire2Learn, eFolio) and two open-source (Mahara,
ELGG) systems are reviewed and analysed. Where possible, proprietary systems
were reviewed by accessing demonstration web sites. In case demonstration web
sites were not available, the systems were reviewed by analysing user or
administrator documentation, video demonstrations, attending demonstration
seminars, and external reviews.

It is important to note here that this section is not aiming to find the best
system, but rather to evaluate \ep~systems that are currently available and
successful. Examining strengths and weaknesses of these systems can provide a
better foundation for understanding and development of an \ep~aided environment
that could support \LLLsn.

\subsubsection{PebblePad}

PebblePad\footnote{\url{http://www.pebblepad.co.uk}} is a proprietary web-based
\ep~system. However, its designers tend to call it not just an \ep~system, but
Personal Learning System that can be used in a variety of learning contexts
\citep{PebbleLearningLtd2010}. The system is popular and primarily used in the
UK Higher Education sector and has been involved in a number of JISC funded
ePortfolio research projects including
ePistle\footnote{\url{http://www.jisc.ac.uk/whatwedo/programmes/edistributed/epistle}}
and File-Pass\footnote{\url{http://www.jisc.ac.uk/whatwedo/programmes/edistributed/filepass}}.

According to the PebblePad technical specification, the back-end of the system
requires Windows and SQL Servers to run. The front-end uses Flash, which can
create a challenge for the web application's accessibility, usability and
performance. To function, Flash-based applications require plugin which in turns
might cause many standard browser features not perform as the user would expect
or may even cause a browser crash. Flash applications do not work on many mobile
and portable devices. In addition, because Flash applications are compiled into
binary files, screen readers used on web sites with support of sight impaired
cannot read them resulting in poor accessibility.

PebblePad has a customizable user interface which includes user-defined size and
style of text and background colours. Institutions can have their own interface
that fits in with the institutional branding.

Items stored in PebblePad repository are called assets. There are thirteen asset
types that are subdivided into three core types, such as: uploaded files, single
assets and aggregating assets. Creation of some assets can be guided by
step-by-step wizard. Assets can be shared with others, inside or outside of an
institution, for a certain period of time through user-defined permissions. If a
person, who needs to see an asset, is not a part of PebblePad community, a
temporary username and password will be automatically created allowing them to
view a shared asset. Assets allow for setting up a wide range of permissions,
which can include commenting and copying rights, collaboration and re-sharing
of a shared asset with a third party.

\begin{figure}[htb]
\centering
\includegraphics[width=0.9\textwidth]{CH4-F3-PebblePad}
\caption[PebblePad ePortfolio system example]{PebblePad ePortfolio system example
\citep{PebbleLearningLtd}}
\label{fig:ppep}
\end{figure}

However, despite these features, working with assets in PebblePad has its
disadvantages. The way the system's repository is structured, asset tracking and
finding in PebblePad is noted to be not user-friendly \citep{Overton2009}. Due
to poor asset management, users can end up deleting files and breaking links
between assets, or forgetting to update the hyperlinks of the changed assets,
which results in missing files for someone who is viewing the asset. Users
cannot upload files larger than 10MB.

PebblePad has an interface with the Moodle LMS that allows \ep~users to have
single sing-on with LMS and also export items from Moodle to their ePortfolio.
The system supports Leap2a and IMS eP as well as import from any RSS or Atom
compliant system. According to vendors website, as for the middle of 2011,
the price of PebblePad adoption ranged from 14,95GBP for individual accounts
hosted by the company to 1GBP per user for the largest customers hosting
the system themselves. After graduation from the sponsoring institution,
students can get a free 12-month personal account managed by Pebble Learning.
 
\subsubsection{Mahara}
Mahara\footnote{\url{http://mahara.org}} is an open source \ep~system started in
2006 funded by New Zealand Tertiary Education Commission. The system is a
standalone web application and does not require any kind of LMS or another
system installed. Its modular and extensible architecture that resembles the
architecture of Moodle LMS. This can be explained by the fact that developer
community of Mahara is deeply involved in the Moodle community. The system is
claimed to be highly 'pluggable' which allows adding various Web 2.0 web
services and establish interoperability with other systems
\citep{MaharaGovernanceGroup2011}.

Mahara functionality includes a number of standard \ep~features like file
repository, reflection tools in form of blogs, presentation and sharing tools as
well as elements of social networking like friends lists, forums, message board
and e-mail. Mahara has internal r\'{e}sum\'{e} builder which allows users to
create their digital CV with various information options. Sharing is done
through pages which are called 'views'. Users can create single views or
collections of views and fill them with artifacts from their \ep~repository.
Views can be created from scratch as well as from a template developed by
another user.

\begin{figure}[htb]
\centering
\setlength\fboxsep{0pt}
\setlength\fboxrule{0.5pt}
\fbox{\includegraphics[width=0.9\textwidth]{CH4-F4-Mahara}}
\caption{Mahara ePortfolio 1.3 example}
\label{fig:maharaep}
\end{figure}

A group portfolio is available for collaboration purposes. Compared to personal
accounts, groups have functionality limited to creating and maintaining pages
(views), forums and file repository.

As it became popular in ePortfolio systems throughout the recent years
\citep{Waters2009}, Mahara comes with a user-to-user permissions control. Users
can set up three levels of access to parts of their ePortfolios (private,
individual and public) which defines what items and information others can see.
Currently the Mahara system does allow sharing views with others or making them
public, but giving feedback is restricted to registered users.

Mahara supports a complete LEAP2A interoperability which allows to import
portfolio content to Mahara and export to another \ep~system, provided that
this interoperability standard is implemented at the other side. In addition,
export in form of static HTML pages is supported.

The latest version of Mahara supports single sign-on with Moodle, which means
that users can log on to both systems using only one account. Unofficial plugins
developed by the community allow for submitting views as assignments to Moodle.
However, this functionality is not included in official release. The road-map of
Moodle 2.0 included a repository plugin for Mahara that would allow direct
export of artifacts from LMS to \ep. Meanwhile, Moodle 2.1.1 release still does
not support this functionality.

\subsubsection{ELGG}
ELGG\footnote{\url{http://elgg.org}} is an open source social networking and
social publishing platform started in 2004 and released under the GNU Public
License v2. It was originally aimed at higher education, but is currently used
in many contexts from business to sport. Developers of ELGG call it a
\textit{social engine to empower social environment}.

Most the end-user functionality comes from plugins which can be loaded into
system. This review examines a standard installation. ELGG is supported by an
extensive community which has contributed a large number of plugins. In
general, most of these plugins are aimed at supporting social networking.

Features available in the standard platform installation include user management
and administration, social networking components (like friends list and ``the
wire''), blogging, message board, file repository, private messaging, pages, and
bookmarks. Additional components can be installed by administrator as plugins
and can be used within the entire system.

In ELGG, each account has a profile page which links to all available artifacts
created by the user through adding or removing widgets. Except for the profile
page, there is no standard way of aggregating artifacts for presentation.
The profile page is as well the main option for showcasing as users cannot have
multiple ePortfolios.

Unlike other ePortfolio systems, ELGG has a quite limited choice of permissions.
Artifacts in the system can be either private/public, or shared with friends
or logged-in users. There is no way of having multiple permission settings or
user-to-user permissions for artifacts.

Similar to Mahara ePortfolio system, ELGG has groups for collaboration. Groups
have the same system components (e.g. blogs, pages, files) as single profiles,
and these options can be set up or removed at any time.

ELGG has no reporting system for users which would show a number of page
visits, file downloas, etc. However, minor reporting functionality is available
for administrative purposes.

In 2007, interoperability between ELGG and the open source LMS Moodle was
established for single sign-on and courses integration. However, since Moodle
1.9 there is no news on plugin updates. Information has been found on the
Internet about a proprietary plugin being developed for ELGG-Moodle integration,
although no up-to-date documentation is currently available that would describe
this plugin.

\begin{figure}[htb]
\centering
\setlength\fboxsep{0pt}
\setlength\fboxrule{0.5pt}
\fbox{\includegraphics[width=0.9\textwidth]{CH4-F5-ELGG}}
\caption{ELGG 1.8 example}
\label{fig:elgg}
\end{figure}

\subsubsection{BlackBoard ePortfolio}
After BlackBoard\footnote{\url{http://www.blackboard.com/}} took over WebCT in
February 2006, this popular LMS provider developed an \ep~toolkit the most
recent release of which is currently a part of BlackBoard Learn 9.1. This
\ep~system is designed as an add-on to the LMS environment and can not be used
as a stand-alone product. On one hand, it means that all users must have
BlackBoard LMS account to be able to access \ep. On the other hand, it gives
some advantages which other \ep~systems might lack, such as single sign-on with
LMS, direct import of graded materials from Blackboard courses and links to
course goals and objectives.

BlackBoard \ep~is available in Basic and Personal Portfolio versions. Basic
Portfolio has an \ep~set-up wizard for learners who need guidance. However, it
is largely dependent on functionality available in LMS. Without activation of
various features, the repository might be restricted to text and hyperlinks
only. Personal Portfolio provides more flexibility and functionality. Therefore,
this version will be reviewed further as BlackBoard \ep.

\begin{figure}[htb]
\centering 
\setlength\fboxsep{0pt}
\setlength\fboxrule{0.5pt}
\fbox{\includegraphics[width=0.9\textwidth]{CH4-F6-BB}}
\caption[BlackBoard ePortfolio example]{BlackBoard ePortfolio example
\citep{UniversityofTorontoScarborough2010}}
\label{fig:bbep}
\end{figure}

In the system, ePortfolio owners have control over the content, access, layout
and style of their portfolio. ePortfolios can be created from available
templates predefined by an administrator or a lecturer, or they can be created
from scratch. A variety of video, audio and text file types is supported as well
as an HTML editor for creating pages. Reflections are facilitated in form of
blogs or threaded topics. Content is separated from portfolios which allows
reuse of the artifacts. It has been reported that because of this separation
artifact management is not intuitive and might be too complex for students for
effective use of tools \citep{Clark2009}. In addition, portfolios can be linked
to learning objectives defined by lecturers, administrators or learners
themselves.

When necessary, BlackBoard \ep~can be shared with people inside the
institutional community through system username, groups and courses as well as
outside -- via email or creating a guest account which is by default active for
30 days. However, availability of these sharing options is set up by system
administrator who can allow or restrict any of these options. Depending on
access level, users can leave their feedback in form of comments. Comments
cannot be attached to individual artifacts and are stored within single pages of
\ep. BlackBoard \ep~system has a basic reporting system where users can enable
tracking, and gather basic data about views of their portfolios. At the
completion of studies \ep~can be downloaded and saved as HTML in a ZIP archive.

Overall, BlackBoard \ep~is good for creating portfolio of student course or
program work and for linking to a course of study
\citep{UniversityofTorontoScarborough2010}. 

According to \citet{Sweat-Guy2007}, the cost of 12 months license for
BlackBoard \ep~in 2006-2007 was 20,900USD which did not include the cost of
prior purchase and adoption of LMS. To date, no information was found on current
development status and future releases.

\subsubsection{Desire2Learn}
Desire2Learn\footnote{\url{http://www.desire2learn.com}} \ep~is a proprietary
\ep~system developed by Desire2Learn Incorporated. It can be deployed as a
standalone application or as a part of a Desire2Learn Learning Environment. As a
result of close working relationship of the developing company with Microsoft,
this \ep~system, as well as all Desire2Learn software, is built on Microsoft
technologies, such as SQL Server and Windows Server \citep{AAEEBL2011a}.

\begin{figure}[htb]
\centering
\setlength\fboxsep{0pt}
\setlength\fboxrule{0.5pt}
\fbox{\includegraphics[width=0.9\textwidth]{CH4-F7-D2L}}
\caption[Desire2Learn ePortfolio example]{Desire2Learn ePortfolio example
\citep{Desire2LearnIncorporated2011}}
\label{fig:d2ep} 
\end{figure}

Most artifacts can be uploaded to the system from external resources. Some of
them, such as HTML files, can be created within the environment. This also
includes creating audio recordings which is a unique functionality compared to
other \ep~systems. Currently, Desire2Learn developers are looking into adding
support for creating of video records.

There is a standard for ePortfolio systems range of functionality associated
with artifacts: they can be grouped, shared with others, commented on, assessed
directly or submitted as an assignment. Assessment results, such as grades,
competencies or quiz details, can be saved as \ep~artifacts as well.

In addition to individual artifacts, other types of items in the \ep~system are
collections, presentations, reflections and forms. Collections are used to group
artifacts and can be created manually or automatically based on a defined tag
set. Forms provide a way of developing artifacts with standard field types. This
can be used for creating templates for evaluations, resumes, or self-evaluation.
Presentations are personal web sites that present a set of artifacts in an
organized way allowing users to choose theme, set up layout and manage content.
Reflections are a separate form of the artifacts. They can be associated with
artifacts or presentations and can be a part of collection or presentation.

Feedback can be applied to individual artifacts, collections, reflections or
entire presentations. If needed, evaluators can review all comments made by
peers. Users can add assessment rubrics to artifacts that require specific
type of evaluation. More comprehensive assessment features are available via
integration with other Desire2Learn LMS tools.

Desire2Learn \ep~has reporting capabilities for administrators and teachers which
support tracking usage and accessing detailed information on competency
achievement by students. Minor reporting is available to users in form of
presentations access logs.

Any part or an entire \ep~contents can be imported or exported using
either XML, or HTML format. XML is a native format of the system and allows to
import \ep~to another Desire2Learn \ep~system instance.

No estimate cost of the Desire2Learn \ep~was discovered as vendors do not
disclose pricing information, explaining that each case is unique to each
institution.
 
\subsubsection{eFolio}

The eFolio\footnote{\url{http://www.avenetefolio.com}} system is a software
service hosted and maintained by Avenet Web Solutions. Developed in 2001
together with the University of Minnesota, eFolio currently has a large user
base, the biggest of which are
eFolioMinnesota\footnote{\url{http://www.efoliominnesota.com}} (over 60,000
active users) and eFolioWorld\footnote{\url{http://www.efolioworld.com}} (over
34,000 active users) \citep{AAEEBL2011}. Being said that eFolio is a hosted
service, it is still possible to get self-hosted solutions for very large
implementations.

\begin{figure}[htb]
\centering
\setlength\fboxsep{0pt}
\setlength\fboxrule{0.5pt}
\includegraphics[width=0.9\textwidth]{CH4-F8-eFolio}
\caption[eFolio system example]{eFolio system example \citep{EFolioMinnesota2011}}
\label{fig:efolio}
\end{figure}

Every account in eFolio can have multiple portfolios which are organized as
sites. When users start creating a site, they can use a wizard under the ``To
Do" category for filling the pages with the relevant content. The latest version
of the system has a drag and drop site management interface which makes it easy to
create sites and pages. eFolio comes with a number of build-in display and style
formats. However, according to the comments \citep{AAEEBL2011}, the system
does not provide as much page layout flexibility as some users would expect.

Everything saved in eFolio is located in the ``My Content'' category. Currently,
its standard storage capacity is 50MB per user, but this limit can be
negotiated. My Content groups items by data types such as artifacts, courses
taken, goals, images, URLs, employment, etc. As well, users can add (create) the
items on a fly while building their sites. Users can set up content properties
and formatting, add other content related to the item and write multiple
reflections.

All sites are by default private, but can be set to public once finished.
Underage user sites are always private by default and cannot be made public.
Instead, owners can invite a specific person to review their site. Access to the
web site is granted through creating visitors. Each visitor gets an email with
login details if the site is private or site URL if it is public. Even users who
already have accounts in the system get a visitor account for each specific site
access. Visitors can leave feedback to any item with feedback properties set up.
Depending on these properties, feedback can be in the form of Likert scale,
question/answer or free-form text.

eFolio does not provide collaboration functionality and all web sites in the
system are personal. For assessment purposes eFolio has questionnaires of
various types which students can use to address assessment criteria.

eFolio supports the IMS ePortfolio standard which allows users to import/export
their ePortfolio content. The system can be integrated with the Moodle LMS.
However, no detailed information was found on what kind of integration this is.

After graduating or leaving a sponsoring institution, users can continue
using their eFolio for an annual fee of 15USD. Prices for institutions depend
on numbers of users, and can range between 15USD - 4USD per user
\citep{AAEEBL2011}.

\subsubsection{Discussion and Summary}
%TODO: Write Summary
Summary for eP systems, what they have in common, what is unique, etc.

Table with mapping of recommendations to the features.

\subsection{\ep~Systems in Light of Lifelong Learning}

Considering that the expectation around \ep~systems is that these systems
support \LLLsn, the question is whether they are doing it effectively in light
of the recommendations discovered in the literature. Due to the fact that the
literature provides only highly conceptual recommendations, it is difficult to
translate these into formal requirements. Therefore, assessment whether these
recommendations are met by the features implemented in the systems turns into a
challenging task. 

The previous section described an attempt to address this proble trying to map
the recommendations aganst the features of the reviewed \ep~systems. However,
due to the formal ambiguity of these recommendations, the results of this
analysis should not be considered complete. The researcher had to follow common
sense and personal experience to perform this analytical mapping. To solve this
problem, bringing the recommendations to the practical level is the next
important step towards a better understanding of what is expected from the
environment that can support \LLLsn. One can argue that the majority of these
recommendations cannot be addressed by just providing an improved system
\citep{Schaffert2008}. However, from another perspective, a better system might
aid to supporting various important aspects of learning that usually stay
outside of focus of many learning environments.

Although, no prior research was discovered that would look at the \ep~systems
from the \LLLs perspective, there is a number of issues known among the
\ep~community that might be relevant to \LLLs support. For example, current
\ep~systems have difficulty helping students to link abstract knowledge to
practical experience which is an important part of understanding one's personal
progress and achievements \citep{Chou2009}. Interoperability between different
\ep~systems as well as other learning systems is quite poor despite of the
existing standards \citep{Clark2011}. Assuming that \ep s are lifelong, they are
supposed to cope with large amounts of data \citep{Butler2010}. However,
practice shows that current systems can barely offer efficient methods for
managing data repository to users who have been using them extensively for just
a couple of years. There are also issues of ethics, privacy and intellectual
property where \ep~users need to decide who owns the data and how it can be used
\citep{Challis2005}.

The problems mentioned here are just some examples that are not likely to draw a
complete picture. To get a deeper insight into \ep~issues and understand what
improvements are required for \ep~systems to fulfil the promise of efficient
\LLLs support, a deeper analysis of the area is required.

\section{Summary}

Based on the deliberations outlined in this chapter, additional technical
requirements can be considered along with the recommendations for successful
\LLLs support, such as:

\begin{itemize}
	\item A good virtual learning environment should facilitate the development of
	\LLLs skills;
	\item It should fit with university needs;
	\item It has to be accepted by student learners;
	\item It should create a bridge between institutional and personal learning.
\end{itemize}

\ep~system seems to fit well into this picture. It brings a balance into the
world of learning environments, and has potential of closing the gap the exists
between LMS and Web 2.0. Reviewing \ep~systems showed that the systems currently
available world-wide offer a range of opportunities for \LLLsn. Each system
comes with commonly valuable functionality that promises support for important
aspects of learning. However, are they mature enough to be a part of the
environment that provides comprehensive support for \LLLsn? Previous section
discussed that current \ep~systems might still lack some elements important for
\LLLsn. To support this hypothesis, the next chapter will explore the needs for
\LLLs supported by \ep~system based on the major stakeholders perspective.
University students and lecturers have been interviewed to get their insight on
the requirements and to understand whether these comply with the literature
review findings.
