\chapter{Research Framework\label{cha:method}}

\section{Objectives}

To achieve the research aim a number of objectives are proposed:
\begin{itemize}
  \item To determine student and institutional requirements for a lifelong learning
environment within the university context;
  \item To map these requirements against the e-tools, and ePortfolio system in
particular, already used in universities to support lifelong learning;
  \item To implement the features required in ePortfolio systems in conjunction
with LMS to satisfy the defined requirements;
  \item To evaluate how these combined systems meet the needs of all
stakeholders in supporting lifelong learning.
\end{itemize}

\section{Research Questions}

\begin{enumerate}
  \item \textit{What is the concept of \LLLs and its connection to the
  universities?}
	\begin{itemize}
	  \item What is the role of \LLLs in the university context?
	  \item What is the motivation of universities in supporting \LLLsn?
  	  \item What are the existing university policies for supporting \LLLsn?
      \item What are the components of \LLLs environments in universities?
      \item What are the requirements for successful \LLLs support in
   universities?
	\end{itemize} 
	
   \item \textit{What e-tools are available to support \LLLs within the
   university context?}
	\begin{itemize}
		\item What e-tools are available to support \LLLsn:
			\begin{itemize}
				\item in general?
				\item in universities?
			\end{itemize}
		\item What are the conceptual strengths and weaknesses of these e-tools in
university context?
		\item What is the relationship between LMS and e-tools support for \LLLs in
university context?
	\end{itemize}

	\item \textit{How can LMS and/or ePortfolio systems be extended to support
	students in a university context in \LLLsn?}
	\begin{itemize}
		\item What features are available now in these systems?
		\item What are the students and institutional requirements for LMS and
		ePortfolio to support \LLLsn?
		\item How can these requirements be translated and implemented into new or
		improved features?
	\end{itemize}

	\item \textit{Do this extended environment meet the needs of the stakeholders
in university teaching and learning contexts?}
	\begin{itemize}
		\item How can lecturers use new features to provide students with their
guidance and help them to understand \LLLs skills?
		\item How can students address institutional graduate attributes and other
		skills using new features?
		\item How can new features help students track their learning progress, manage
ePortfolio knowledge and content, demonstrate and share their achievements
with others?
	\end{itemize}
\end{enumerate}

\section{Research Approach}

\subsection{Design Science Research Methodology}

\subsection{Design Science Research Applied to This Project}

\subsubsection{Stage 1. Problem identification and motivation}

\subsubsection{Stage 2. Objectives for a Solution}

\subsubsection{Stage 3. Design and Development}

The prototype development in this project followed established
software engineering practices that interleaved coding and revision, forming
iterative development cycles, as it shown at \ref{fig:prototype}.

\begin{figure}[htb]
\centering
\includegraphics[height=0.3\textheight]{CH2-F2-Prototype}
\caption[Ptototyping]{Prototyping (based on \citet*[p.~411]{Sommerville2007})}
\label{fig:prototype}
\end{figure}

\subsubsection{Stage 4. Demonstration}

\subsubsection{Stage 5. Evaluation}

\subsubsection{Stage 6. Communication}

\section{Methodological Limitations}

\section{Related Work}
\label{sec:related}
As the field of lifelong learning became popular, there were a number of studies
aimed to explore lifelong learning support in various contexts. To date research
similar to this project has not been identified, although projects found were a
valuable source of information and examples of previous research experience.

\begin{itemize}

  \item Lifelong Learning in London for
  All\footnote{\url{http://www.lkl.ac.uk/research/l4all.html}} (L4All): This
  project is focused on developing of lifelong learning system to support
  independent learners (particularly those 16+ learners who traditionally have
  not participated in higher education) by recording their learning pathways.
  This project aimed to provide lifelong learners in the London region with
  access to information and resources that facilitates their progression from
  secondary education to further education or from secondary education directly
  to higher education \citep{Freitas2006};

  \item The Regional Interoperability Project on Progression for Lifelong
Learning\footnote{\url{http://www.jisc.ac.uk/whatwedo/programmes/edistributed/rippll.aspx
}} (RIPPLL): This project was going to establish a model of cross-sector
collaboration in personal development planning technology in the UK. The aim was
to make interoperable all the major existing electronic systems for study- based
progress files in use in further and higher education to provide an easier
transition process from school to further education \citep{Hartnell-Young2006};

  \item ELGG-Moodle: In autumn 2006 Klagenfurt University, Austria was piloting
the project aimed to integrated Moodle LMS and ELGG platform. This integration
was used for professional development for all academic staff. Project outcomes
provided integration between systems such as single login and file transfer
\citep{Attwell2007}.

\end{itemize}

\section{Summary}