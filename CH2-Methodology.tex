\chapter{Research Framework\label{cha:method}}
%823words
The main purpose of this chapter is to describe a research approach used in this
study. The first section of the chapter identifies the objectives that need to
be addressed in order to achieve the goal of this study, followed by the research
questions in Section \ref{sec:questions} raised from these objectives.

Design Science Research (DSR) methodology was adopted as the main research
methodology to address the research questions. This methodology emphasizes the
problem-solving and performance-improving paradigms and is oriented towards
creating and evaluating IT artifacts \citep{Hevner2004}. A five-stage research
project framework is outlined in Section \ref{sec:method} which explains each
stage of the project and the methods applied. Methodological limitations are
brought to considerations in Section \ref{sec:limits}. This chapter concludes
with the discussion of related work and projects carried out in the area of
\LLLsn.

\section{Research Objectives}

Understanding what kind of technical solution is required to effectively support
\LLLs in universities is an overarching goal of the research. This goal brings
up a number of objectives that need to be addressed:
\begin{description}
  \item[Objective 1.] To determine student and institutional requirements for a
  \LLLs environment within the university context;
  \item[Objective 2.] To map these requirements against the systems already used
  in universities to support \LLLsn;
  \item[Objective 3.] To design and implement the features required in an
  environment that supports \LLLs to satisfy the defined requirements;
  \item[Objective 4.] To evaluate how this environment meets the needs of all
  stakeholders in supporting \LLLsn.
\end{description}

\section{Research Questions}
\label{sec:questions}
Based on the context of the research objectives, this study addressed the
following research questions supported by sub-questions:

\begin{description}
  \item[RQ1:] \textit{What is the concept of \LLLs and its connection to the
  universities?}
	\begin{itemize}
	  \item What is the role of \LLLs in the university context?
	  \item What is the motivation of universities in supporting \LLLsn?
  	  \item What are the existing university policies for supporting \LLLsn?
      \item What are the components of \LLLs environments in universities?
      \item What are the requirements for successful \LLLs support in
   universities?
	\end{itemize} 
	
   \item[RQ2:] \textit{What e-tools are available to support \LLLs within the
   university context?}
	\begin{itemize}
		\item What e-tools are available to support \LLLsn:
			\begin{itemize}
				\item in general?
				\item in universities?
			\end{itemize}
		\item What are the conceptual strengths and weaknesses of these e-tools in
university context?
		\item What is the relationship between LMS and e-tools support for \LLLs in
university context?
	\end{itemize}

	\item[RQ3:] \textit{How can LMS and/or ePortfolio systems be extended to
	support students in a university context in \LLLsn?}
	\begin{itemize}
		\item What features are available now in these systems?
		\item What are the students and institutional requirements for LMS and
		ePortfolio to support \LLLsn?
		\item How can these requirements be translated and implemented into new or
		improved features?
	\end{itemize}

	\item[RQ4:] \textit{Do this extended environment meet the needs of the
	stakeholders in university teaching and learning contexts?}
	\begin{itemize}
		\item How can lecturers use new features to provide students with their
guidance and help them to understand \LLLs skills?
		\item How can students address institutional graduate attributes and other
		skills using new features?
		\item How can new features help students track their learning progress, manage
ePortfolio knowledge and content, demonstrate and share their achievements
with others?
	\end{itemize}
\end{description}

\section{Research Approach}
\label{sec:method}

Finding the most efficient research approach is an important part of any
research study. A properly selected approach helps to obtain answers to the
research questions while working within the framework that uses methods that
have been verified and tested for validity \citep{Kumar2005}.

Multi-paradigmatic field of ICT offers a number of methodologies drawn from the
variety of research philisophies \citep{Vaishnavi2007}.

This section looks 

This section looks 
 
This section looks

This section looks  

\subsection{Design Science Research Methodology}

According to \citet{Peffers2008}, design science research (DSR) originates from
engineering and computer science where design is a component of the research
process. \citet{Iivari2009} define DSR as \inlinequote{a research activity that
invents or builds new, innovative artifacts for solving problems or achieving
improvements}. This approach is fundamentally a problem-solving and is used in
ICT where there is a need to extend the existing boundaries of current systems
or address the important problems by creating new and innovative solutions and
artifacts \citep{Hevner2004}. The artifacts can be described as constructs
(vocabulary of a domain), methods (algorithms), models (abstractions),
instantiations (prototype systems), and better theories
\citep{Hevner2010,Vaishnavi2007}.

Hevner \citeyearpar{Hevner2004} identifies seven guidelines for effective DSR:

\begin{table}[htb]
  \begin{center}
    \begin{tabular}{| l | p{6.5cm} |}
    \hline
     \multicolumn{1}{|c|}{\textbf{Guidelines}} &
     \multicolumn{1}{c|}{\textbf{Description}} \\
     \hline
     Guideline 1: Design as an Artifact & Research must produce a viable
    artifact such as a construct, a model, a method or an instantiation \\ \hline
     Guideline 2: Problem Relevance & Research must develop technology-based
     solutions to important and relevant problems \\ \hline 
     Guideline 3: Design Evaluation & Proper valuation mthods must be used to
     demonstrate artifact's quality and efficacy \\ \hline 
     Guideline 4: Research Contributions & Research must provide clear
     contributions to the research areas \\ \hline 
     Guideline 5: Research Rigor & Rigorous methods must be applied to
     construction and evaluation of the artifacts \\ \hline 
     Guideline 6: Design as a Search Process & Research must incorporate a
     search process to find and effective solution to the problem \\ \hline
     Guideline 7: Communication of Research & Research must be effectivey
     communicated to relevant audiences \\ \hline
    \end{tabular}
  \end{center}
  \caption{Design Science Research Guidelines \citep{Hevner2004}}
\end{table}

Requirements for DSR contribution, defined by \citet{March2008}, include (1)
identification of a problem, (2) demonstration that there are no existing
adequate solutions in the area, (3) development of an innovative artifact that
addresses the problem, (4) evaluation of the artifact, (5) communication of the
knowledge added to the area, and (6) understanding of the implications for
theory and practice.

This set of requirements closely resembles the DRS methodology process described
by \citet{Peffers2008} (see Figure \ref{fig:peffers}) and research model phases
found in \citet{Vaishnavi2007} (see Figure \ref{fig:vaishnavi}).

\begin{figure}[h!]
\centering
\includegraphics[height=0.95\textheight]{CH2-F1-Peffers}
\caption[Design Science Research Methodology Process Model]{Design Science
Research Methodology Process Model \citep{Peffers2008}}
\label{fig:peffers}
\end{figure}

\FloatBarrier

\begin{figure}[htp]
\centering
\includegraphics[width=0.5\textwidth]{CH2-F2-Vaishnavi}
\caption[Design Science Reseach Cycle]{Design Science Reseach Cycle \citep{Vaishnavi2007}}
\label{fig:vaishnavi}
\end{figure}

\FloatBarrier

Unlike Vaishnavi's research cycle, Peffer's model distinguishes between
\textit{Demonstration} and \textit{Evaluation} of the artifact. Demonstration is
used to show that the implemented idea works, while evaluation is more formal
from of measuring how well the artifact supports a solution to the problem
\citep{Peffers2008}.

\subsection{Design Science Research Applied to This Project}

The research framework in the current research is adapted from an ICT design
science research cycle established by \citet{Vaishnavi2007}. They identify five
phases in the research model: (a) awareness of a problem, (b) suggestions, (c)
development, (d) evaluation, and (e) conclusion.

\subsubsection{Stage 1. Problem identification and motivation}

\subsubsection{Stage 2. Objectives for a Solution}

\subsubsection{Stage 3. Design and Development}

The prototype development in this project followed established
software engineering practices that interleaved coding and revision, forming
iterative development cycles, as it shown at \ref{fig:prototype}.

\begin{figure}[htb]
\centering
\includegraphics[height=0.3\textheight]{CH2-F3-Prototype}
\caption[Ptototyping]{Prototyping based on \citet*[p.~411]{Sommerville2007}}
\label{fig:prototype}
\end{figure}

\subsubsection{Stage 4. Demonstration}

\subsubsection{Stage 5. Evaluation}

It is important for evaluation to be treated not as an isolated process, but as
a part of design process \citep{Cleven2009}.

Due to the context of this research -- \LLLsn, -- it had a complex evaluation
design. 

Case studies were used to evaluate prototype from the mature students
perspective. This approach was favoured over others due to its internal and
external validity, control and in-depth examination of each case
\citep{Yin2009}.

\subsubsection{Stage 6. Communication}

\section{Methodological Limitations}
\label{sec:limits}

Both positive and negative findings should be weighed against methodological
limitations

\section{Related Work}
\label{sec:related}
As the field of \LLLs became popular, there were a number of studies aimed to
explore \LLLs support in various contexts. To date, research similar to this
project has not been identified, although projects found were a valuable source
of information and examples of previous research experience.

\begin{itemize}

  \item Lifelong Learning in London for
  All\footnote{\url{http://www.lkl.ac.uk/research/l4all.html}} (L4All): This
  project is focused on developing of \LLLs system to support independent
  learners (particularly those 16+ learners who traditionally have not
  participated in higher education) by recording their learning pathways. This
  project aimed to provide lifelong learners in the London region with access to
  information and resources that facilitates their progression from secondary
  education to further education or from secondary education directly to higher
  education \citep{Freitas2006};

  \item The Regional Interoperability Project on Progression for Lifelong
Learning\footnote{\url{http://www.jisc.ac.uk/whatwedo/programmes/edistributed/rippll.aspx
}} (RIPPLL): This project was going to establish a model of cross-sector
collaboration in personal development planning technology in the UK. The aim was
to make interoperable all the major existing electronic systems for study-based
progress files in use in further and higher education to provide an easier
transition process from school to further education \citep{Hartnell-Young2006};

  \item ELGG-Moodle: In autumn 2006 Klagenfurt University, Austria was piloting
the project aimed to integrated Moodle LMS and ELGG platform. This integration
was used for professional development for all academic staff. Project outcomes
provided integration between systems such as single login and file transfer
\citep{Attwell2007}.

  \item Accessible \LLLc for Higher
Education\footnote{\url{http://www.eu4all-project.eu}} (EU4ALL):
\end{itemize}

\section{Summary}