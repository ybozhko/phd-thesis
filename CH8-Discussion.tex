\chapter{Discussion\label{cha:discussion}}
% 2160
This chapter aims to bring together the knowledge developed during this research
project. It looks back at the topics discussed earlier in the thesis in order to
understand various aspects: whether the research questions were answered; what
issues were faced in the course of this project and how they could be avoided in
future research; what the implications for theory and practice should be
considered; and how the outcomes of this research can be practically applied
in the real world.

\section{Reflections on Research Questions}

The overall research goal in this thesis was stated as follows:

\shortquote{\textbf{To suggest improvements to \ep~systems to offer university
students a learner-centred e-learning environment that can provide them with
support and facilitate their \LLLs process.}}

To support achievement of this goal, four research questions were developed
(Chapter \ref{cha:method}) and systematically addressed in the course of this
thesis. The current section reflects back on the questions posed in this
research project.

\shortquote{RQ1. What is the concept of \LLLs and its connection to universities?}

According to reviews of the area described in Chapter \ref{cha:litrev}, it
can be concluded that the concept of \LLLs is not a simple one. Lifelong
learning has gone through various changes in its concept as well as meaning.
Even now, there are debates about what the actual meaning of \LLLs is
\citep{Griffin2002}. The term used in this thesis follows the common
understanding of the theories where \LLLs consists of formal, informal and
non-formal types of learning with long-term, life-wide and self-directed
characteristics \citep{Longworth2003,Rubenson2002,Schuetze2006}. This concept
puts the learner in charge of their learning and moves educational providers --
universities in the context of this thesis -- to the role of learning
facilitators \citep{Boshier2000}.

What do universities have to do with \LLLs anyway? Looking from the tertiary
education perspective, time spent on studying in universities takes about 5\% of
all learning done by an average individual over their life span (calculated
based on statistics provided by Dench \citeyearpar[pp.~28-37]{Dench2010}). While
this might not appear to be much, a university is still considered to be a
place where professionals are prepared to join the world of global economics
and an educated society. As a result, the pressure on universities to produce
highly qualified graduates is high. Current economic and social agendas require
these graduates to possess not only specific professional skills, but also
skills that are called \LLLs and are less likely to be developed by learning in
just formal settings. This is where the need of incorporating other types of
learning in addition to formal learning in universities comes from.

Currently, many universities are looking into ways of providing students
with better support that would reflect their needs as lifelong learners. One of
type of such support is providing a learning environment conducive to the
principles of flexible, life-wide, and self-directed learning.

\shortquote{RQ2. How are available e-tools used to support \LLLs within the
university context?}

Exploring the area of learning spaces (Chapter \ref{cha:systudy}) shows that
the field of ICT currently provides a large number of systems and tools
available for supporting all types of learning. Some of these tools are provided
by educational institutions while others are freely available online on the
Internet. While all of them have benefits for various types of learning, the
gaps that exist between these tools are difficult to bridge. While LMS dominates
in the world of institutional learning spaces and provides support for formal
learning according to university needs, the Web 2.0 tools with their high
potential of supporting informal learning are almost completely neglected by the
universities due to various reasons. These tools cannot offer the required
degree of security and regulation while at the same time providing opportunities
for institutional control and management. Often learners themselves prefer a
separation of these largely social tools away from their classroom.

The advent of \ep~systems in support of learning brings an opportunity to cover
these gaps. They are learner-centred, provide users with private space and a
space for collaboration, as well as opportunities for reflection and sharing.
Although, \ep~systems were developed to meet the high demand of providing
support for \LLLsn, this research showed that it is difficult to measure their
success in doing so. The field of \ep~systems development is still rapidly
changing. Therefore, further exploration of their potential was required.

\shortquote{RQ3. How can LMS and/or \ep~systems be extended to support students
in the university context for \LLLsn?}

An investigation through a series of interviews with the stakeholders resulted
in development of requirements for better support of \LLLs on a system level
(Chapter \ref{cha:model}). Providing support from a LMS perspective was
abandoned at that point as improving of the \ep~system was considered more
important. High priority requirements were translated into features (Section
\ref{sec:elicit}) and implemented in a system prototype as described in Chapter
\ref{cha:prototype}. The Mahara \ep~system was used as a base system for the
development stage.

Looking back at the requirements development stage, it would be valid to ask
whether the information gathered was enough to answer the posed question.
The participant selection procedure showed that the pool of potential
participants with suitable experiences was very small. No information was found
in related software engineering literature that would say how many stakeholders
was enough to start developing requirements. Therefore, the researcher followed
common sense by examining the results of the interviews to understand whether
data saturation had been reached. Having finished interviews with the
participants of the initial sample, the conclusion was that this sample had been
sufficient for the requirements extraction.

In addition, the elicitation of the requirements was conducted under the
assumption that the stakeholders involved in this research were qualified enough
to provide high quality information that could be used later. All lecturers were
experienced in their field of teaching and had practice of utilizing the
\ep~system in their work with students. In turn, the students they recommended
for participation in this project were all mature individuals with experience of
using \ep~systems in their learning. Measuring the stakeholders' expertise or
maturity would go beyond the scope of this project. Notwithstanding, the premise
that the stakeholders had a rich set of knowledge and experiences should be
considered as an underlying assumption behind this research stage.

\shortquote{RQ4. How does this extended environment meet the needs of
stakeholders in university teaching and learning contexts?}

The evaluation stage of this research, as described in Chapter
\ref{cha:evaluation}, attempted to catch various perspectives of the stakeholders by developing a
complex evaluation design that incorporated three studies, each using different
evaluation approach. Through these studies, it was shown that the
representatives of all of the stakeholder groups were overall satisfied with the
new features supporting \LLLs demonstrated in the prototype implementation. It
can be concluded that in general their requirements were met.

Arguably, the artificial settings of all studies might have influenced the
general reliability and validity of the outcomes of this evaluation. A real-life
evaluation was not possible to conduct due to various reasons among which were
the timeframes required for evaluating \LLLs and missing framework of
institutional support for students. This stage was conducted under the
assumption that the conditions of the evaluations would resemble those of the
real world situations. Taking into account that the evaluated system was a
functional prototype and could not be launched into a real world setting, it
was a reasonable trade between getting full scale evaluation results against
getting no results at all. As was required, additional measures were undertaken
to increase validity of the overall evaluation. However, the results of the
evaluations have to be accepted with an understanding of the conditions and
assumptions of the studies.

\section{Reflections on Research Methodology}

Include further reflective commentary on the research methodology used and
discuss the implications of the researcher occupying the role of evaluator in
the research process and how this informs the research result.

\section{Implications For Theory and Practice}

This section discusses the lessons learnt and conclusions that can be drawn
based on results of this research. It suggests the changes that could be made in
various areas in order to improve the quality of \LLLs support in universities.

\subsection[Finding Balance]{Finding balance between what is good for teaching
and good for learning}

The conversations with lecturers over the course of this research has emphasized
the challenges faced in doing justice to the varied demand of teaching, research
and administrative responsibilities. It is not possible to infinitely add on the
lecturers responsibilities every time a university administration decides to
introduce a new solution for supporting learning. Even now, it is difficult for
lecturers to find a balance in what should be used for teaching and learning.
This research confirmed a common belief that different people usually have
different perspectives. For lecturers it is especially important as they have to
look at things not from only their own teaching perspective, but from the
perspective of learners as well. It was easy to see that some of the
participants had difficulties with understanding learners' needs due to not
being able to put themselves into their students' place.

On the other hand, it would not be a reasonable decision to ask lecturers to
employ technologies that might significantly affect their established
teaching process in terms of producing high workloads. For example, changing the
way lecturers give material to students or the way they mark learners'
assignments might increase the time that has to be spent on these activities to
get quality outcomes. Therefore, while introducing new systems to the learning
environment, university management should also perform a holistic review of
approaches to teaching that balances new requirements or added workload with
reduced effort in other areas.

\subsection[Cooperation and Communication]{Importance of cooperation in 
introducing new technologies or new teaching principles}

A surprising to the researcher discovery made during the course of this project
was that communication between various departments was in quite a poor state. If
the lecturers were more engaged with teaching and learning projects carried out
in their departments or schools, they were less likely to know about similar
research projects in other areas of the university. However, it should be
reminded here that these conclusions were made as the results of observations of
practice at Massey University.

Work with the lecturers at various stages of this project showed the importance
of communication and cooperation between the university departments. The
lecturers who were research participants were usually highly surprised when they
discovered that some other departments were using or had past experience of
employing exactly the same technologies with their students. A lot of issues and
troubles for new institutional developments could have been avoided if proper
communication or presenting departments' research was done across the university.

\subsection[Learning to Use New Technology]{Learning to use new technology}

A curious fact was noticed in the course of this research: the students who
participated in various studies of this project had practically no complaints
about LMS used in their university. In contrast, the \ep~system raised
criticisms and expressions of dissatisfaction, although LMS and the \ep~systems
were discussed equally. How is it that the system that has been built for
learners displeases these very learners so much? A possible explanation for this
phenomena might be a lack of student teaching of how to use \ep~systems.

Looking at the current state of LMS use in universities, technology level help
might be all that is required for students to master this type of system. Use of
LMS is generally simple, and can be summarized in such activities as accessing
and uploading files, and in some cases posting to forums or answering quizzes.

Due to long history of LMS being employed in universities, lecturers already
have their own established strategies of using these systems, especially in
relation to delivery of distance education. Furthermore, help for aspiring
lecturers can be acquired from the literature. For example, Packt e-Learning
Library\footnote{\url{http://packtlib.packtpub.com/e-learning} (Accessed April
16, 2012)} currently publishes books covering Moodle, Blackboard, Sakai, and
other LMS. In addition to numerous online resources and literature on how to use
LMS to their full potential, there are staff trainings that are usually
organized by many universities for those lecturers who need extra assistance.

With the support of university administration \citep{DiBenedetto2005}, 
lecturers now need to learn how to guide students through \LLLsn, and need to
master new technology that supports it \citep{Levin2008} -- \ep~systems in the
context of this thesis. Compared to LMS, work with \ep~systems is more complex.
It goes beyond simple file management and requires understanding of the
importance of reflection, purposeful selection of artifacts, and demonstration
of personal achievements. Students who have never been taught all these things
might not see the value of an \ep~system as well.

Understanding of these issues is currently developing in the universities
around the world. For instance, based on the case studies presented at
PebbleBash 2010 conference \citep{PebbleLearningLtd2010a}, many universities in
the UK and Australia have moved towards teaching students how to use \ep s,
showcase skills and achievements, write meaningful reflections, and provide
useful peer feedback.

In addition to the above issues, one of the lecturer-participants of this
research noticed that students should start their learning journey with an
\ep~system on early stages of their studying. Doing so, rather than trying to
fit \ep s into the students' established way of learning, might result in more
positive outcomes. This, in turn, might lead to higher acceptance of the
\ep~concepts and systems among students. Although, this assumption needs to
be carefully tested in the future.

\section{Practical Applications}

The results of this research can be applied to many domains. These can be
education, employment and workplace, knowledge and information management, media
sharing, etc. The principles and implementations of this research can be used
in providing support for various learning processes in universities, such as
development and understanding skills and learning concepts, demonstrating
achievements, assessment of knowledge, and others.

Since its first release in 2006, Mahara -- the system used as the base
\ep~system for prototyping -- gained a world-wide popularity. It is currently
successfully employed in many colleges and universities in the UK, Australia,
USA, and others (as can be seen from the numerous case studies and examples
described in the quarterly Mahara
Newsletter\footnote{\url{https://mahara.org/newsletter} (Accessed April 18,
2012)}). In addition, the 4th MaharaUK
conference\footnote{\url{http://maharauk.org} (Accessed April 18, 2012)} is
underway in July 2012 bringing an opportunity for educators, learning providers,
and developers to share their ideas and experiences on how to enhance the use of
\ep s in education. In light of such success, the Mahara developing community
welcomes contributions that can help to improve this \ep~system.

The outcomes of this research project also have the potential for applications
in the New Zealand context, in higher education as well as secondary. With
additional improvements and necessary corrections based on the results of
evaluations, the prototype implementations can be added to the official release
of Mahara \ep~system which is currently becoming popular in New Zealand with an
expectation of providing \LLLs support for every New Zealand citizen.

As an example of \ep s gaining popularity in New Zealand, MyPortfolio service
(based on the Mahara \ep~system) is currently available nationally for
schools\footnote{\url{http://myportfolio.school.nz} (Accessed April 16, 2012)}.
New Zealand Ministry of Education made it free for all schools in New Zealand
and offered assistance with training school staff until the end of 2012
\citep{NewZealandMinistryofEducation2012}. At the time of writing this thesis,
other discussions were underway on the role which Ako
Aotearoa\footnote{\url{http://akoaotearoa.ac.nz} (Accessed April 16, 2012)} --
New Zealand's National Centre for Tertiary Teaching Excellence -- might be able
to play in supporting similar \ep~service for the entire tertiary
sector\footnote{\url{http://myportfolio.ac.nz} (Accessed April 16, 2012)}.

\section[General Applicability]{General Applicability of the Prototype
Features to the \ep~Systems}

A number of \ep~systems -- PebblePad, BlackBoard \ep, Desire2Learn, eFolio,
Mahara, and ELGG -- were reviewed earlier in this thesis (Chapter
\ref{cha:systudy}) to understand what current \ep~technology can offer in terms
of supporting \LLLsn. Although the prototype implementations were developed in
one specific \ep~system, Mahara (described in Chapter \ref{cha:prototype}), the
ultimate goal was to suggest improvements that could be applied to any existing
\ep~system. Due to the differences in the system architectures and programming
languages, the prototype implementations cannot be directly reused without
adaptation. Conceptually they can be adjusted to be used in other \ep~systems.

Adjusting the implementations for another \ep~system means that the system
should be analysed against the guidelines for supporting \LLLsn. This had been
attempted in the review of the existing \ep~systems (Chapter \ref{cha:systudy}).
Considering that in the course of this research these guidelines were translated
into system requirements (Chapter \ref{cha:model}), the gaps in the features
offered by the majority of the existing \ep~systems can be identified. This can
help the \ep~systems' developers to add new features and functionality to the
existing systems with the aim of improving support for \LLLs and learner
experience. In addition to the improvements that can be done in the existing
systems, the identified features can also be used as a guiding specification for
the requirements management when a new \ep~system is developed.

It is possible that the features for supporting \LLLs can be implemented in many
different ways, other than described in this thesis. The prototype just
demonstrated one of the ways of implementing these features. It does not claim
to be the only or the best way of development. However, it is important to
remember that the studies conducted to evaluate the implementations showed that
these were positively accepted by stakeholders.
