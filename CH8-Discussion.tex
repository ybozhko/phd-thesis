\chapter{Discussion\label{cha:discussion}}
%310
This chapter aims to bring together. It looks back at the topics
discussed earlier in the thesis in order to understand 

issues and how they could have been avoided

whether the research questions were answered

what are the implications of this research 

and how the knowledge can be applied in real world

\section{Answers to Research Questions}

The overall research goals in this thesis was stated as follows:

\shortquote{\textbf{To design and implement a learner-centered e-learning
environment which will support and facilitate the \LLLs process in
universities.}}

To support achievement of this goal, four research questions were developed and
systematically addressed throughout this thesis. The answers to these questions
are summarized in this section.

\shortquote{What is the concept of \LLLs and its connection to the
universities?}

According to the reviews of the area described in Chapter \ref{cha:litrev}, it
can be concluded that the concept of \LLLs is not as simple. Lifelong learning
went through various changes in concept as well as in meaning. The term employed
in this thesis \ldots

\shortquote{What e-tools are available to support \LLLs within the
university context?}

Exploring the area of learning spaces (Chapter \ref{cha:systudy}) showed that
currently field of ICT provides a large number of systems and tools available
for supporting learning.

\shortquote{How can LMS and/or ePortfolio systems be extended to support
students in a university context in lifelong learning?}

An investigation through a series of interviews with the stakeholders resulted
in developemnt of requirements for better support of \LLLs on a system level
(Chapter \ref{cha:model}). Providing support from LMS perspective was abandoned
at that point as improvement of the \ep~system was considered more important.
High priority requirements were translated into features (Section
\ref{sec:elicit}) and implemented in a system prototype as described in Chapter
\ref{cha:prototype}.

\shortquote{Does this extended environment meet the needs of the stakeholders in
university teaching and learning contexts?}

Through the number of evaluation studies described in Chapter
\ref{cha:evaluation}, it was show that \ldots

\section{Implications}

Text

\section{Applications}

Text
