\chapter{Discussion\label{cha:discussion}}
%2325
This chapter aims to bring together the knowledge developed during this research
project. It looks back at the topics discussed earlier in the thesis in order to
understand various aspects: whether the research questions were answered; what
issues were faced in the course of this project and how they could be avoided in
future research; what are the implications of this research for theory and
practice; and how the knowledge can be applied in real world.

\section{Answers to Research Questions}

The overall research goals in this thesis was stated as follows:

\shortquote{\textbf{To design and implement a learner-centered e-learning
environment which will support and facilitate the \LLLs process in
universities.}}

To support achievement of this goal, four research questions were developed
(Chapter \ref{cha:method}) and systematically addressed throughout this thesis.
The answers to these questions are summarized in this section.

\shortquote{RQ1. What is the concept of \LLLs and its connection to universities?}

According to the reviews of the area described in Chapter \ref{cha:litrev}, it
can be concluded that the concept of \LLLs is not simple. Lifelong learning
went through various changes in concept as well as in meaning. Even now,
there are debates about what the actual meaning of \LLLs is. The term employed
in this thesis follows the common understanding of the theories where \LLLs
consists of formal, informal and non-formal types of  earning with long-term,
life-wide and self-directed characteristics. This concept puts the learner in
charge of learning and moves educational providers -- universities in the
context of this thesis -- to the role of learning facilitators.

So, what do universities have to do with \LLLs anyway? Looking from tertiary
education perspective, time spent on studying in the universities takes about
5\% of all learning done by an average individual over their life span
(calculated based on statistics provided by Dench
\citeyearpar[28-37]{Dench2010}). While this might not look like too much, a
university is still considered to be a place where professionals are prepared to
join the world of global economics. As a result, the pressure on the university
to produce highly qualified graduates is high. Current economics demands these
graduates to possess not only their specific professional skills, but also
skills that are called \LLLs and are less likely to be developed by just
learning in the classroom. This is where the need of incorporating other types
of learning in addition to formal learning in the universities comes from.
Currently, the universities are looking into ways of providing students with
better support that would reflect their needs as lifelong learners. One of the
types of such support is providing a proper learning environment.

\shortquote{RQ2. How are available e-tools used to support \LLLs within the
university context?}

Exploring the area of learning spaces (Chapter \ref{cha:systudy}) shows that
the field of ICT currently provides a large number of systems and tools
available for supporting all types of learning. Some of these tools are provided
by educational institutions while others are freely available online on the
Internet. All of them have their benefit for various types of learning. However,
the gaps that exist between these tools are difficult to bridge. While LMS
dominates in the world of institutional spaces and provides support for formal
learning according to the university standards, the Web 2.0 tools with their
high potential of supporting informal learning are almost completely neglected
by the universities due to various reasons.

\ep~systems coming to the stage of learning brings an opportunity to cover these
gaps. Although, the \ep~system are developed to provide a highly demanded
support for \LLLsn, it is difficult to measure its success in doing so. The
field of \ep~systems development is still young and therefore rapidly changing.

This research came about very timely, when a lot of studies from the first
\ep~systems' adoption waves were finished and needed understanding of what went
wrong and how the discovered problems can be solved \citep{Batson2010}. Taking
their experience and adding new knowledge to the existing grounds might bring
better theories that would change and improve the ways this technology could be
adopted and used in the future.

\shortquote{RQ3. How can LMS and/or \ep~systems be extended to support students
in the university context for \LLLsn?}

An investigation through a series of interviews with the stakeholders resulted
in development of requirements for better support of \LLLs on a system level
(Chapter \ref{cha:model}). Providing support from LMS perspective was abandoned
at that point as improvement of the \ep~system was considered more important.
High priority requirements were translated into features (Section
\ref{sec:elicit}) and implemented in a system prototype as described in Chapter
\ref{cha:prototype}.

Looking back at the requirements development stage, it would be valid to ask
whether the information gathered was enough to answer the posed question.
Participants selection procedure showed that the pool of potential participants
with suitable experiences was very small. No information was found in related
software engineering literature that would say how many stakeholders was enough
to start developing requirements. Therefore, the researcher followed common
sense by examining the results of the interviews to understand whether data
saturation had been reached.

In addition, the elicitation of the requirements was conducted under the
assumption that the stakeholders involved in this research were competent enough
to provide high quality information that could be used later. All lecturers were
experienced in their field of teaching and had practice of utilizing the
\ep~system in their work with students. In turn, the students they recommended
for participation in this project were all mature individuals with experience of
using \ep~systems in their learning. Measuring the stakeholders expertise or
maturity would go far beyond the scope of this project. Notwithstanding, this
should be considered as an underlying assumption behind this research stage.

\shortquote{RQ4. How does this extended environment meet the needs of
stakeholders in university teaching and learning contexts?}

Evaluation stage of this research, as described in Chapter \ref{cha:evaluation},
attempted to catch various perspectives of the stakeholders by developing a
complex evaluation design that incorporated three studies, each using different
evaluation approach. Through these studies it was shown that the representatives
of all of the stakeholders were largely satisfied with the implementations in
the \ep~system prototype. Therefore, it can be concluded that in general their
requirements were met.

Arguably, artificial settings of all studies might have influenced the general
reliability of the outcomes of this evaluation. This stage was conducted
under the assumption that the conditions of the evaluations would resemble those
of the real world situations. Taking into account that the evaluated system was
a functional prototype and could not be launched into the real worlds setting,
it was a reasonable trade between getting high quality evaluation results
against getting no results at all. As was required, additional measures were
undertaken to increase validity of the overall evaluation. However, the results
of the evaluations have to be accepted with understanding the conditions and
assumptions of the studies.

\section{Implications For Theory and Practice}

This section discusses the lessons learnt and conclusions that can be drawn
based on results of this research. It suggests the changes that could be made in
various areas in order to improve the quality of \LLLs support in universities.

\subsection[Cooperation and Communication]{Work directed towards introducing
new technologies or new teaching principles should be done in cooperation}

A surprising discovery made during the course of this research was that
communication between various departments was poor. If the lecturers were
familiar with the research carried out in their departments or schools, they
were less likely to know about similar research projects in other areas of
the university. Work with the lecturers at various stages of this project showed
the importantce of communication and cooperation between the universty
departments. The lecturers who were research participants were usually highly
surprised when they discovered that some other departments were using or had
past experience of employing exactly the same technologies with their students.
A lot of issues and troubles could have been avoided if proper communication or
presenting departments' research was done across the university.

\subsection[Finding Balance]{Finding balance between what is good for teaching
and good for learning}

For lecturers it might be difficult to find a balance in what should be used for
teaching and learning. This research confirmed a known fact that different
people have different perspectives. For lecturers it is especially important as
they have to look at things not from only their own teaching perspective, but as
well from the perspective of learners. It was easy to notice that some of the
participants had difficulty with understanding learners' needs due to not being
able to put themselves into their students' place.

On the other hand, it would not be a reasonable decision to ask lecturers to
employ technologies that might significantly adversly affect their established
teaching process. For example, changing the way lecturers give material to
students or the way they mark learners' assignments might increase the time that
has to be spent on these activities to get quality outcomes. Therefore, while
introducing new systems to the learning environment, university management
should also make sure that it would be possible for the lecturers to find this
kind of balance without harm to learning and teaching process.

\subsection[Learning to Use Technology]{Using new technology needs to be learnt
as any other things around}

A curious fact was noticed in the course of this research: the students who
participated in various studies of this project had practically no complaints
about LMS used in their university, although LMS and the \ep~systems were
discussed equally. In contrast, the \ep~system raised criticism and expressions
of dissatisfaction. How is it that the system that has been built for learners
displeases these very learners so much? There is a number of possible
explanations for this. Some of them are discussed here. Others are explored in
the following section.

Universities usually offer a number of supporting courses for students on how to
use the institutional systems. Unfortunately, the \ep~system is not on the list
of such systems. One of the lecturer-participants of this research noticed that
using the \ep~system should be taught to students the same way as any other
system. However, currently, in best case students get a number of video tutorials
or manual web-pages on the \ep~system use. These were shown to be insufficient. 

\ep~trials, and this research is not an exception, are rarely carried out with
the students who are recent high school graduates. There is a number of reasons
for this: students who have already studied at the university for some time are
more familiar with the requirements for the systems, better understand the needs
and can provide their opinion comparing with their own previous experience.

However, looking from different perspective might draw another picture. While
studying at the university, students are developing their own expectations and
experiences of using various systems. In this case, if the \ep~system is
introduced at the later stages of a study program, students who have never been
taught the value of reflection, purposeful selection of artifacts and
demonstration of personal achievements might not see the value of the \ep~system
as well. Starting their learning journey with the \ep~system rather than trying
to fit \ep~into the students' usual way of learning might result in more
positive outcomes and lead to higher acceptance of the \ep~concepts and systems
among the students. Although, this assumption needs to be carefully tested in
the future.

\subsection[Shaping Technology for Learning Needs]{Learning theory and practice
should shape the way technology is developed, not the other way around}

Following the discussion in the previous section, the reality of nowadays is
such that technology is shaping the way learning and teaching are performed. It
is not unusual for the university to employ the new technology without
understanding the needs of learners, but simply because it offers higher degree
of security and regulation as well as more opportunities for control and
management. An approach that is driven by the needs of the university rather
than the learner cannot have much potential from the learner's perspective. This
research attempted to take into account this perspective placing a learner, a
student in this context, and their needs first.

Unlike in business settings where a consumer usually dictates the software
features that are being developed, students generally do not have such influence
on technology which they have to use. This means that very often learning
technology is being developed separately from its main stakeholder -- a learner.
As a result, such technology is more likely to reflect developers' understanding
of the problem rather than users'. One of the recommendations based on the
explorations of the area of this project is to create an environment where
systems for learning are being built in cooperation with learners and teaching
experts. Listening to users' needs first and the needs of finances and
management second is very important for successful outcomes of any project. This
would ensure that the developed technology created for learning is accepted  by
its intended consumers -- the learners.

\section{Applications}

The results of this research can be applied in many areas and domains. Among
these can be education, employment and workplace, knowledge and information
management, media sharing, etc. The principles and implementations of this
research can be used for providing support for various learning processes in the
universities, such as developing understanding of skills and learning concepts,
demonstrating achievements, assessment of knowledge, and others. This was shown
by a number of evaluation studies described in Chapter \ref{cha:evaluation}.

Formal requirements specification for \LLLs support developed in the course of
this project (Chapters \ref{cha:model} and \ref{cha:prototype}) can be used by
the \ep~systems' developers to add new features and functionality to the
existing systems with the aim of improving learners' experience. This
specification can also be used as a guiding specification in requirements
management of the new system development.

To support the applicability of this research, with additional improvements
and necessary corrections its implementations could be added to the official
release of Mahara \ep~system -- the system used as a basic \ep~system for
prototyping. This \ep~system is getting more popularity in New Zealand with an
expectation of providing \LLLs support for every New Zealand citizen.
