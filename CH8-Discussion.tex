\chapter{Discussion\label{cha:discussion}}
%1425
This chapter aims to bring together the knowledge developed during this research
project. It looks back at the topics discussed earlier in the thesis in order to
understand various things: whether the research questions were answered; what
issues were faced in the course of this project and how they could be avoided in
future research; what are the implications of this research for theory and
practice; and how the knowledge can be applied in real world.

%Each paragraph establishing or clarifying position, confirming or questioning
% the results, offering alternative meaning or interpretation.

\section{Answers to Research Questions}

The overall research goals in this thesis was stated as follows:

\shortquote{\textbf{To design and implement a learner-centered e-learning
environment which will support and facilitate the \LLLs process in
universities.}}

To support achievement of this goal, four research questions were developed
(Chapter \ref{cha:method}) and systematically addressed throughout this thesis.
The answers to these questions are summarized in this section.

\shortquote{What is the concept of \LLLs and its connection to the
universities?}

According to the reviews of the area described in Chapter \ref{cha:litrev}, it
can be concluded that the concept of \LLLs is not as simple. Lifelong learning
went through various changes in concept as well as in meaning. Even now,
there are known debates what the actual meaning of \LLLs is. The term employed
in this thesis follows the common understanding of the theories where \LLLs
consists of formal, informal and non-formal types of  earning with long-term,
life-wide and self-directed characteristics. This concept puts learner in charge
of learning and moves educational providers -- universities in the context of
this thesis -- to the role of learning facilitators.

So, what universities have to do with \LLLs anyway? Looking from tertiary
education perspective, time spent on studying in the universities takes about
5\% of all learning done by an average individual over their life span. While
this might not look like too much, a university is still considered to be a
place where professionals are prepared to join the world of global economics. As
a result, the pressure on the university to produce highly qualified graduates
is high. Current economics demands these graduates to possess not only their
specific professional skills, but also skills that are called \LLLs and are less
likely to be developed by just learning in the classroom. This is where the
need of incorporating other types of learning in addition to formal learning in
the universities comes from.

\shortquote{What e-tools are available to support \LLLs within the university
context?}

Exploring the area of learning spaces (Chapter \ref{cha:systudy}) shows that
currently field of ICT provides a large number of systems and tools available
for supporting learning. Some of these tools are provided by educational
institutions while others are freely available online on the Internet. All
of them have their benefit for various types of learning. However, the gaps
that exist between these tools are difficult to bridge. While LMS dominates in
the world of institutional spaces and provides support for formal learning
according to university standards, the Web 2.0 tools with their high potential
of supporting informal learning are almost completely neglected by the
universities due to various reasons.

\ep~systems coming to the stage of learning brings an opportunity to cover these
gaps. Although, the \ep~system are claimed to be developed to provide highly
demanded support for \LLLsn, it is difficult to measure its success in doing so.
The field of \ep~systems development is still young and therefore rapidly
changing with normal to any technology trend ups and downs.

This research came about very timely, when a lot of studies from the first
\ep~adoption waves were finished and needed understanding of what went wrong and
what went right. Taking their experience and adding new knowledge to the
existing grounds might bring better theories that would change and improve the
ways this technology is used right now.

\shortquote{How can LMS and/or \ep~systems be extended to support students in a
university context in \LLLsn?}

An investigation through a series of interviews with the stakeholders resulted
in development of requirements for better support of \LLLs on a system level
(Chapter \ref{cha:model}). Providing support from LMS perspective was abandoned
at that point as improvement of the \ep~system was considered more important.
High priority requirements were translated into features (Section
\ref{sec:elicit}) and implemented in a system prototype as described in Chapter
\ref{cha:prototype}.

Looking back at the requirements developement stage, it would be valid to ask
whether the information gathered was enough to answer the posed question.
Participants selection procedure showed that the pool of potential participants
with suitable experiences was very small. No information was found in related
software engineering literature that would say how many stakeholders was enough
to start developing requirements. Therefore, the researcher followed common
sense by examining the results of the interviews to understand whether data
saturation had been reached.

In addition, the elicitation of the requirements was conducted under the
assumption that the stakeholders involved in this research were competent enough
to provide high quality information that could be used later. All lecturers were
experienced in their field of teaching and had practice of utilizing the
\ep~system in their work with students. In turn, the students they recommended
for participation in this project were all mature individuals with experience of
using \ep~systems in their learning. Measuring the stakeholders expertice or
maturity would go beyond the scope of this project; nonetheless, this should be
considered as underlying assumption behind this research stage.

\shortquote{Does this extended environment meet the needs of the stakeholders in
university teaching and learning contexts?}

Through the number of evaluation studies described in Chapter
\ref{cha:evaluation}, it was shown that the stakeholders were satisfied with the
implementations in the \ep~system prototype. Therefore, it can be concluded that
in general their requirements were met. Arguably, artificial settings of all
studies might have influenced the genaral reliability of the outcomes of this
evaluation. Although, taking into account that the system was a functional
prototype and could not be launched into the real worlds setting, it was a
reasonalble trade between getting high quality evaluation results against
getting no results at all. As it was required, additional measures were
undertaken to increase validity of the overall evaluation.

The results of the evaluations in this case have to be accepted with some
precautions.

\section{Implications For Theory and Practice}

%Of what use is this research to educators and what they could/might do
%differently now?
This section discusses the lessons learnt and conclusions that can be drawn
based on results of this research. It suggests the changes that could be made in
various areas in order to improve the quality of \LLLs support in universities.

\subsection[Cooperation and Communication]{Work directed towards introducing
new technologies or new teaching principles should be done in cooperation}

Surprising discovery made during the course of this research was that
communication between various departments was largely non-existent. If the
lecturers were familiar about the research that was carried out in their
departments or schools, they were less likely to know about similar research in
other structures of the university. Work with the lecturers at various stages of
this project showed how necessary communication and cooperation are. Lecturers
who were research participants were usually highly surprised when they
discovered that some other departments were using or had past experience of
employing exactly the same technologies with their students. A lot of issues and
troubles could have been avoided if proper communication or presenting
departments' research was done across the university.

\subsection[Finding Balance]{Finding balance between what is good for teaching
and good for learning}

For lecturers it might be difficult to find a balance what should be used for
teaching and learning. This research confirmed a known fact that different
people have different perspectives. For lecturers it is especially important as
they have to look at things not from only their own teaching perspective, but as
well from the perspective of learners. It was easy to notice that some of the
participants had difficulty with understanding learners' needs due to not being
able to put themselves into their students' place.

On the other hand, it would be not fair to ask lecturers to employ technologies
that would significantly affect their teaching process not in positive way. For
example, changing the way lecturers give material to students or the way they
mark learners' assignments might increase the time that has to be spent on
these activities to get quality outcomes. Therefore, while introducing new
systems to the learning environment, university management should also make
sure that it would be possible for the lecturers to find this kind of balance
without harm to learning and teaching process.

\subsection[Learning to Use Technology]{Using new technology needs to be learnt
as any other things around}

\subsection[Shaping Technology for Learning Needs]{Learning theory and practice
should shape the way technology is developed, not the other way around}



\section{Applications}

Text
