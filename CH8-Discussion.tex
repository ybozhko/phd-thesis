\chapter{Discussion\label{cha:discussion}}
%2160
This chapter aims to bring together the knowledge developed during this research
project. It looks back at the topics discussed earlier in the thesis in order to
understand various aspects: whether the research questions were answered; what
issues were faced in the course of this project and how they could be avoided in
future research; whether there are any technical issues that should to be
considered; and how the outcomes of this research can be practically applied in
real world.

\section{Reflections on Research Questions}

The overall research goal in this thesis was stated as follows:

\shortquote{\textbf{To design and implement a learner-centred e-learning
environment which will support and facilitate the \LLLs process in
universities.}}

To support achievement of this goal, four research questions were developed
(Chapter \ref{cha:method}) and systematically addressed in the course of this
thesis. Current section reflects back on the questions posed in this research
project.

\shortquote{RQ1. What is the concept of \LLLs and its connection to universities?}

According to the reviews of the area described in Chapter \ref{cha:litrev}, it
can be concluded that the concept of \LLLs is not simple. Lifelong learning
went through various changes in its concept as well as in its meaning. Even now,
there are debates about what the actual meaning of \LLLs is \citep{Griffin2002}.
The term employed in this thesis follows the common understanding of the
theories where \LLLs consists of formal, informal and non-formal types of
learning with long-term, life-wide and self-directed characteristics
\citep{Longworth2003,Rubenson2002,Schuetze2006}. This concept puts the learner
in charge of their learning and moves educational providers -- universities in
the context of this thesis -- to the role of learning facilitators
\citep{Boshier2000}.

So, what do universities have to do with \LLLs anyway? Looking from tertiary
education perspective, time spent on studying in universities takes about 5\% of
all learning done by an average individual over their life span (calculated
based on statistics provided by Dench \citeyearpar[pp.~28-37]{Dench2010}). While
this might not look like too much, a university is still considered to be a
place where professionals are prepared to join the world of global economics
and an educated society. As a result, the pressure on universities to produce
highly qualified graduates is high. Current economic and social agendas require
these graduates to possess not only their specific professional skills, but also
skills that are called \LLLs and are less likely to be developed by just
learning in formal settings. This is where the need of incorporating other types
of learning in addition to formal learning in universities comes from.

Currently, many universities are looking into the ways of providing students
with better support that would reflect their needs as lifelong learners. One of
the types of such support is providing a learning environment conducive to the
principles of flexible, life-wide, and self-directed learning.

\shortquote{RQ2. How are available e-tools used to support \LLLs within the
university context?}

Exploring the area of learning spaces (Chapter \ref{cha:systudy}) shows that
the field of ICT currently provides a large number of systems and tools
available for supporting all types of learning. Some of these tools are provided
by educational institutions while others are freely available online on the
Internet. All of them have their benefits for various types of learning.
However, the gaps that exist between these tools are difficult to bridge. While
LMS dominates in the world of institutional learning spaces and provides support
for formal learning according to the university needs, the Web 2.0 tools with
their high potential of supporting informal learning are almost completely
neglected by the universities due to various reasons. These tools cannot offer a
required degree of security and regulation while at the same time the
opportunities for institutional control and management. Often, learners
themselves prefer a separation of these largely social tools away from their
classroom.

The advent of \ep~systems in support of learning brings an opportunity to cover
these gaps. They are learner-centred, provide users with private space and a
space for collaboration, as well as opportunities for reflection and sharing.
Although, \ep~systems were developed to meet the high demand of providing
support for \LLLsn, this research showed that it is difficult to measure their
success in doing so. The field of \ep~systems development is still rapidly
changing. Therefore, a further exploration of their potential was required.

\vspace{2 mm}

\shortquote{RQ3. How can LMS and/or \ep~systems be extended to support students
in the university context for \LLLsn?}

\vspace{2 mm}

An investigation through a series of interviews with the stakeholders resulted
in development of requirements for better support of \LLLs on a system level
(Chapter \ref{cha:model}). Providing support from LMS perspective was abandoned
at that point as improving of the \ep~system was considered more important.
High priority requirements were translated into features (Section
\ref{sec:elicit}) and implemented in a system prototype as described in Chapter
\ref{cha:prototype}. The Mahara \ep~system was used as a base system for the
development stage.

\vspace{2 mm}

Looking back at the requirements development stage, it would be valid to ask
whether the information gathered was enough to answer the posed question.
Participants selection procedure showed that the pool of potential participants
with suitable experiences was very small. No information was found in related
software engineering literature that would say how many stakeholders was enough
to start developing requirements. Therefore, the researcher followed common
sense by examining the results of the interviews to understand whether data
saturation had been reached. Having finished interviews with the participants
of the initial sample, the conclusion was that this sample had been sufficient
for the requirements extraction.

\vspace{2 mm}

In addition, the elicitation of the requirements was conducted under the
assumption that the stakeholders involved in this research were qualified enough
to provide high quality information that could be used later. All lecturers were
experienced in their field of teaching and had practice of utilizing the
\ep~system in their work with students. In turn, the students they recommended
for participation in this project were all mature individuals with experience of
using \ep~systems in their learning. Measuring the stakeholders expertise or
maturity would go beyond the scope of this project. Notwithstanding, the premise
that the stakeholders had a rich set of knowledge and experiences should be
considered as an underlying assumption behind this research stage.

\shortquote{RQ4. How does this extended environment meet the needs of
stakeholders in university teaching and learning contexts?}

Evaluation stage of this research, as described in Chapter \ref{cha:evaluation},
attempted to catch various perspectives of the stakeholders by developing a
complex evaluation design that incorporated three studies, each using different
evaluation approach. Through these studies it was shown that the representatives
of all of the stakeholder groups were overall satisfied with the new features
supporting \LLLs demonstrated with the prototype implementation. Therefore, it
can be concluded that in general their requirements were met.

Arguably, artificial settings of all studies might have influenced the general
reliability and validity of the outcomes of this evaluation. A real-life
evaluation was not possible to conduct due to various reasons among which were
the timeframes required for evaluating \LLLs and missing framework of
institutional support for students. This stage was conducted under the
assumption that the conditions of the evaluations would resemble those of the
real world situations. Taking into account that the evaluated system was a
functional prototype and could not be launched into the real worlds setting, it
was a reasonable trade between getting full scale evaluation results against
getting no results at all. As was required, additional measures were undertaken
to increase validity of the overall evaluation. However, the results of the
evaluations have to be accepted with understanding the conditions and
assumptions of the studies.

\section{Technical Considerations}
A number of technologies were used in this project to implement the prototype.
The main focus was on utilizing the new features offered by HTML5 standard and
reusing of existing JavaScript libraries (described in Chapter
\ref{cha:prototype}). In the course of this project, these technologies showed
to be a good choice as they allowed for rapid prototype interactions and easy
code maintenance and execution.

The benefit of using HTML5 for extracting video/audio fragments and drawing
concept maps was that it did not require any additional plugins to be installed
on the user side. This can be considered an advantage over Flash-based
implementations which depend on plugin and sometimes may require a specific
version of this proprietary plugin to perform correctly. Using HTML5 helped to
ensure that the prototype features would work on most web browsers as expected.

Along with the benefits, these technologies also created a number of challenges
for the development. Among these was a relative novelty of the HTML5 standard
compared to the established and widely-used HTML4. The Web Hypertext Application
Technology Working Group\footnote{\url{http://www.whatwg.org} (Accessed April
16, 2012)} (WHATWG) and World Wide Web Consortium\footnote{\url{http://www.w3.org}  (Accessed April 16,
2012)} (W3C) started working together on HTML5 standard in 2007
\citep{WorldWideWebConsortium2012}. However, at the time of the prototype
development in 2011 this standard was still a working draft. As a result, some
browsers did not provide full support for the new features offered by HTML5 such
as embedded video/audio and canvas elements.

To provide dynamic user-website interactions and enhanced user interfaces, the
implementations also used scripts, libraries, and modules based on
JavaScript. This can be a potential problem for users who have JavaScript
disabled on their browsers as they would not be able to use features that rely
on JavaScript functionality. Especially, it can be a case for institutional
environments where computers are usually set up by the IT support groups.
According to the researcher's experience, these tend to disable JavaScript
executions for security reasons (for example, to prevent cross site scripting
attacks).

Despite these challenges, it can be concluded that technologies employed in the
prototype development offered a wide range of libraries, modules, and APIs
that made implementations less complicated and time consuming. At the same time,
these technologies helped to develop high quality features and functionalities,
as well as acceptable user interfaces.

\section{Practical Applications}

The results of this research can be applied to many domains. Among these can be
education, employment and workplace, knowledge and information management, media
sharing, etc. The principles and implementations of this research can be used
in providing support for various learning processes in universities, such as
development and understanding skills and learning concepts, demonstrating
achievements, assessment of knowledge, and others.

Moreover, since its first release in 2006, Mahara -- the system used as a basic
\ep~system for prototyping -- gained a world-wide popularity. It is currently
successfully employed in many colleges and universities in the UK, Australia,
USA, and others (as can be seen from the numerous case studies and examples
described in the quarterly Mahara
Newsletter\footnote{\url{https://mahara.org/newsletter} (Accessed April 18,
2012)}). In addition, the 4th MaharaUK
conference\footnote{\url{http://maharauk.org} (Accessed April 18, 2012)} is
underway in July 2012 bringing an opportunity for educators, learning providers,
and developers to share their ideas and experiences on how to enhance the use of
\ep s in education. In light of such success, Mahara developing community
welcomes contributions that can help to improve this \ep~system.

The outcomes of this research project also have the potential for applications
in the New Zealand context, in higher education as well as secondary. With
additional improvements and necessary corrections based on the results of
evaluations, the prototype implementations can be added to the official release
of Mahara \ep~system which is currently getting popular in New Zealand with an
expectation of providing \LLLs support for every New Zealand citizen.

As an example of \ep s gaining popularity in New Zealand, MyPortfolio service
(based on the Mahara \ep~system) is currently available nationally for
schools\footnote{\url{http://myportfolio.school.nz} (Accessed April 16, 2012)}.
New Zealand Ministry of Education made it free for all schools in New Zealand
and offered assistance with training school staff until the end of 2012
\citep{NewZealandMinistryofEducation2012}. At the time of writing this thesis,
other discussions were underway on the role which Ako
Aotearoa\footnote{\url{http://akoaotearoa.ac.nz} (Accessed April 16, 2012)} --
New Zealand's National Centre for Tertiary Teaching Excellence -- might be able
to play in supporting similar \ep~service for the entire tertiary
sector\footnote{\url{http://myportfolio.ac.nz} (Accessed April 16, 2012)}.

\section[General Applicability]{General Applicability of the Prototype
Features to the \ep~Systems}

A number of \ep~systems -- PebblePad, BlackBoard \ep, Desire2Learn, eFolio,
Mahara, and ELGG -- were reviewed earlier in this thesis (Chapter
\ref{cha:systudy}) to understand what current \ep~technology can offer in terms
of supporting \LLLsn. Although the prototype implementations were developed in
one specific \ep~system, Mahara (described in Chapter \ref{cha:prototype}), the
ultimate goal was to suggest improvements that could be applied to any existing
\ep~system. Due to the differences in the system architectures and programming
languages, the prototype implementations cannot be directly reused without
adaptation. However, conceptually they can be adjusted to be used in other
\ep~systems.

Adjusting the implementations for another \ep~system means that the system
should be analysed against the guidelines for supporting \LLLsn, as had been
attempted in the review of the existing \ep~systems (Chapter \ref{cha:systudy}).
Considering that in the course of this research these guidelines were translated
into system requirements (Chapter \ref{cha:model}), the gaps in the features
offered by the majority of the existing \ep~systems can be identified. This can
help the \ep~systems' developers to add new features and functionality to the
existing systems with the aim of improving support for \LLLs and learner
experience. In addition to the improvements that can be done in the existing
systems, the identified features can also be used as a guiding specification for
the requirements management when a new \ep~system is developed.

It is possible that the features for supporting \LLLs can be implemented in many
different ways, other than described in this thesis. The prototype just
demonstrated one of the ways of implementing these features. It does not claim
to be the only or the best way of development. However, it is important to
remember that the studies conducted to evaluate the implementations showed that
these were positively accepted by stakeholders.
