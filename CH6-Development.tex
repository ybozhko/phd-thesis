\chapter{Prototype - Development and Implementation\label{cha:prototype}}

\section{Architecture}

\section{Development Toolkit}

\section{Implementations}

\subsection{Version Control Elements}

\subsection{Concept Map Module}
McAleese \citeyearpar{Mcaleese1998} formally defines a concept map as a directed
acyclic graph that consists of a set of Concept Labels and a non-empty set of
Relationships between Concepts. Putting it simply, concept maps are graphical
representation of the hierarchy of knowledge concepts and connections between
them \citep{Novak2008}.

Concept maps are dynamic, process-oriented and give learners an opportunity to
engage in the learning process [7] which is important for lifelong learning
[10], [11]. Maps are created over time by the learner who is engaged in a
process of reflection, collecting and selecting appropriate examples of their
work. With concept maps learners can interpret their personal knowledge and map
this knowledge and individual examples against the existing theories. The
hierarchical nature of the concept map allows for organizing concepts from the
high level abstract concept to the more specific concepts. This property can be
used by students for managing and structuring data in their \ep s.



Describing future directions for \ep~technology, Cambridge
\citeyearpar{Cambridge2010} suggested that visualization in the form of concept
maps could be a potential way of generating reflections.
 
\subsection{Artefacts' Fragments Extraction}

\subsection{Progress Tracking}

\subsection{Advanced Sharing}

\section{Prototype Iterations and User Tests}
 
\section{Summary}