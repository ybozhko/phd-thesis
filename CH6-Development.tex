\chapter{Prototype - Development and Implementation\label{cha:prototype}}

\section{Architecture}

Draw a picture LMS eP

\section{Development Toolkit}

Eclipse . Packages of jaquery, HTML5 for fragments, etc. Lots of javascript for
better user interactions.

\section{Implementations}

Note: Insert a table to say which feature addresses which problem.

\subsection{Version Control Elements}

\subsection{Concept Map Module}
\citet{Mcaleese1998} formally defines a concept map as a directed
acyclic graph that consists of a set of Concept Labels and a non-empty set of
Relationships between Concepts. Putting it simply, concept maps are graphical
representation of the hierarchy of knowledge concepts and connections between
them \citep{Novak2008}.

Concept maps are dynamic, process-oriented and give learners an opportunity to
engage in the learning process \citep{Mcaleese1998} which is important for \LLLs 
\citep{Schuetze2006,Divjak2004}. Maps are created over time by the learner who is
engaged in a process of reflection, collecting and selecting appropriate examples of their
work. With concept maps learners can interpret their personal knowledge and map
this knowledge and individual examples against the existing theories. The
hierarchical nature of the concept map allows for organizing concepts from the
high level abstract concept to the more specific concepts. This property can be
used by students for managing and structuring data in their \ep s.

Concept maps have been already successfully used to in education to communicate
complex ideas, assess understanding of learning objectives, elicit knowledge and
provide conceptual frame for learning \citep{Novak2010}.


Describing future directions for \ep~technology, Cambridge
\citeyearpar{Cambridge2010} suggested that visualization in the form of concept
maps could be a potential way of generating reflections.

\subsection{Artifacts' Fragments Extraction}

This would allow the presentation of different arteficts to different audiences,
but at the same time saving duplication of materials as all artefacts will be
hosted together.

\subsection{Progress Tracking}

\subsection{Advanced Sharing Options}

\section{Prototype Iterations and User Tests}
 
\section{Summary}