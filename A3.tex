\chapter{Component Requirements \label{app:specification}}

\section{User Stories}

User stories were used in this thesis for representing requirements of the
implemented features. They are an easy way of stating functional requirements
and communicating them in a way that can be understood by end users
\citep{Crispin2003}.

Each user story tells a story about user interacting with the system. It is not
necessary to capture everything at once in user stories. Usually, some most
valuable aspects of the system features are described.

A standard way of capturing user stories is in the following form
\citep{Cohn2004,Coplien2010}:

\shortquote{As a [role/actor] I can [function/action] so that
[rationale/achievement].}

For example,

\shortquote{As a student I can perform a search on concept maps to help me find
the concepts.}

Each user story can be accompanied by its acceptance criteria. Acceptance
criteria or tests are used to confirm that a story is completed, fully
implemented, and works as expected \citep{Cohn2004}. They also provide more
details of a user story. The tests for the requirements described in this
appendix were a part of the prototype implementation and testing process.

The purpose of each of the components has already been described in Chapter
\ref{cha:model}. The following sections of this appendix outline the
requirements of the implemented components in the form of user stories.

\section{Version Control Elements}

\begin{center} \small
    \tablefirsthead{
     \hline
     \multicolumn{2}{|l|}{\textbf{Component:} Version Control Elements} \\
     \hline}
    \tablehead{
     \hline
     \multicolumn{2}{|l|}{\small\sl continued from previous page}\\
     \hline
     \multicolumn{2}{|l|}{\textbf{Component:} Version Control Elements} \\
     \hline}
    \tabletail{
     \hline
     \multicolumn{2}{|r|}{\small\sl continued on next page}\\
     \hline}
    \tablelasttail{\hline}
	\topcaption{Implemented requirements for version control elements}
    \begin{supertabular}{|l p{10cm}|}
     1 & Be able to create new version of the page \\ 
	   & Acceptance Criteria:  
	     \begin{itemize}[noitemsep,nolistsep]
	        \item Text
	        \item Text
	     	\item Text
	     \end{itemize} \\ \hline
     2.1 & Be able to delete any version of the page  \\ 
     	 & Acceptance Criteria:  
	       \begin{itemize}[noitemsep,nolistsep]
	         \item Text
	         \item Text
	         \item Text
	       \end{itemize} \\ \hline
     2.2 & Delete all versions if the page is deleted completely \\ 
         & Acceptance Criteria:  
	       \begin{itemize}[noitemsep,nolistsep]
	         \item Text
	         \item Text
	         \item Text
	       \end{itemize} \\ \hline
     3.1 & Be able to edit page names of any version of the page \\ \hline 
     3.2 & Be able to edit page content of any version of the page \\
     \hline 
     4.1 & Be able to share a specific version of the page with other users \\
     \hline 
     4.2 & Be able to share all version of the page with other users \\
     \hline 
     5 & Be able to navigate between page versions  \\ \hline
     6 & Be able to leave feedback for any version of the page  \\ \hline
    \end{supertabular}
    \label{tab:req1}
\end{center} 

\section{Concept Mapping Module}

\begin{center} \small
    \tablefirsthead{
     \hline
     \multicolumn{2}{|l|}{\textbf{Component:} Concept Mapping} \\
     \hline}
    \tablehead{
     \hline
     \multicolumn{2}{|l|}{\small\sl continued from previous page}\\
     \hline
     \multicolumn{2}{|l|}{\textbf{Component:} Concept Mapping} \\
     \hline}
    \tabletail{
     \hline
     \multicolumn{2}{|r|}{\small\sl continued on next page}\\
     \hline}
    \tablelasttail{\hline}
	\topcaption{Implemented requirements for concept mapping module}
    \begin{supertabular}{|l p{10cm}|}
     1 & Be able to create a concept map \\ \hline
     2.1 & Be able to add concepts to concept map  \\ \hline
     2.2 & Be able to add definitions to the concepts  \\ \hline
     3 & Be able to edit concepts and definitions  \\ \hline
     4 & Be able to delete concepts and definitions  \\ \hline
     5 & Be able to change type of a concept map element  \\ \hline
     6 & Be able to add examples to the map from the \ep~repository \\ \hline
     G8 & Be able to view added examples  \\ \hline
     G9 & Be able to share a concept map  \\ \hline
    \end{supertabular}
    \label{tab:req2}
\end{center}

\section{Artifacts' Fragments Extraction}

\begin{center} \small
    \tablefirsthead{
     \hline
     \multicolumn{2}{|l|}{\textbf{Component:} Artifacts' Fragments Extraction} \\
     \hline}
    \tablehead{
     \hline
     \multicolumn{2}{|l|}{\small\sl continued from previous page}\\
     \hline
     \multicolumn{2}{|l|}{\textbf{Component:} Artifacts' Fragments Extraction} \\
     \hline}
    \tabletail{
     \hline
     \multicolumn{2}{|r|}{\small\sl continued on next page}\\
     \hline}
    \tablelasttail{\hline}
	\topcaption{Implemented requirements for artifact fragments extraction}
    \begin{supertabular}{|l p{10cm}|}
     G1 & Text \\ \hline
     G2 & Text  \\ \hline
     G3 & Text  \\ \hline
     G4 & Text  \\ \hline
     G5 & Text  \\ \hline
     G6 & Text  \\ \hline
     G7 & Text  \\ \hline
     G8 & Text  \\ \hline
     G9 & Text  \\ \hline
    \end{supertabular}
    \label{tab:req3}
\end{center}

\section{Learning Progress Tracking}

\begin{center} \small
    \tablefirsthead{
     \hline
     \multicolumn{2}{|l|}{\textbf{Component:} Progress Tracking} \\
     \hline}
    \tablehead{
     \hline
     \multicolumn{2}{|l|}{\small\sl continued from previous page}\\
     \hline
     \multicolumn{2}{|l|}{\textbf{Component:} Progress Tracking} \\
     \hline}
    \tabletail{
     \hline
     \multicolumn{2}{|r|}{\small\sl continued on next page}\\
     \hline}
    \tablelasttail{\hline}
	\topcaption{Implemented requirements for progress tracking}
    \begin{supertabular}{|l p{10cm}|}
     G1 & Text \\ \hline
     G2 & Text  \\ \hline
     G3 & Text  \\ \hline
     G4 & Text  \\ \hline
     G5 & Text  \\ \hline
     G6 & Text  \\ \hline
     G7 & Text  \\ \hline
     G8 & Text  \\ \hline
     G9 & Text  \\ \hline
    \end{supertabular}
    \label{tab:req4}
\end{center}

\section{Shared Resources Management}

\begin{center} \small
    \tablefirsthead{
     \hline
     \multicolumn{2}{|l|}{\textbf{Component:} Advanced Sharing Options} \\
     \hline}
    \tablehead{
     \hline
     \multicolumn{2}{|l|}{\small\sl continued from previous page}\\
     \hline
     \multicolumn{2}{|l|}{\textbf{Component:} Advanced Sharing Options} \\
     \hline}
    \tabletail{
     \hline
     \multicolumn{2}{|r|}{\small\sl continued on next page}\\
     \hline}
    \tablelasttail{\hline}
	\topcaption{Implemented requirements for advanced sharing options}
    \begin{supertabular}{|l p{10cm}|}
     G1 & Text \\ \hline
     G2 & Text  \\ \hline
     G3 & Text  \\ \hline
     G4 & Text  \\ \hline
     G5 & Text  \\ \hline
     G6 & Text  \\ \hline
     G7 & Text  \\ \hline
     G8 & Text  \\ \hline
     G9 & Text  \\ \hline
    \end{supertabular}
    \label{tab:req5}
\end{center}