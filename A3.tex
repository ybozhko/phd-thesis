\chapter{Component Requirements \label{app:specification}}

\section{User Stories}

User stories were used in this thesis for representing requirements of the
implemented features. They are an easy way of stating functional requirements
and communicating them in a way that can be understood by end users
\citep{Crispin2003}.

Each user story tells a story about user interacting with the system. It is not
necessary to capture everything at once in user stories. Usually, some most
valuable aspects of the system features are described.

A standard way of capturing user stories is in the following form
\citep{Cohn2004,Coplien2010}:

\shortquote{As a [role/actor] I can [function/action] so that
[rationale/achievement].}

For example,

\shortquote{As a student I can perform a search on concept maps to help me find
the concepts.}

Each user story can be accompanied by its acceptance criteria. Acceptance
criteria or tests are used to confirm that a story is completed, fully
implemented, and works as expected \citep{Cohn2004}. They also provide more
details of a user story. The tests for the requirements described in this
appendix were a part of the prototype implementation and testing process.

The purpose of each of the components has already been described in Chapter
\ref{cha:model}. The following sections of this appendix outline the
requirements of the implemented components in the form of user stories.

\section{Version Control Elements}

\begin{center} \small
    \tablefirsthead{
     \hline
     \multicolumn{2}{|l|}{\textbf{Component:} Version Control Elements} \\
     \hline}
    \tablehead{
     \hline
     \multicolumn{2}{|l|}{\small\sl continued from previous page}\\
     \hline
     \multicolumn{2}{|l|}{\textbf{Component:} Version Control Elements} \\
     \hline}
    \tabletail{
     \hline
     \multicolumn{2}{|r|}{\small\sl continued on next page}\\
     \hline}
    \tablelasttail{\hline}
	\topcaption{Implemented requirements for version control elements}
    \begin{supertabular}{|l p{12cm}|}
     1.01 & \textit{User can create a new version of a page} \\ 
	   & Acceptance Criteria:
	     \begin{itemize}[nosep,label=--]
	        \item A page has one (first) version by default
	        \item Page is not a system type or group type
	        \item Page is not submitted for review
	     	\item Name for the new version of the page satisfies the requirements for
	     	page names 
	     	\item Content of the latest version is copied to the new version
	     \end{itemize} \\ \hline
     1.02 & \textit{User can remove a version of a page}  \\ 
     	 & Acceptance Criteria:  
	       \begin{itemize}[nosep,label=--]
	         \item Only one version at a time can be removed
	         \item If all versions are removed, the page is removed as well
	         \item Verify that links between versions are properly established when
	         a version that has next version is removed
	       \end{itemize} \\ \hline
     1.03 & \textit{User can remove a page} \\ 
         & Acceptance Criteria:  
	       \begin{itemize}[nosep,label=--]
	         \item Verify that all versions are removed if the page is removed
	         completely \end{itemize} \\ \hline
     1.04 & \textit{User can edit any version of the page} \\ 
         & Acceptance Criteria:  
	       \begin{itemize}[nosep,label=--]
	         \item User can edit page names of any version of the page
	         \item User can edit page content of any version of the page
	       \end{itemize} \\   
     \hline 
     1.05 & \textit{User can share a page with other users} \\
          & Acceptance Criteria:  
	       \begin{itemize}[nosep,label=--]
	         \item User can share a specific version of the page
	         \item User can share all versions of the page
	         \item Verify that other users have proper permissions to view the page
	       \end{itemize} \\    
     \hline    
     1.06 & \textit{User can navigate between page versions}  \\ 
          & Acceptance Criteria:  
	       \begin{itemize}[nosep,label=--]
	         \item User can navigate to a specific version 
	         \item User can navigate through the versions using a navigation menu
	       \end{itemize} \\    
     \hline 
     1.07 & \textit{User can leave feedback on the page}  \\  
          & Acceptance Criteria:  
	       \begin{itemize}[nosep,label=--]
	         \item User can leave feedback on any version of the page
	       \end{itemize} \\    
     \hline   
    \end{supertabular}
    \label{tab:req1}
\end{center} 

\section{Concept Mapping Module}

\begin{center} \small
    \tablefirsthead{
     \hline
     \multicolumn{2}{|l|}{\textbf{Component:} Concept Mapping} \\
     \hline}
    \tablehead{
     \hline
     \multicolumn{2}{|l|}{\small\sl continued from previous page}\\
     \hline
     \multicolumn{2}{|l|}{\textbf{Component:} Concept Mapping} \\
     \hline}
    \tabletail{
     \hline
     \multicolumn{2}{|r|}{\small\sl continued on next page}\\
     \hline}
    \tablelasttail{\hline}
	\topcaption{Implemented requirements for concept mapping module}
    \begin{supertabular}{|l p{12cm}|}
     2.01 & \textit{User can create a concept map} \\ 
	   & Acceptance Criteria:  
	     \begin{itemize}[nosep,label=--]
	        \item User cannot create a map without completing all the mandatory
	        fields of the form
	        \item An additional custom timeframe can be selected from the list
	     \end{itemize} \\ \hline
     2.02 & \textit{User can load a concept map in a diagram perspective}  \\ 
     	 & Acceptance Criteria:  
	       \begin{itemize}[nosep,label=--]
	         \item Nodes of different type have different colours
	         \item Verify that the hierarchy is displayed properly
	         \item Number of examples attached to each definition node is displayed
	         \item User can switch to another perspective at any time 
	         \item User expand or collapse any node
	         \item User expand or collapse an entire diagram
	       \end{itemize} \\ \hline	     
     2.03 & \textit{User can add nodes to concept map}  \\ 
     	 & Acceptance Criteria:  
	       \begin{itemize}[nosep,label=--]
	         \item User can add concept nodes
	         \item User can add definition nodes
	         \item Verify that concept-type node cannot be added to the
	         definition-type node
	         \item Node is created if the mandatory field of the form is completed
	         \item Verify that the page is dynamically reloading to display changes
	       \end{itemize} \\ \hline
     2.04 & \textit{User can edit nodes} \\ 
     	 & Acceptance Criteria:  
	       \begin{itemize}[nosep,label=--]
	         \item User can rename a concept node 
	         \item User can rename a definition node
	         \item Verify that the page is dynamically reloading to display changes
	       \end{itemize} \\ \hline
     2.05 & \textit{User can delete nodes} \\   
     	 & Acceptance Criteria:  
	       \begin{itemize}[nosep,label=--]
	         \item User can delete a concept node 
	         \item User can delete a definition node
			 \item Verify that the key concept node cannot be removed
	         \item Verify that all sub-nodes are removed together with the parent
	         node
	         \item Verify that all examples are detached
	         \item Verify that the page is dynamically reloading to display changes 
	       \end{itemize} \\ \hline
     2.06 & \textit{User can change type of a concept map element} \\   
     	 & Acceptance Criteria:  
	       \begin{itemize}[nosep,label=--]
	         \item A definition-type node can be changed to a concept-type node
	         \item A concept-type node can be changed to a definition-type node
	         \item Verify that all examples are detached from definition-type node
	         if its type is changed
	         \item Verify that the page is dynamically reloading to display changes
	       \end{itemize} \\ \hline
     2.07 & \textit{User can create examples from artifacts from in an
     \ep~repository}   \\
     	 & Acceptance Criteria:  
	       \begin{itemize}[nosep,label=--]
	         \item User cannot create an example without completing all the
	         mandatory fields of the form
	         \item By default, a new example is listed as a free fragment
	         \item By default, a date of an example is a date of an artifact 
	       \end{itemize} \\ \hline
     2.08 & \textit{User can add examples to the map from a diagram perspective} \\
     	 & Acceptance Criteria:  
	       \begin{itemize}[nosep,label=--]
	         \item User can load a list of free fragments when adding example
	         \item User can select one or more free fragments as examples for the
	         concept map
	         \item Verify that examples can be added to a definition-type node only
	         \item Verify that the page is dynamically reloading to display changes
	       \end{itemize} \\ \hline 
     2.09 & \textit{User can view examples}  \\ 
     	 & Acceptance Criteria:  
	       \begin{itemize}[nosep,label=--]
	         \item Examples have to be listed in descending date order
	         \item Verify that correct examples are displayed across a hierarchy 
	         \item Examples list can be collapsed/expanded
	       \end{itemize} \\ \hline
     2.10 & \textit{User can manage examples}  \\ 
     	 & Acceptance Criteria:  
	       \begin{itemize}[nosep,label=--]
	         \item User can edit examples
	         \item User can delete examples
	         \item User can copy examples
	         \item Verify that all content is copied to a new example
	         \item Verify that an example is copied as a free fragment
	       \end{itemize} \\ \hline	 
     2.11 & \textit{User can share a concept map}  \\ 
     	 & Acceptance Criteria:  
	       \begin{itemize}[nosep,label=--]
	         \item Map can be shared through e-mail, user-name, group-name, and
	         secret URL
	         \item User can allow/restrict comments on map and/or examples
	       \end{itemize} \\ \hline
     2.12 & \textit{User can perform a search on a concept map}  \\ 
     	 & Acceptance Criteria:  
	       \begin{itemize}[nosep,label=--]
	         \item User can find a node through search field
	         \item Search result is highlighted on the map
	         \item If there are more than one result, they are displayed in turns
	         when user selects search again
	       \end{itemize} \\ \hline
    \end{supertabular}
    \label{tab:req2}
\end{center}

\section{Artifact Fragments Extraction}

\begin{center} \small
    \tablefirsthead{
     \hline
     \multicolumn{2}{|l|}{\textbf{Component:} Artifact Fragments Extraction} \\
     \hline}
    \tablehead{
     \hline
     \multicolumn{2}{|l|}{\small\sl continued from previous page}\\
     \hline
     \multicolumn{2}{|l|}{\textbf{Component:} Artifact Fragments Extraction} \\
     \hline}
    \tabletail{
     \hline
     \multicolumn{2}{|r|}{\small\sl continued on next page}\\
     \hline}
    \tablelasttail{\hline}
	\topcaption{Implemented requirements for artifact fragments extraction}
    \begin{supertabular}{|l p{12cm}|}
     3.01 & \textit{User can see fragments of a file} \\ 
	   & Acceptance Criteria:  
	     \begin{itemize}[nosep,label=--]
	        \item Number of fragments for each file is displayed in repository
	        manager
	        \item User can see a list of all fragments with their details for each
	        file
	     \end{itemize} \\ \hline
     3.02 & \textit{User can extract a fragment from image file} \\ 
	   & Acceptance Criteria:  
	     \begin{itemize}[nosep,label=--]
	        \item Image file has to have JPEG, PNG, or GIF extension
	        \item An appropriate part of an image is selected
	        \item Entire image is saved as a fragment if no part is selected 
	     \end{itemize} \\ \hline
     3.03 & \textit{User can extract a fragment from video file}  \\ 
     	 & Acceptance Criteria:  
	       \begin{itemize}[nosep,label=--]
	         \item Video file has to have OGV, MP4, 3GP, WEBM extension
	         \item Start time of the fragment has to be specified
	         \item End time of the fragment has to be specified
	         \item End time cannot be longer than the entire video file
	       \end{itemize} \\ \hline
     3.04 & \textit{User can extract a fragment from text file} \\ 
         & Acceptance Criteria:  
	       \begin{itemize}[nosep,label=--]
	         \item Text file has to have TXT extension
	         \item An appropriate text fragment is selected
	       \end{itemize} \\ \hline
     3.05 & \textit{User can extract a fragment from blog} \\ 
         & Acceptance Criteria:  
	       \begin{itemize}[nosep,label=--]
	         \item Verify that blogposts are not drafts
	         \item More than one blogpost from the list can be selected
	         \item At least one blogpost from the list has to be selected
	       \end{itemize} \\   
     \hline 
     3.06 & \textit{User can extract a fragment from non-supported files} \\
          & Acceptance Criteria:  
	       \begin{itemize}[nosep,label=--]
	         \item User gets the warning that this file type is not supported for
	         fragment extraction
	         \item An entire file is added to the fragment for download 
	       \end{itemize} \\    
     \hline 
     3.07 & \textit{User can extract a fragment from a bookmark}  \\ 
          & Acceptance Criteria:  
	       \begin{itemize}[nosep,label=--]
	         \item An entire URL is selected as a fragment
	         \item User has to specify last access date to a URL
	       \end{itemize} \\    
     \hline 
    \end{supertabular}
    \label{tab:req3}
\end{center}
\newpage
\section{Learning Progress Tracking}

\begin{center} \small
    \tablefirsthead{
     \hline
     \multicolumn{2}{|l|}{\textbf{Component:} Progress Tracking} \\
     \hline}
    \tablehead{
     \hline
     \multicolumn{2}{|l|}{\small\sl continued from previous page}\\
     \hline
     \multicolumn{2}{|l|}{\textbf{Component:} Progress Tracking} \\
     \hline}
    \tabletail{
     \hline
     \multicolumn{2}{|r|}{\small\sl continued on next page}\\
     \hline}
    \tablelasttail{\hline}
	\topcaption{Implemented requirements for progress tracking}
    \begin{supertabular}{|l p{12cm}|}
     4.01 & \textit{User can load a concept map in a timeline perspective} \\ 
	   & Acceptance Criteria:  
	     \begin{itemize}[nosep,label=--]
	        \item Default timeframe view is year-month
	        \item Default available timeframes are year and year-month
	        \item Verify that items are loaded in correct order
	     	\item User can scroll the timeline if it over-floats the web page
	     \end{itemize} \\ \hline
     4.02 & \textit{User can apply custom filter to the timeline}  \\ 
     	 & Acceptance Criteria:  
	       \begin{itemize}[nosep,label=--]
	         \item Default concept level is key concept
	         \item User can select any concept from the concept hierarchy
	         \item When a concept is selected, a timeline is dynamically reloaded
	         to display examples related to the selected concept and its
	         sub-concepts 
	         \item Verify that filter is applied to the concepts only and is not
	         changing selected timefame 
	       \end{itemize} \\ \hline
     4.03 & \textit{User can see examples on the timeline} \\ 
         & Acceptance Criteria:  
	       \begin{itemize}[nosep,label=--]
	         \item User can select an example for displaying
	         \item User can display only one example at a time
	       \end{itemize} \\ \hline
     4.04 & \textit{User can create a custom timeframe} \\ 
         & Acceptance Criteria:  
	       \begin{itemize}[nosep,label=--]
	         \item User cannot create a custom timeframe without completing all the
	         mandatory fields of the form
	         \item Verify that timeframes are properly formatted and parsed
	       \end{itemize} \\   
     \hline 
     4.05 & \textit{User can manage custom timeframes} \\ 
         & Acceptance Criteria:  
	       \begin{itemize}[nosep,label=--]
	         \item User can see the list of concept maps to which timeframes
	         belongs
	         \item User can edit a custom timeframe
	         \item User can delete a custom timeframe
	         \item If timeframe is removed, it is detached from the concept map
	       \end{itemize} \\   
     \hline 
     4.06 & \textit{User can apply a timeframe to the timeline} \\
          & Acceptance Criteria:  
	       \begin{itemize}[nosep,label=--]
	         \item Custom timeframes are added to the list of available timeframes
	         \item User can select a custom timeframe
	         \item When a custom timeframe is selected, a timeline is dynamically
	         reloaded to display examples grouped according to the custom timeframe
	         \item Verify that filter is applied to the timeframes only and is not
	         changing selected concept 
	       \end{itemize} \\    
     \hline 
    \end{supertabular}
    \label{tab:req4}
\end{center}

\section{Shared Resources Management}

\begin{table}[htb]
\setlength{\abovecaptionskip}{0pt}
\caption{Implemented requirements for shared resources management}
\begin{center} \small
	\begin{tabular}{|l p{12cm}|}
     \hline
     \multicolumn{2}{|l|}{\textbf{Component:} Shared Resources Management} \\
     \hline
     5.01 & \textit{User can see a page access/sharing history} \\ 
	   & Acceptance Criteria:  
	     \begin{itemize}[nosep,label=--]
	        \item User can access sharing history through a page settings
	        \item User actions are recorded every time a page access settings are
	        changed
	     	\item User cannot remove records from the sharing history  
	     \end{itemize} \\ \hline
     5.02 & \textit{User can re-share pages through the sharing history}  \\ 
     	 & Acceptance Criteria:  
	       \begin{itemize}[nosep,label=--]
	         \item User can select re-share option from the sharing history
	         \item By default re-sharing starts from the current date 
	         \item By default re-sharing end field is empty
	         \item User can specify other access dates
	       \end{itemize} \\ \hline
     5.03 & \textit{User can control a level of feedback provided for concept
     maps}   \\
         & Acceptance Criteria:  
	       \begin{itemize}[nosep,label=--]
	         \item User can allow/restrict comments on a concept map
	         \item User can allow/restrict comments on examples
	       \end{itemize} \\ \hline
     5.04 & \textit{Users are notified about access expiry} \\ 
         & Acceptance Criteria:  
	       \begin{itemize}[nosep,label=--]
	         \item User can set up notification date
	         \item User can select automatic notification
	         \item Verify that system sends a system message to the users
	         registered in the system
		     \item Verify that system sends an email to the users not registered in
		     the system 
		   \end{itemize} \\ \hline
    \end{tabular}
    \label{tab:req5}
\end{center}
\end{table}